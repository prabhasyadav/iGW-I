%% Generated by Sphinx.
\def\sphinxdocclass{report}
\documentclass[letterpaper,10pt,english]{sphinxmanual}
\ifdefined\pdfpxdimen
   \let\sphinxpxdimen\pdfpxdimen\else\newdimen\sphinxpxdimen
\fi \sphinxpxdimen=.75bp\relax

\PassOptionsToPackage{warn}{textcomp}
\usepackage[utf8]{inputenc}
\ifdefined\DeclareUnicodeCharacter
% support both utf8 and utf8x syntaxes
  \ifdefined\DeclareUnicodeCharacterAsOptional
    \def\sphinxDUC#1{\DeclareUnicodeCharacter{"#1}}
  \else
    \let\sphinxDUC\DeclareUnicodeCharacter
  \fi
  \sphinxDUC{00A0}{\nobreakspace}
  \sphinxDUC{2500}{\sphinxunichar{2500}}
  \sphinxDUC{2502}{\sphinxunichar{2502}}
  \sphinxDUC{2514}{\sphinxunichar{2514}}
  \sphinxDUC{251C}{\sphinxunichar{251C}}
  \sphinxDUC{2572}{\textbackslash}
\fi
\usepackage{cmap}
\usepackage[T1]{fontenc}
\usepackage{amsmath,amssymb,amstext}
\usepackage{babel}



\usepackage{times}
\expandafter\ifx\csname T@LGR\endcsname\relax
\else
% LGR was declared as font encoding
  \substitutefont{LGR}{\rmdefault}{cmr}
  \substitutefont{LGR}{\sfdefault}{cmss}
  \substitutefont{LGR}{\ttdefault}{cmtt}
\fi
\expandafter\ifx\csname T@X2\endcsname\relax
  \expandafter\ifx\csname T@T2A\endcsname\relax
  \else
  % T2A was declared as font encoding
    \substitutefont{T2A}{\rmdefault}{cmr}
    \substitutefont{T2A}{\sfdefault}{cmss}
    \substitutefont{T2A}{\ttdefault}{cmtt}
  \fi
\else
% X2 was declared as font encoding
  \substitutefont{X2}{\rmdefault}{cmr}
  \substitutefont{X2}{\sfdefault}{cmss}
  \substitutefont{X2}{\ttdefault}{cmtt}
\fi


\usepackage[Bjarne]{fncychap}
\usepackage[,numfigreset=1,mathnumfig]{sphinx}

\fvset{fontsize=\small}
\usepackage{geometry}


% Include hyperref last.
\usepackage{hyperref}
% Fix anchor placement for figures with captions.
\usepackage{hypcap}% it must be loaded after hyperref.
% Set up styles of URL: it should be placed after hyperref.
\urlstyle{same}

\addto\captionsenglish{\renewcommand{\contentsname}{Background}}

\usepackage{sphinxmessages}




\title{Groundwater I}
\date{Nov 06, 2020}
\release{}
\author{P.\@{} K.\@{} Yadav, T.\@{} Reimann and many others}
\newcommand{\sphinxlogo}{\vbox{}}
\renewcommand{\releasename}{}
\makeindex
\begin{document}

\pagestyle{empty}
\sphinxmaketitle
\pagestyle{plain}
\sphinxtableofcontents
\pagestyle{normal}
\phantomsection\label{\detokenize{intro::doc}}


The site provides an interactive JUPYTER book with contents typical of a \sphinxstyleemphasis{introductory} groundwater course taught at higher UG level or the early PG level at universities.

The contents/structure provided here are mostly from those developed and lectured by \sphinxstylestrong{Prof. Rudolf Liedl} at \DUrole{xref,myst}{TU Dresden}.

The contents are geared towards \sphinxstylestrong{learning through computing}. The computing part is entirely based on \sphinxstyleemphasis{Python} programming language. Previous programming/coding experiences is not required or expected to gain from the provided contents.

The contents of the book are divided into:
\begin{enumerate}
\sphinxsetlistlabels{\arabic}{enumi}{enumii}{}{.}%
\item {} 
Lecture Parts

\item {} 
Tutorial Parts

\item {} 
Self\sphinxhyphen{}learning tools

\end{enumerate}

The \sphinxstyleemphasis{lecture} parts are combination of \sphinxstylestrong{texts} and \sphinxstylestrong{simpler} numerical examples. Only minimum \sphinxstyleemphasis{Python} codes are available on this part.
The \sphinxstyleemphasis{tutorial} are mostly \sphinxstylestrong{numerical} examples. This aims at teaching also the basic of Python coding. The focus remains on illustrating lecture contents. \sphinxstyleemphasis{Self\sphinxhyphen{}learning tools} are interactive tools that supports the lecture and tutorial components and enhances understanding. The codes of the tools can require higher knowledge in Python programming. Therefore, codes are hidden.

All codes and contents provided in this interactive book are licensed under \sphinxhref{https://creativecommons.org/licenses/by/4.0/}{Creative Commons BY 4.0}
Codes are available at this \sphinxhref{https://github.com/prabhasyadav/Groundwater-I}{GitHUB repository}
\begin{quote}

\sphinxstylestrong{The development of the book in based on the \sphinxstylestrong{wonderful} work of the \sphinxhref{https://jupyterbook.org/intro.html}{JUPYTER Book Team}}
\end{quote}


\chapter{Main contributors of this version of the book}
\label{\detokenize{intro:main-contributors-of-this-version-of-the-book}}
The contents are developed by (not in any order):
\begin{itemize}
\item {} 
Prof. Rudolf Liedl (TU Dresden)

\item {} 
Prof Charles Werth (Uni\sphinxhyphen{}Texas Austin, US)

\item {} 
Prof. B. R. Chahar (Indian Institute of Technology Delhi, Delhi)

\item {} 
Dr.rer. nat. Prabhas K Yadav (TU Dresden)

\item {} 
Dr. Ing. Thomas Reimann (TU Dresden)

\item {} 
M.Sc. Hanieh Mehrdad (Student Assistant/Numerical contents\sphinxhyphen{} TU Dresden)

\item {} 
Anton Köhler (Student Assistant/Model contents \sphinxhyphen{} TU Dresden)

\item {} 
Abiral Poudel (Student Assistant/CS\sphinxhyphen{}IT support  \sphinxhyphen{} TU Dresden)

\item {} 
Sophie Pförtner (Student Assistant/Numerical content \sphinxhyphen{} TU Dresden)

\item {} 
Anne  Pförtner (Student Assistant/Numerical content \sphinxhyphen{} TU Dresden)

\item {} 
Alexander Oy (Student Assistant/Numerical content and CS\sphinxhyphen{}IT Support \sphinxhyphen{} TU Dresden)

\end{itemize}


\chapter{Acknowledgments}
\label{\detokenize{intro:acknowledgments}}
This work is partly supported:
\begin{enumerate}
\sphinxsetlistlabels{\arabic}{enumi}{enumii}{}{.}%
\item {} 
The \sphinxhref{https://tu-dresden.de/tu-dresden/organisation/rektorat/prorektor-bildung-und-internationales/zill/e-learning/multimediafonds}{Multimediafonds, TU\sphinxhyphen{}Dresden}

\item {} 
The \sphinxhref{https://www.dfg.de/}{ESTIMATE, project, DFG}

\end{enumerate}


\section{About this Groundwater Course and Contents}
\label{\detokenize{contents/background/00_general:about-this-groundwater-course-and-contents}}\label{\detokenize{contents/background/00_general::doc}}
The contents provided in this website that forms an interactive book, is based on the Groundwater course created, maintained and lectured by
\sphinxstylestrong{Prof. Rudolf Liedl} for over 15 years at the \sphinxhref{https://tu-dresden.de/bu/umwelt/hydro/igw}{Institute of Groundwater Manangement} of \DUrole{xref,myst}{TU Dresden}.

The course structure and contents have been used as an \sphinxstyleemphasis{Introductory} course on Groundwater for M.Sc. level students coming from multiple academic backgrounds.

The \sphinxstylestrong{texts} provided in this interactive book is mostly based on the Prof. Liedl’s lecture slides and the \sphinxstylestrong{interactive codes} are conversion from MS Excel® spreadsheet to \sphinxstyleemphasis{Python} codes.
\begin{quote}

\sphinxstylestrong{This interactive web\sphinxhyphen{}book is dedicated to Prof. Rudolf Liedl efforts to teach, train and inspire us and other students over the years.}
\end{quote}


\subsection{Basic contents structure}
\label{\detokenize{contents/background/00_general:basic-contents-structure}}
The contents of this interactive book can be broadly divided in the following three groups:
\begin{enumerate}
\sphinxsetlistlabels{\arabic}{enumi}{enumii}{}{.}%
\item {} 
Aquifer properties and groundwater flow

\item {} 
Transport in groundwater

\item {} 
Groundwater Modelling

\end{enumerate}

The first group (\sphinxstylestrong{aquifer properties and groundwater flow}) makes the core of this course. Here the very basic of groundwater, subsurface structure, properties that quantify groundwater mass and volume budgets, and the flow and other dynamics processes, e.g., abstraction using wells.

The \sphinxstylestrong{transport in groundwater} topics focus on the quality aspects of groundwater. In particular, transport equations with and without inclusion of chemical reactions (e.g., sorption, decay) are considered. Eventually, few analytical solutions of transport problems are discussed.

The \sphinxstylestrong{Groundwater Modelling} is for introducing the realm of computer modelling of groundwater and transport. Fundamentals of mathematical modelling, e.g., finite different methods, is introduced. The focus remains towards eventual use of MODFLOW (Flopy and Modelmuse interface), which is introduced in a short tutorial form.


\subsection{What do you need in this course?}
\label{\detokenize{contents/background/00_general:what-do-you-need-in-this-course}}
You will need the following:
\begin{itemize}
\item {} 
Laptop/Smartphone/tablet more convenient with internet connection

\item {} 
Recommended is installed JUPYTER interface.

\end{itemize}

The course contents can be received as:
\begin{itemize}
\item {} 
Lectures: Texts reading and interactive manipulation of codes/problems. Short questions (self\sphinxhyphen{}test) should be used to check the understanding of the contents.

\item {} 
Tutorials: Should be understand by manipulating the existing codes. The tutorials should be then independently solved using \sphinxstyleemphasis{JUPYTER} interface (Mybinder or Personal system)

\item {} 
Additional tools: The course provides several simulation tools\sphinxhyphen{} e.g., sieve analysis, effective conductivity, Advection\sphinxhyphen{}dispersion etc. These tools (also \sphinxstyleemphasis{Python}) provide high\sphinxhyphen{}level interactivity. These tools should be used to enhance learning.

\item {} 
Question banks/exam questions should be used to self\sphinxhyphen{}check the level of understanding.

\end{itemize}


\section{Python Programming language}
\label{\detokenize{contents/background/01_python:python-programming-language}}\label{\detokenize{contents/background/01_python::doc}}

\subsection{What is it?}
\label{\detokenize{contents/background/01_python:what-is-it}}
Python is an open\sphinxhyphen{}source interpreted, high\sphinxhyphen{}level, general\sphinxhyphen{}purpose programming language. This means:
\begin{itemize}
\item {} 
Interpreted/high\sphinxhyphen{}level language: This makes we avoid the nuances of fundamental coding as done by computer programmers/engineers.

\item {} 
General purpose programming language and \sphinxstyleemphasis{Open\sphinxhyphen{}source} ecosystem: This means it is extensible. Already over 200,000 Python packages are available (\sphinxhref{https://pypi.org/}{Check Here}). Also, it means being \sphinxstylestrong{free} of cost.

\item {} 
For Groundwater: We can use Python packages such as \sphinxstylestrong{Numpy} (for numerical computing), \sphinxstylestrong{Scipy} (for scientific computing), \sphinxstylestrong{Sympy} (for symbolic computing), \sphinxstylestrong{Matplotlib} (for plotting) etc. for our computing and modelling in the course.

\end{itemize}


\subsection{A bit of history of Python Programming language}
\label{\detokenize{contents/background/01_python:a-bit-of-history-of-python-programming-language}}
Python is now over 15 years old programming language. Its development can be traced from:
\begin{itemize}
\item {} 
\sphinxstylestrong{Guido van Rossum} began developing Python in 1980 at Centrum Wiskunde \& Informatica (CWI), the Netherlands. Its implementation (\sphinxstyleemphasis{Python v.1}) was released in 1994.

\item {} 
\sphinxstyleemphasis{Python 2.0}, released in 2000 became one of the most used general purpose programming language. Python 2.0 is now being replaced by \sphinxstyleemphasis{Python 3.0} (from 2020).

\item {} 
\sphinxstyleemphasis{Python 3.0} will be used in our class. It is \sphinxstyleemphasis{not} 100\% compatible with earlier versions of \sphinxstyleemphasis{Python.}

\item {} 
\sphinxstyleemphasis{Python} name comes from the British comedy group Monty Python (Van Rossum enjoyed their show). The official \sphinxstyleemphasis{Python} documentation \sphinxhref{https://www.python.org/doc/}{(Check here)} also contains various references to Monty Python routines.

\end{itemize}


\subsection{Why use Python Programming Language}
\label{\detokenize{contents/background/01_python:why-use-python-programming-language}}
Many reasons but to put a few points here:
\begin{itemize}
\item {} 
\sphinxstyleemphasis{Python} is a common tool among engineers, experts and researchers at universities and industry.

\item {} 
\sphinxstyleemphasis{Python} is system independent, therefore it is highly portable. Beside, it is a versatile (multi\sphinxhyphen{}purpose) language.

\item {} 
\sphinxstyleemphasis{Python} is incredibly flexible and can be adapted to specific local needs using enormous number of \sphinxstylestrong{PACKAGES}. Beside, it can easily interface with other languages e.g., C++, Java.

\item {} 
\sphinxstyleemphasis{Python} is under incredibly active development, improving greatly, and supported wildly by both professional and academic developers.

\end{itemize}

\begin{sphinxVerbatim}[commandchars=\\\{\}]
\PYG{n}{p1} \PYG{o}{=} \PYG{n}{pn}\PYG{o}{.}\PYG{n}{pane}\PYG{o}{.}\PYG{n}{Markdown}\PYG{p}{(}\PYG{l+s+s2}{\PYGZdq{}\PYGZdq{}\PYGZdq{}}
\PYG{l+s+s2}{\PYGZsh{}\PYGZsh{}\PYGZsh{} Python popularity}
\PYG{l+s+s2}{\PYGZdq{}\PYGZdq{}\PYGZdq{}}\PYG{p}{)}

\PYG{n}{p2} \PYG{o}{=} \PYG{n}{pn}\PYG{o}{.}\PYG{n}{pane}\PYG{o}{.}\PYG{n}{PNG}\PYG{p}{(}\PYG{l+s+s2}{\PYGZdq{}}\PYG{l+s+s2}{images/bg1\PYGZus{}f1.png}\PYG{l+s+s2}{\PYGZdq{}}\PYG{p}{,} \PYG{n}{width}\PYG{o}{=}\PYG{l+m+mi}{500}\PYG{p}{)} 

\PYG{n}{p3}\PYG{o}{=} \PYG{n}{pn}\PYG{o}{.}\PYG{n}{pane}\PYG{o}{.}\PYG{n}{Markdown}\PYG{p}{(}\PYG{l+s+s2}{\PYGZdq{}\PYGZdq{}\PYGZdq{}}\PYG{l+s+s2}{\PYGZlt{}/br\PYGZgt{}\PYGZlt{}/br\PYGZgt{}\PYGZlt{}/br\PYGZgt{}}

\PYG{l+s+s2}{+ \PYGZus{}Python\PYGZus{} has become a mainstream computing language. }
\PYG{l+s+s2}{+ Details of the plot are [here](shorturl.at/htwQ7). }
\PYG{l+s+s2}{+ This all means \PYGZhy{} it is good to learn to code in \PYGZus{}Python\PYGZus{}}
\PYG{l+s+s2}{\PYGZdq{}\PYGZdq{}\PYGZdq{}}\PYG{p}{)}

\PYG{n}{p4} \PYG{o}{=} \PYG{n}{pn}\PYG{o}{.}\PYG{n}{Column}\PYG{p}{(}\PYG{n}{p1}\PYG{p}{,}\PYG{n}{p2}\PYG{p}{)}

\PYG{n}{pn}\PYG{o}{.}\PYG{n}{Row}\PYG{p}{(}\PYG{n}{p4}\PYG{p}{,} \PYG{n}{p3}\PYG{p}{)}
\end{sphinxVerbatim}

\begin{sphinxVerbatim}[commandchars=\\\{\}]
Row
    [0] Column
        [0] Markdown(str)
        [1] PNG(str, width=500)
    [1] Markdown(str)
\end{sphinxVerbatim}


\section{Very basics of Python Programming}
\label{\detokenize{contents/background/01_python:very-basics-of-python-programming}}
\sphinxstyleemphasis{Python} is a very extensive language. To get started we learn the very fundamentals of the language.

\sphinxstylestrong{Fundamentals of Python Programming Language}

\noindent\sphinxincludegraphics[width=600\sphinxpxdimen]{{bg1_f2}.png}

\sphinxstylestrong{Data Types in Python}

\noindent\sphinxincludegraphics[width=600\sphinxpxdimen]{{bg1_f3}.png}

\sphinxstylestrong{Basic operators in Python}

Refer to \sphinxstyleemphasis{Python} documentation for complete description. Python documentation is very extensive and can be obtained from \sphinxhref{https://www.python.org/doc/}{here}

\noindent\sphinxincludegraphics[width=600\sphinxpxdimen]{{bg1_f4}.png}

\sphinxstylestrong{A FUNCTION in Python}

A \sphinxstyleemphasis{function} in a programming provide an ability to develop a reusable code\sphinxhyphen{}block with an option of several operations. This means, a function have input (or a set of input) and provide an (or a set of) output.

\noindent\sphinxincludegraphics[width=600\sphinxpxdimen]{{bg1_f5}.png}

Semicolon (:) in line 1 and \sphinxstyleemphasis{Indentation} after line 1 are required. \sphinxstylestrong{def}, \sphinxstylestrong{return} are \sphinxstyleemphasis{Python} keywords. There are quite few of them.


\section{How much Python Programming should we know?}
\label{\detokenize{contents/background/01_python:how-much-python-programming-should-we-know}}
This is probably the most important question. The \sphinxstylestrong{clear} answer at least for this course is that \sphinxstylestrong{no} programming has to be learned.
This course do not expect any pre\sphinxhyphen{}coding skills. This course is intended for Basic Groundwater teaching, and that is the focus.
But, how about learning groundwater by coding?

Eventually, the depth of programming to learn is an individual choice. This course considers programming as a tool to learn better.

In this course the \sphinxstyleemphasis{codes} can are written in a way so that it can be easily read. In addition, this interactive book will allow quite many of the \sphinxstyleemphasis{code} to be edited and executed in the book itself. For more advanced learning the popular notebook interface \sphinxstylestrong{JUPYTER} is to be used.

\sphinxstylestrong{JUPYTER} interface, on which interface this book is developed, is very briefly explained in the next section.


\section{JUPYTER Notebook Interface for Python}
\label{\detokenize{contents/background/02_jupyter:jupyter-notebook-interface-for-python}}\label{\detokenize{contents/background/02_jupyter::doc}}
As we mentioned earlier, this interactive book/course can be used without any programming experience. Code based calculations that is part of this book can be mostly executed in the book itself. The codes provided in the book, still with only limited skill programming skill, can be adopted for more illustrative use. This can be done in two ways. First approach, and a quick one, will be to use the web\sphinxhyphen{}based tool called \sphinxhref{https://mybinder.org/}{Binder Project}. The second approach will be to use the codes in the off\sphinxhyphen{}line systems. Common to both approach is very useful computing interface called \sphinxstylestrong{JUPYTER}.

Here we briefly learn about \sphinxstylestrong{JUPYTER} interface.

\sphinxstylestrong{JUPYTER} is a computing interface and has been in development since 2015. \sphinxstyleemphasis{JUPYTER} provide computing interface for several programming language, and thus its name is derived from:
\begin{itemize}
\item {} 
\sphinxstylestrong{JU} : Julia programming language

\item {} 
\sphinxstylestrong{PY} : Python programming language

\item {} 
\sphinxstylestrong{R}  : R programming language

\end{itemize}

More important aspects of \sphinxstylestrong{JUPYTER} computing interface:
\begin{itemize}
\item {} 
Browser\sphinxhyphen{}based tool: \sphinxhyphen{} i.e., should also run in smartphone/tabs in a

\item {} 
OPen\sphinxhyphen{}source: i.e., community based development and personalization

\item {} 
Active development: Very actively under development, especially from academic/research sector.

\end{itemize}


\section{How to use JUPYTER}
\label{\detokenize{contents/background/02_jupyter:how-to-use-jupyter}}
JUPYTER has a block\sphinxhyphen{}based interface (called \sphinxstylestrong{CELL}). Each \sphinxstyleemphasis{Cell} is either an \sphinxstylestrong{input (In{[}1{]})} cell or corresponding an \sphinxstylestrong{output (Out{[}1{]})}  cell. The \sphinxstyleemphasis{input} and \sphinxstyleemphasis{output} cells can be interactively operated.

\noindent\sphinxincludegraphics[width=600\sphinxpxdimen]{{bg2_f1}.png}

The \sphinxstyleemphasis{cell\sphinxhyphen{}based} interface is very intuitive specially for learning as it can show combine for example the code or mathematical concept with a visualization, i.e., both the cause and effect can be observed immediately and dynamically.

\noindent\sphinxincludegraphics[width=600\sphinxpxdimen]{{bg2_f2}.png}

For more advanced use, JUPYTER interface can be used for developing codes in different programming languages other than \sphinxstyleemphasis{Python.}

The JUPYTER cheat\sphinxhyphen{}sheet (from \sphinxhref{https://datacamp-community-prod.s3.amazonaws.com/48093c40-5303-45f4-bbf9-0c96c0133c40}{here}{]} can be helpful to get quickly started.

Also, for better computing and learning, installing \sphinxstylestrong{JUPYTER} locally in personal system is encouraged. This can be done using the instructions provided \sphinxhref{https://jupyter.readthedocs.io/en/latest/install.html}{here}.

\begin{sphinxVerbatim}[commandchars=\\\{\}]
\PYG{c+c1}{\PYGZsh{} required libraries }
\PYG{k+kn}{import} \PYG{n+nn}{numpy} \PYG{k}{as} \PYG{n+nn}{np}
\PYG{k+kn}{import} \PYG{n+nn}{matplotlib}\PYG{n+nn}{.}\PYG{n+nn}{pyplot} \PYG{k}{as} \PYG{n+nn}{plt}
\PYG{k+kn}{import} \PYG{n+nn}{panel} \PYG{k}{as} \PYG{n+nn}{pn}
\PYG{k+kn}{import} \PYG{n+nn}{pandas} \PYG{k}{as} \PYG{n+nn}{pd}
\PYG{n}{pn}\PYG{o}{.}\PYG{n}{extension}\PYG{p}{(}\PYG{l+s+s2}{\PYGZdq{}}\PYG{l+s+s2}{katex}\PYG{l+s+s2}{\PYGZdq{}}\PYG{p}{)}
\PYG{k+kn}{import} \PYG{n+nn}{ipywidgets} \PYG{k}{as} \PYG{n+nn}{widgets}

\PYG{c+c1}{\PYGZsh{}from IPython.display import Image, Video}

\PYG{c+c1}{\PYGZsh{}import warnings}
\PYG{c+c1}{\PYGZsh{}warnings.filterwarnings(\PYGZsq{}ignore\PYGZsq{})}
\end{sphinxVerbatim}


\section{Course Introduction \sphinxhyphen{} Groundwater I}
\label{\detokenize{contents/background/03_basic_hydrogeology:course-introduction-groundwater-i}}\label{\detokenize{contents/background/03_basic_hydrogeology::doc}}

\subsection{Contents of “Groundwater Module ”}
\label{\detokenize{contents/background/03_basic_hydrogeology:contents-of-groundwater-module}}
The entire contents of this interactive book can be listed with the following points:
\begin{itemize}
\item {} 
general overview

\item {} 
types of aquifers, properties of aquifers

\item {} 
forces and pressure in the subsurface

\item {} 
laws of groundwater flow and some applications (e.g. groundwater wells)

\item {} 
quantification of aquifer parameter values via pumping tests

\item {} 
transport of chemicals (solutes) in groundwater

\item {} 
retardation and degradation of chemicals in groundwater

\item {} 
groundwater modelling

\end{itemize}


\subsection{Suggested Literature:}
\label{\detokenize{contents/background/03_basic_hydrogeology:suggested-literature}}\begin{itemize}
\item {} 
Brassington R. (1988): Field hydrogeology, Wiley \& Sons.

\item {} 
Domenico P. A., Schwartz F. W. (1990): Physical and chemical hydrogeology, Wiley \& Sons.

\item {} 
Fetter C. W. (2001): Applied hydrogeology, Prentice Hall.

\item {} 
Freeze R. A., Cherry J. A. (1979): Groundwater, Prentice Hall.

\item {} 
Heath R. C. (1987): Basic groundwater hydrology, USGS Water Supply Paper 2220.

\item {} 
Price M. (1996): Introducing groundwater, Chapman and Hall.

\end{itemize}

Additional literature details are provided in the text when used.


\subsection{What is Hydrogeology?}
\label{\detokenize{contents/background/03_basic_hydrogeology:what-is-hydrogeology}}
\sphinxstyleemphasis{\sphinxstylestrong{Hydrogeology}} is the study of the laws governing the movement of subterranean water, the mechanical, chemical, and thermal interaction of this water with the porous solid, and the transport of energy and chemical constituents by the flow.
(Domenico and Schwartz, 1990)

The dominant reliance of groundwater for the drinking globally has made hydrogeology a very important academic course. Also, it is a very important research field. Therefore, several \sphinxstylestrong{techniques} and \sphinxstylestrong{methods} are now available to explore and understand \sphinxstylestrong{Hydrogeological Process}. The methods and techniques can be broadly categorized to:
\begin{enumerate}
\sphinxsetlistlabels{\arabic}{enumi}{enumii}{}{.}%
\item {} 
Field works

\item {} 
Laboratory experiments

\item {} 
Computer modeling

\end{enumerate}

\sphinxstyleemphasis{Computer modelling} is often the most economical method but its usefullness rely of data obtained from \sphinxstyleemphasis{Field works} and \sphinxstyleemphasis{Laboratory experiments.} Thus, the sequence of techniques/methods to be adopted depends on the available site information.

\begin{sphinxVerbatim}[commandchars=\\\{\}]
\PYG{n}{im1} \PYG{o}{=} \PYG{n}{pn}\PYG{o}{.}\PYG{n}{pane}\PYG{o}{.}\PYG{n}{PNG}\PYG{p}{(}\PYG{l+s+s2}{\PYGZdq{}}\PYG{l+s+s2}{images/L01\PYGZus{}f\PYGZus{}1c.png}\PYG{l+s+s2}{\PYGZdq{}}\PYG{p}{,} \PYG{n}{width}\PYG{o}{=}\PYG{l+m+mi}{250}\PYG{p}{)}
\PYG{n}{im2} \PYG{o}{=} \PYG{n}{pn}\PYG{o}{.}\PYG{n}{pane}\PYG{o}{.}\PYG{n}{PNG}\PYG{p}{(}\PYG{l+s+s2}{\PYGZdq{}}\PYG{l+s+s2}{images/L01\PYGZus{}f\PYGZus{}1b.png}\PYG{l+s+s2}{\PYGZdq{}}\PYG{p}{,} \PYG{n}{width}\PYG{o}{=}\PYG{l+m+mi}{275}\PYG{p}{)}
\PYG{n}{im3} \PYG{o}{=} \PYG{n}{pn}\PYG{o}{.}\PYG{n}{pane}\PYG{o}{.}\PYG{n}{PNG}\PYG{p}{(}\PYG{l+s+s2}{\PYGZdq{}}\PYG{l+s+s2}{images/L01\PYGZus{}f\PYGZus{}1a.png}\PYG{l+s+s2}{\PYGZdq{}}\PYG{p}{,} \PYG{n}{width}\PYG{o}{=}\PYG{l+m+mi}{280}\PYG{p}{)}

\PYG{n}{pn}\PYG{o}{.}\PYG{n}{Row}\PYG{p}{(}\PYG{n}{im1}\PYG{p}{,} \PYG{n}{im2}\PYG{p}{,} \PYG{n}{im3}\PYG{p}{)}
\end{sphinxVerbatim}

\begin{sphinxVerbatim}[commandchars=\\\{\}]
Row
    [0] PNG(str, width=250)
    [1] PNG(str, width=275)
    [2] PNG(str, width=280)
\end{sphinxVerbatim}


\subsection{Example: Groundwater Extraction Well}
\label{\detokenize{contents/background/03_basic_hydrogeology:example-groundwater-extraction-well}}
Groundwater is extracted using a groundwater well applying \sphinxstyleemphasis{hydrogeological} methods and techniques. The procedure followed can be summarized in the following steps:
\begin{enumerate}
\sphinxsetlistlabels{\arabic}{enumi}{enumii}{}{.}%
\item {} 
The appropriate extraction location is identified

\item {} 
Drilling machine are used to obtain sub\sphinxhyphen{}surface structure, i.e. or well logs are obtained. The process is also called well logging.

\item {} 
Well logs are studied in detail to identify the characteristics of the subsurface\sphinxhyphen{} e.g., how thick is the aquifer or identify environmental consequence of water extraction.

\item {} 
The construction of well begins

\end{enumerate}

Groundwater extraction using well is a challenge when aquifers are located very deep from the surface, e.g., in deserts.

\begin{sphinxVerbatim}[commandchars=\\\{\}]
\PYG{n}{video1} \PYG{o}{=} \PYG{n}{pn}\PYG{o}{.}\PYG{n}{pane}\PYG{o}{.}\PYG{n}{Video}\PYG{p}{(}\PYG{l+s+s2}{\PYGZdq{}}\PYG{l+s+s2}{images/L01\PYGZus{}f\PYGZus{}2.mp4}\PYG{l+s+s2}{\PYGZdq{}}\PYG{p}{,} \PYG{n}{width}\PYG{o}{=}\PYG{l+m+mi}{600}\PYG{p}{,} \PYG{n}{height}\PYG{o}{=}\PYG{l+m+mi}{400}\PYG{p}{,} \PYG{n}{loop}\PYG{o}{=}\PYG{k+kc}{False}\PYG{p}{)}
\PYG{n}{video1}
\end{sphinxVerbatim}

\begin{sphinxVerbatim}[commandchars=\\\{\}]
Video(str, height=400, sizing\PYGZus{}mode=\PYGZsq{}fixed\PYGZsq{}, width=600)
\end{sphinxVerbatim}

\begin{sphinxVerbatim}[commandchars=\\\{\}]
\PYG{c+c1}{\PYGZsh{}gif\PYGZus{}pane = pn.pane.GIF(\PYGZsq{}images/L01\PYGZus{}f\PYGZus{}2.gif\PYGZsq{}, width=500)}
\PYG{c+c1}{\PYGZsh{}gif\PYGZus{}pane}
\PYG{n}{video2} \PYG{o}{=} \PYG{n}{pn}\PYG{o}{.}\PYG{n}{pane}\PYG{o}{.}\PYG{n}{Video}\PYG{p}{(}\PYG{l+s+s2}{\PYGZdq{}}\PYG{l+s+s2}{images/L01\PYGZus{}f\PYGZus{}3.mp4}\PYG{l+s+s2}{\PYGZdq{}}\PYG{p}{,} \PYG{n}{width}\PYG{o}{=}\PYG{l+m+mi}{600}\PYG{p}{,} \PYG{n}{height}\PYG{o}{=}\PYG{l+m+mi}{400}\PYG{p}{,} \PYG{n}{loop}\PYG{o}{=}\PYG{k+kc}{False}\PYG{p}{)}
\PYG{c+c1}{\PYGZsh{}Video(\PYGZdq{}images/L01\PYGZus{}f\PYGZus{}3.mp4\PYGZdq{}, width=600, embed=True) }
\PYG{n}{video2} 

\PYG{n}{pn1} \PYG{o}{=} \PYG{n}{pn}\PYG{o}{.}\PYG{n}{pane}\PYG{o}{.}\PYG{n}{Markdown}\PYG{p}{(}\PYG{l+s+s2}{\PYGZdq{}\PYGZdq{}\PYGZdq{}}
\PYG{l+s+s2}{**Wells** are placed on the layer or that aquifer part which allows feasible extraction of groundwater. }
\PYG{l+s+s2}{The extraction leads to drop of groundwater level. To ensure that there is sustainable extraction,}
\PYG{l+s+s2}{the drops in the level has to be monitored. Quite often this is done through \PYGZus{}computer modelling\PYGZus{}. There}
\PYG{l+s+s2}{already exists several computer models that can use the well logs data (also called Borehole) and provide}
\PYG{l+s+s2}{good estimations of the effects due to extraction. The \PYGZus{}computer models\PYGZus{} are also able to predict the effects}
\PYG{l+s+s2}{at larger scales, e.g., regional scales. \PYGZus{}Computer models\PYGZus{} are oftenly used these days be agencies to determine}
\PYG{l+s+s2}{quantities such as **travel time**, **capture zones** or obtain **isochrones**, which are used for deciding on}
\PYG{l+s+s2}{groundwater extraction programmes.}
\PYG{l+s+s2}{\PYGZdq{}\PYGZdq{}\PYGZdq{}}\PYG{p}{)}

\PYG{n}{pn}\PYG{o}{.}\PYG{n}{Row}\PYG{p}{(}\PYG{n}{pn1}\PYG{p}{,} \PYG{n}{video2}\PYG{p}{)}
\end{sphinxVerbatim}

\begin{sphinxVerbatim}[commandchars=\\\{\}]
Row
    [0] Markdown(str)
    [1] Video(str, height=400, sizing\PYGZus{}mode=\PYGZsq{}fixed\PYGZsq{}, width=600)
\end{sphinxVerbatim}


\subsection{Groundwater and Global Water Cycle}
\label{\detokenize{contents/background/03_basic_hydrogeology:groundwater-and-global-water-cycle}}
Water bodies that exist on earth is connected, and they function as a cycle, called \sphinxstylestrong{Global Water Cycle}. It is estimated that over 57, 700 Km\(^3\) of water actively participates in the cycle each year. \sphinxstylestrong{Precipitation} and \sphinxstylestrong{evaporation} are the two main components of the cycle in which \sphinxstylestrong{temperature} plays the critical role. In the cycle, \sphinxstylestrong{Groundwater} receives water from \sphinxstyleemphasis{precipitation,} It then contributes to \sphinxstyleemphasis{evaporation} through subsurface flow or through mostly human intervention (e.g., use for drinking water).

The water cycle provides an approach to judge the sustainability of groundwater extraction. The sustainability of extraction can be obtained if extraction rate approximately equals the replenishing rate. Often the replenishing rate of groundwater is much slower and this has led to groundwater stress in many parts of the world.

\begin{sphinxVerbatim}[commandchars=\\\{\}]
\PYG{c+c1}{\PYGZsh{}gif\PYGZus{}pane = pn.pane.GIF(\PYGZsq{}images/L01\PYGZus{}f\PYGZus{}2.gif\PYGZsq{}, width=500)}
\PYG{c+c1}{\PYGZsh{}gif\PYGZus{}pane}
\PYG{n}{video3} \PYG{o}{=} \PYG{n}{pn}\PYG{o}{.}\PYG{n}{pane}\PYG{o}{.}\PYG{n}{Video}\PYG{p}{(}\PYG{l+s+s2}{\PYGZdq{}}\PYG{l+s+s2}{images/L01\PYGZus{}f\PYGZus{}4.mp4}\PYG{l+s+s2}{\PYGZdq{}}\PYG{p}{,} \PYG{n}{width}\PYG{o}{=}\PYG{l+m+mi}{600}\PYG{p}{,} \PYG{n}{height}\PYG{o}{=}\PYG{l+m+mi}{400}\PYG{p}{,} \PYG{n}{loop}\PYG{o}{=}\PYG{k+kc}{False}\PYG{p}{)}
\PYG{c+c1}{\PYGZsh{}Video(\PYGZdq{}images/L01\PYGZus{}f\PYGZus{}3.mp4\PYGZdq{}, width=600, embed=True) }
\PYG{n}{video3} 
\end{sphinxVerbatim}

\begin{sphinxVerbatim}[commandchars=\\\{\}]
Video(str, height=400, sizing\PYGZus{}mode=\PYGZsq{}fixed\PYGZsq{}, width=600)
\end{sphinxVerbatim}

\begin{sphinxVerbatim}[commandchars=\\\{\}]
\PYG{c+c1}{\PYGZsh{}gif\PYGZus{}pane = pn.pane.GIF(\PYGZsq{}images/L01\PYGZus{}f\PYGZus{}2.gif\PYGZsq{}, width=500)}
\PYG{c+c1}{\PYGZsh{}gif\PYGZus{}pane}
\PYG{n}{fig5} \PYG{o}{=} \PYG{n}{pn}\PYG{o}{.}\PYG{n}{pane}\PYG{o}{.}\PYG{n}{PNG}\PYG{p}{(}\PYG{l+s+s2}{\PYGZdq{}}\PYG{l+s+s2}{images/L01\PYGZus{}f\PYGZus{}5.png}\PYG{l+s+s2}{\PYGZdq{}}\PYG{p}{,} \PYG{n}{width}\PYG{o}{=}\PYG{l+m+mi}{600}\PYG{p}{)} 
\PYG{c+c1}{\PYGZsh{}Video(\PYGZdq{}images/L01\PYGZus{}f\PYGZus{}3.mp4\PYGZdq{}, width=600, embed=True) }
 

\PYG{n}{pn1} \PYG{o}{=} \PYG{n}{pn}\PYG{o}{.}\PYG{n}{pane}\PYG{o}{.}\PYG{n}{Markdown}\PYG{p}{(}\PYG{l+s+s2}{\PYGZdq{}\PYGZdq{}\PYGZdq{}}\PYG{l+s+s2}{ }

\PYG{l+s+s2}{\PYGZsh{}\PYGZsh{}\PYGZsh{} Water balance by continents}

\PYG{l+s+s2}{Groundwater receives water from the \PYGZus{}infiltration\PYGZus{} of **runoff** water. }

\PYG{l+s+s2}{\PYGZdq{}\PYGZdq{}\PYGZdq{}}\PYG{p}{)}

\PYG{n}{pn}\PYG{o}{.}\PYG{n}{Row}\PYG{p}{(}\PYG{n}{pn1}\PYG{p}{,} \PYG{n}{fig5}\PYG{p}{)}
\end{sphinxVerbatim}

\begin{sphinxVerbatim}[commandchars=\\\{\}]
Row
    [0] Markdown(str)
    [1] PNG(str, width=600)
\end{sphinxVerbatim}


\subsection{The Hydrological Balance}
\label{\detokenize{contents/background/03_basic_hydrogeology:the-hydrological-balance}}
Since \sphinxstyleemphasis{groundwater} is part of the global water cycle, the balance of the cycle becomes an important topic. In general:
\begin{itemize}
\item {} 
The \sphinxstyleemphasis{hydrological balance} provides a relationship between various flow rates for a certain area. It is based on the conservation of water volume.

\item {} 
expressed in words:  \sphinxstyleemphasis{inflow} equals \sphinxstyleemphasis{outflow} plus \sphinxstyleemphasis{change in storage}

\item {} 
expressed by a formula:

\end{itemize}
\begin{equation*}
\begin{split}
P = ET + R + \Delta S
\end{split}
\end{equation*}
\begin{sphinxShadowBox}
\sphinxstylesidebartitle{Where,}

where, 
\(P\) = \sphinxstyleemphasis{Precipitation,  \(ET\) = Evapotranspiration,  \(R\) = Runoff,}  and  \(\Delta S\) = \sphinxstyleemphasis{Change in Storage}
\end{sphinxShadowBox}

The \sphinxstyleemphasis{change in storage} can be interpreted in the following way:
\begin{itemize}
\item {} 
change in storage \(\Delta S > 0\) : Water volume is increasing with time in the investigation area.

\item {} 
change in storage \(\Delta S < 0\):Water volume is decreasing with time in the investigation area.

\item {} 
change in storage \(\Delta S = 0\): Water volume does not change with time in the investigation area (steady\sphinxhyphen{}state or stationary situation, i.e. inflow equals outflow).

\end{itemize}


\subsection{Water Volume}
\label{\detokenize{contents/background/03_basic_hydrogeology:water-volume}}
\sphinxstylestrong{So how much water do we have?}
It is estimated* that the total volume of water on Earth amounts to ca. 1 358 710 150 km\(^3\) (\(\approx\) 1018 m\(^3\)).

\noindent{\hspace*{\fill}\sphinxincludegraphics[height=200\sphinxpxdimen]{{L01_f_6}.png}\hspace*{\fill}}

The total volume of fresh water on Earth amounts to ca. \(38\times 10^6\) km\(^3\) (\(\approx\) 1016 m\(^3\)).

\sphinxstyleemphasis{*Gleick P. (1996): Water re\sphinxhyphen{}sources, in: Schneider S. H. (ed.), Encyclopedia of climate and weather 2, Oxford Univ. Press.}


\subsection{Volume of Available Fresh Water}
\label{\detokenize{contents/background/03_basic_hydrogeology:volume-of-available-fresh-water}}
\sphinxstylestrong{Fresh water} are water with low concentrations of dissolved salts and other total dissolved solids, i.e., sea/ocean water or brackish water are not fresh water. Human activities (drinking water) are directly dependent on fresh activities.

\sphinxstylestrong{So how much \sphinxstyleemphasis{fresh water} do we have?}

It is estimated* that the total volume of available fresh water (liquid) on Earth amounts to ca. 8 831 600 km\(^3\) (\(\approx\) 1016 m\(^3\)).

\noindent{\hspace*{\fill}\sphinxincludegraphics[height=200\sphinxpxdimen]{{L01_f_7}.png}\hspace*{\fill}}

\sphinxstyleemphasis{*Gleick P. (1996): Water re\sphinxhyphen{}sources, in: Schneider S. H. (ed.), Encyclopedia of climate and weather 2, Oxford Univ. Press.}


\subsection{Continental distribution of fresh water components}
\label{\detokenize{contents/background/03_basic_hydrogeology:continental-distribution-of-fresh-water-components}}
\noindent{\hspace*{\fill}\sphinxincludegraphics[height=500\sphinxpxdimen]{{L01_f_8}.png}\hspace*{\fill}}


\subsection{Volume and Mass Budget}
\label{\detokenize{contents/background/03_basic_hydrogeology:volume-and-mass-budget}}
Very basics of volume and mass budget \sphinxhyphen{} let us start with \sphinxstyleemphasis{budget.}

\sphinxstylestrong{Budget} = quantitative comparison of \sphinxstyleemphasis{growth} (or \sphinxstyleemphasis{production}) and \sphinxstyleemphasis{loss} in a system

Budgets can be put together for various quantities:
\begin{itemize}
\item {} 
energy

\item {} 
mass \(\leftarrow\) needed to quantify transport of solutes in groundwater

\item {} 
volume \(\leftarrow\)  needed to quantify groundwater flow

\item {} 
momentum

\item {} 
electric charge

\item {} 
number of inhabitants

\item {} 
animal population

\item {} 
money (bank account!)

\item {} 
and many others

\end{itemize}

In this course we focus on \sphinxstyleemphasis{Mass Budget} and \sphinxstyleemphasis{Volume Budget.}


\subsection{Volume Budget}
\label{\detokenize{contents/background/03_basic_hydrogeology:volume-budget}}
As discussed in the last topic a \sphinxstyleemphasis{budget} represents the change (e.g., growth and loss). Thus, it is more suitable to quantify the \sphinxstylestrong{volume budget} in terms of a change, representing two different states (e.g., time \((t)\)). More formally, the \sphinxstylestrong{volume budget} (\(\Delta V\)) can be obtained from:
\begin{equation*}
\begin{split}
\Delta V = Q_{in} \cdot \Delta t - Q{out} \cdot \Delta t 
\end{split}
\end{equation*}


\(\Delta t\) = time interval {[}T{]} 
\(\Delta V\) = change of volume in the system {[}L\(^3\){]} 
\(Q_{in}\) = volumetric rate of flow into the system {[}L\(^3\)/T{]} 
\(Q_{out} =\) volumetric rate of flow out of the system {[}L\(^3\)/T{]} 

The following points have to be considered when using the above equation:
\begin{itemize}
\item {} 
Inflow and outflow may each consist of several individual components. 

\item {} 
\(\Delta V = 0\) (no change in  volume) is tantamount to steady\sphinxhyphen{}state or stationary (= time\sphinxhyphen{}independent) conditions. 

\item {} 
For steady\sphinxhyphen{}state conditions we have: \(Q_{in} = Q_{out}\)

\end{itemize}


\subsubsection{Water Budget for a Catchment}
\label{\detokenize{contents/background/03_basic_hydrogeology:water-budget-for-a-catchment}}
The equation provided for \sphinxstyleemphasis{volume budget} looks simple but in practice it is very complicated as several \sphinxstyleemphasis{inflow} and \sphinxstyleemphasis{outflow} components must be considered. Quantifying these components can be a challenging task.

For quantifying water budget of a catchment, one has to consider the following components:

\sphinxstylestrong{To be considered:}
\begin{itemize}
\item {} 
precipitation

\item {} 
evapotranspiration

\item {} 
surface runoff

\item {} 
subsurface runoff

\end{itemize}

Among the above components, quantification of evapotranspiration and subsurface runoff have very high level of uncertainties.

\noindent{\hspace*{\fill}\sphinxincludegraphics[height=500\sphinxpxdimen]{{L01_f_9}.png}\hspace*{\fill}}


\subsection{Example: Estimation of Subsurface Runoff}
\label{\detokenize{contents/background/03_basic_hydrogeology:example-estimation-of-subsurface-runoff}}
Most numbers used in the example do not refer to the catchment shown before!

To calculate the following four\sphinxhyphen{}steps are to be followed:
\begin{itemize}
\item {} 
Step 1: determine rate of inflow in m\(\sp{\text{3}}\)/a

\item {} 
step 2: determine rate of outflow due to evapotranspiration (ET.A) in m\(\sp{\text{3}}\)/a

\item {} 
Step 3: express rate of outflow due to surface runoff in m\(\sp{\text{3}}\)/a

\item {} 
step 4: determine rate of outflow due to subsurface runoff

\end{itemize}

An Example:
For given data, determine the rate of outflow Qout,sub due to subsurface runoff for steady\sphinxhyphen{}state conditions

\begin{sphinxVerbatim}[commandchars=\\\{\}]
\PYG{n}{A} \PYG{o}{=} \PYG{l+m+mi}{4500} \PYG{c+c1}{\PYGZsh{} km\(\sp{\text{2}}\), catchment area}
\PYG{n}{P} \PYG{o}{=} \PYG{l+m+mi}{550} \PYG{c+c1}{\PYGZsh{} mm/a, precipitation}
\PYG{n}{ET} \PYG{o}{=} \PYG{l+m+mi}{200} \PYG{c+c1}{\PYGZsh{} mm/a, evapotranspiration}
\PYG{n}{Qout\PYGZus{}surf} \PYG{o}{=} \PYG{l+m+mi}{40} \PYG{c+c1}{\PYGZsh{} m\(\sp{\text{3}}\)/s, surface runoff}
\PYG{n}{Delta\PYGZus{}V} \PYG{o}{=} \PYG{l+m+mi}{0} \PYG{c+c1}{\PYGZsh{} m\(\sp{\text{3}}\), change in volume = 0 Steady\PYGZhy{}state conditions}

\PYG{c+c1}{\PYGZsh{}Volume budget in this example: P·A = ET·A + Qout,surf + Qout,sub}

\PYG{c+c1}{\PYGZsh{}Step 1    }
\PYG{n}{Qin} \PYG{o}{=} \PYG{n}{P}\PYG{o}{*}\PYG{n}{A}\PYG{o}{*}\PYG{l+m+mi}{10}\PYG{o}{*}\PYG{o}{*}\PYG{l+m+mi}{3}  \PYG{c+c1}{\PYGZsh{}m\(\sp{\text{3}}\)/a, 10\PYGZca{}3 for unit conversion}

\PYG{c+c1}{\PYGZsh{}step 2: }
\PYG{n}{ET\PYGZus{}A} \PYG{o}{=} \PYG{n}{ET}\PYG{o}{*}\PYG{n}{A}\PYG{o}{*}\PYG{l+m+mi}{10}\PYG{o}{*}\PYG{o}{*}\PYG{l+m+mi}{3} \PYG{c+c1}{\PYGZsh{}m\(\sp{\text{3}}\)/a, 10\PYGZca{}3 for unit conversion}

\PYG{c+c1}{\PYGZsh{}Step 3: }
\PYG{n}{Qout\PYGZus{}surf} \PYG{o}{=} \PYG{n}{Qout\PYGZus{}surf} \PYG{o}{*}\PYG{l+m+mi}{365}\PYG{o}{*}\PYG{l+m+mi}{24}\PYG{o}{*}\PYG{l+m+mi}{3600} \PYG{c+c1}{\PYGZsh{}  m\(\sp{\text{3}}\)/a}

\PYG{c+c1}{\PYGZsh{} step 4:}
\PYG{n}{Qout\PYGZus{}sub} \PYG{o}{=} \PYG{n}{Qin} \PYG{o}{\PYGZhy{}} \PYG{n}{ET\PYGZus{}A} \PYG{o}{\PYGZhy{}} \PYG{n}{Qout\PYGZus{}surf} \PYG{c+c1}{\PYGZsh{} m\(\sp{\text{3}}\)/a }



\PYG{n+nb}{print}\PYG{p}{(}\PYG{l+s+s2}{\PYGZdq{}}\PYG{l+s+s2}{The rate of inflow, Qin is }\PYG{l+s+si}{\PYGZob{}0:1.1E\PYGZcb{}}\PYG{l+s+s2}{\PYGZdq{}}\PYG{o}{.}\PYG{n}{format}\PYG{p}{(}\PYG{n}{Qin}\PYG{p}{)}\PYG{p}{,}\PYG{l+s+s2}{\PYGZdq{}}\PYG{l+s+s2}{m}\PYG{l+s+se}{\PYGZbs{}u00b3}\PYG{l+s+s2}{/a }\PYG{l+s+se}{\PYGZbs{}n}\PYG{l+s+s2}{\PYGZdq{}}\PYG{p}{)}\PYG{p}{;} \PYG{n+nb}{print}\PYG{p}{(}\PYG{l+s+s2}{\PYGZdq{}}\PYG{l+s+s2}{The outflow rate due to Evapotranspiration is }\PYG{l+s+si}{\PYGZob{}0:1.1E\PYGZcb{}}\PYG{l+s+s2}{\PYGZdq{}}\PYG{o}{.}\PYG{n}{format}\PYG{p}{(}\PYG{n}{ET\PYGZus{}A}\PYG{p}{)}\PYG{p}{,}\PYG{l+s+s2}{\PYGZdq{}}\PYG{l+s+s2}{m}\PYG{l+s+se}{\PYGZbs{}u00b3}\PYG{l+s+s2}{/a }\PYG{l+s+se}{\PYGZbs{}n}\PYG{l+s+s2}{\PYGZdq{}}\PYG{p}{)}
\PYG{n+nb}{print}\PYG{p}{(}\PYG{l+s+s2}{\PYGZdq{}}\PYG{l+s+s2}{The surface outflow rate, Q\PYGZus{}out\PYGZus{}surf in m}\PYG{l+s+se}{\PYGZbs{}u00b3}\PYG{l+s+s2}{/a is }\PYG{l+s+si}{\PYGZob{}0:1.1E\PYGZcb{}}\PYG{l+s+s2}{\PYGZdq{}}\PYG{o}{.}\PYG{n}{format}\PYG{p}{(}\PYG{n}{Qout\PYGZus{}surf}\PYG{p}{)}\PYG{p}{,}\PYG{l+s+s2}{\PYGZdq{}}\PYG{l+s+s2}{m}\PYG{l+s+se}{\PYGZbs{}u00b3}\PYG{l+s+s2}{/a }\PYG{l+s+se}{\PYGZbs{}n}\PYG{l+s+s2}{\PYGZdq{}}\PYG{p}{)}\PYG{p}{;}\PYG{n+nb}{print}\PYG{p}{(}\PYG{l+s+s2}{\PYGZdq{}}\PYG{l+s+s2}{The subsurface outflow rate, Qout\PYGZus{}surf in m}\PYG{l+s+se}{\PYGZbs{}u00b3}\PYG{l+s+s2}{/a is }\PYG{l+s+si}{\PYGZob{}0:1.1E\PYGZcb{}}\PYG{l+s+s2}{\PYGZdq{}}\PYG{o}{.}\PYG{n}{format}\PYG{p}{(}\PYG{n}{Qout\PYGZus{}sub}\PYG{p}{)}\PYG{p}{,}\PYG{l+s+s2}{\PYGZdq{}}\PYG{l+s+s2}{m}\PYG{l+s+se}{\PYGZbs{}u00b3}\PYG{l+s+s2}{/a }\PYG{l+s+se}{\PYGZbs{}n}\PYG{l+s+s2}{\PYGZdq{}}\PYG{p}{)}
\end{sphinxVerbatim}

\begin{sphinxVerbatim}[commandchars=\\\{\}]
The rate of inflow, Qin is 2.5E+09 m\(\sp{\text{3}}\)/a 

The outflow rate due to Evapotranspiration is 9.0E+08 m\(\sp{\text{3}}\)/a 

The surface outflow rate, Q\PYGZus{}out\PYGZus{}surf in m\(\sp{\text{3}}\)/a is 1.3E+09 m\(\sp{\text{3}}\)/a 

The subsurface outflow rate, Qout\PYGZus{}surf in m\(\sp{\text{3}}\)/a is 3.1E+08 m\(\sp{\text{3}}\)/a 
\end{sphinxVerbatim}


\subsection{Mass Budget}
\label{\detokenize{contents/background/03_basic_hydrogeology:mass-budget}}
The \sphinxstylestrong{mass budget} is quantified similar to the \sphinxstyleemphasis{volume budget.} Mathematically, the \sphinxstyleemphasis{mass budget} is:
\begin{equation*}
\begin{split}\Delta M = J_{in}\cdot \Delta t - J_{out} \cdot \Delta t\end{split}
\end{equation*}
with 
\(\Delta t\) = time interval {[}T{]}
\(\Delta M\) = change of mass in the system {[}M{]}
\(J_{in}\) = rate of mass flow into the system {[}M/T{]}
\(J_{out}\) = rate of mass flow out of the system {[}M/T{]}

Similar to \sphinxstyleemphasis{volume budget,} the following points have to be considered in quantifying mass budget:
\begin{itemize}
\item {} 
Inflow and outflow may each consist of several individual components.

\item {} 
\(\Delta M\) = 0 (no change in mass) is tantamount to steady\sphinxhyphen{}state or stationary (= time\sphinxhyphen{}independent) conditions.

\item {} 
For steady\sphinxhyphen{}state conditions we have: \(J_{in}\)= \(J_{out}\)

\end{itemize}


\subsection{Example of Mass Budget: Radioactive Decay}
\label{\detokenize{contents/background/03_basic_hydrogeology:example-of-mass-budget-radioactive-decay}}
Consider a decay chain comprising of the three chemicals: \sphinxstylestrong{A}, \sphinxstylestrong{B} and \sphinxstylestrong{C}
\begin{itemize}
\item {} 
decay chain: A \(\rightarrow\) B \(\rightarrow\) C       

\item {} 
30\% of \(\text{A}\) and 20\% of \(\text{B}\)  decay each year.

\item {} 
decay rate of \(\text{A}\)   = production rate of \(\text{B}\)   = \(0.3 \cdot a^{-1}\cdot M_A\) 

\item {} 
decay rate of \(\text{B}\) = production rate of \(\text{C}\) = \(0.2\cdot a^{-1}\cdot M_B\) 

\item {} 
mass budgets for \(\text{A}\), \(\text{B}\) and \(\text{C}\):

\end{itemize}
\begin{equation*}
\begin{split}
\Delta M_A &= 0.3 \text{ a $^{-1}$ } \cdot M_A  \cdot \Delta t  \\
\Delta M_B  &= 0.3 \text{a$^{-1}$} \cdot M_A  \cdot \Delta t - 0.2 \text{ a$^{-1}$} \cdot M_B  \cdot \Delta t \\
\Delta M_C &= 0.2 \text{a$^{-1}$} \cdot M_B  \cdot \Delta t
\end{split}
\end{equation*}\begin{itemize}
\item {} 
Similar equations hold for quantitative descriptions of some chemical reactions which correspond to the type A \(\rightarrow\) B \(\rightarrow\) C

\end{itemize}

\begin{sphinxVerbatim}[commandchars=\\\{\}]
\PYG{k}{def} \PYG{n+nf}{mass\PYGZus{}bal}\PYG{p}{(}\PYG{n}{n\PYGZus{}simulation}\PYG{p}{,} \PYG{n}{MA}\PYG{p}{,} \PYG{n}{MB}\PYG{p}{,} \PYG{n}{MC}\PYG{p}{,} \PYG{n}{R\PYGZus{}A}\PYG{p}{,} \PYG{n}{R\PYGZus{}B}\PYG{p}{)}\PYG{p}{:}
    
    \PYG{n}{A} \PYG{o}{=} \PYG{n}{np}\PYG{o}{.}\PYG{n}{zeros}\PYG{p}{(}\PYG{n}{n\PYGZus{}simulation}\PYG{p}{)}
    \PYG{n}{B} \PYG{o}{=} \PYG{n}{np}\PYG{o}{.}\PYG{n}{zeros}\PYG{p}{(}\PYG{n}{n\PYGZus{}simulation}\PYG{p}{)}
    \PYG{n}{C} \PYG{o}{=} \PYG{n}{np}\PYG{o}{.}\PYG{n}{zeros}\PYG{p}{(}\PYG{n}{n\PYGZus{}simulation}\PYG{p}{)} 
    \PYG{n}{time}  \PYG{o}{=} \PYG{n}{np}\PYG{o}{.}\PYG{n}{arange}\PYG{p}{(}\PYG{n}{n\PYGZus{}simulation}\PYG{p}{)}
    
    \PYG{k}{for} \PYG{n}{i} \PYG{o+ow}{in} \PYG{n+nb}{range}\PYG{p}{(}\PYG{l+m+mi}{0}\PYG{p}{,}\PYG{n}{n\PYGZus{}simulation}\PYG{o}{\PYGZhy{}}\PYG{l+m+mi}{1}\PYG{p}{)}\PYG{p}{:}
        \PYG{n}{A}\PYG{p}{[}\PYG{l+m+mi}{0}\PYG{p}{]} \PYG{o}{=} \PYG{n}{MA}
        \PYG{n}{B}\PYG{p}{[}\PYG{l+m+mi}{0}\PYG{p}{]} \PYG{o}{=} \PYG{n}{MB}
        \PYG{n}{C}\PYG{p}{[}\PYG{l+m+mi}{0}\PYG{p}{]} \PYG{o}{=} \PYG{n}{MC}
        \PYG{n}{A}\PYG{p}{[}\PYG{n}{i}\PYG{o}{+}\PYG{l+m+mi}{1}\PYG{p}{]} \PYG{o}{=} \PYG{n}{A}\PYG{p}{[}\PYG{n}{i}\PYG{p}{]}\PYG{o}{\PYGZhy{}}\PYG{n}{R\PYGZus{}A}\PYG{o}{*}\PYG{n}{A}\PYG{p}{[}\PYG{n}{i}\PYG{p}{]}
        \PYG{n}{B}\PYG{p}{[}\PYG{n}{i}\PYG{o}{+}\PYG{l+m+mi}{1}\PYG{p}{]} \PYG{o}{=} \PYG{n}{B}\PYG{p}{[}\PYG{n}{i}\PYG{p}{]}\PYG{o}{+}\PYG{n}{R\PYGZus{}A}\PYG{o}{*}\PYG{n}{A}\PYG{p}{[}\PYG{n}{i}\PYG{p}{]}\PYG{o}{\PYGZhy{}}\PYG{n}{R\PYGZus{}B}\PYG{o}{*}\PYG{n}{B}\PYG{p}{[}\PYG{n}{i}\PYG{p}{]} 
        \PYG{n}{C}\PYG{p}{[}\PYG{n}{i}\PYG{o}{+}\PYG{l+m+mi}{1}\PYG{p}{]} \PYG{o}{=} \PYG{n}{C}\PYG{p}{[}\PYG{n}{i}\PYG{p}{]}\PYG{o}{+}\PYG{n}{R\PYGZus{}B}\PYG{o}{*}\PYG{n}{B}\PYG{p}{[}\PYG{n}{i}\PYG{p}{]}
        \PYG{n}{summ} \PYG{o}{=} \PYG{n}{A}\PYG{p}{[}\PYG{n}{i}\PYG{p}{]}\PYG{o}{+}\PYG{n}{B}\PYG{p}{[}\PYG{n}{i}\PYG{p}{]}\PYG{o}{+}\PYG{n}{C}\PYG{p}{[}\PYG{n}{i}\PYG{p}{]}
        
    \PYG{n}{d} \PYG{o}{=} \PYG{p}{\PYGZob{}}\PYG{l+s+s2}{\PYGZdq{}}\PYG{l+s+s2}{Mass\PYGZus{}A}\PYG{l+s+s2}{\PYGZdq{}}\PYG{p}{:} \PYG{n}{A}\PYG{p}{,} \PYG{l+s+s2}{\PYGZdq{}}\PYG{l+s+s2}{Mass\PYGZus{}B}\PYG{l+s+s2}{\PYGZdq{}}\PYG{p}{:} \PYG{n}{B}\PYG{p}{,} \PYG{l+s+s2}{\PYGZdq{}}\PYG{l+s+s2}{Mass\PYGZus{}C}\PYG{l+s+s2}{\PYGZdq{}}\PYG{p}{:} \PYG{n}{C}\PYG{p}{,} \PYG{l+s+s2}{\PYGZdq{}}\PYG{l+s+s2}{Total Mass}\PYG{l+s+s2}{\PYGZdq{}}\PYG{p}{:} \PYG{n}{summ}\PYG{p}{\PYGZcb{}}
    \PYG{n}{df} \PYG{o}{=} \PYG{n}{pd}\PYG{o}{.}\PYG{n}{DataFrame}\PYG{p}{(}\PYG{n}{d}\PYG{p}{)} \PYG{c+c1}{\PYGZsh{} Generating result table}
    \PYG{n}{label} \PYG{o}{=} \PYG{p}{[}\PYG{l+s+s2}{\PYGZdq{}}\PYG{l+s+s2}{Mass A (g)}\PYG{l+s+s2}{\PYGZdq{}}\PYG{p}{,} \PYG{l+s+s2}{\PYGZdq{}}\PYG{l+s+s2}{Mass B (g)}\PYG{l+s+s2}{\PYGZdq{}}\PYG{p}{,} \PYG{l+s+s2}{\PYGZdq{}}\PYG{l+s+s2}{Mass C (g)}\PYG{l+s+s2}{\PYGZdq{}}\PYG{p}{]}
    \PYG{n}{fig} \PYG{o}{=} \PYG{n}{plt}\PYG{o}{.}\PYG{n}{figure}\PYG{p}{(}\PYG{n}{figsize}\PYG{o}{=}\PYG{p}{(}\PYG{l+m+mi}{6}\PYG{p}{,}\PYG{l+m+mi}{4}\PYG{p}{)}\PYG{p}{)}
    \PYG{n}{plt}\PYG{o}{.}\PYG{n}{plot}\PYG{p}{(}\PYG{n}{time}\PYG{p}{,} \PYG{n}{A}\PYG{p}{,} \PYG{n}{time}\PYG{p}{,} \PYG{n}{B}\PYG{p}{,} \PYG{n}{time}\PYG{p}{,} \PYG{n}{C}\PYG{p}{,} \PYG{n}{linewidth}\PYG{o}{=}\PYG{l+m+mi}{3}\PYG{p}{)}\PYG{p}{;}  \PYG{c+c1}{\PYGZsh{} plotting the results}
    \PYG{n}{plt}\PYG{o}{.}\PYG{n}{xlabel}\PYG{p}{(}\PYG{l+s+s2}{\PYGZdq{}}\PYG{l+s+s2}{Time [Time Unit]}\PYG{l+s+s2}{\PYGZdq{}}\PYG{p}{)}\PYG{p}{;} \PYG{n}{plt}\PYG{o}{.}\PYG{n}{ylabel}\PYG{p}{(}\PYG{l+s+s2}{\PYGZdq{}}\PYG{l+s+s2}{Mass [g]}\PYG{l+s+s2}{\PYGZdq{}}\PYG{p}{)} \PYG{c+c1}{\PYGZsh{} placing axis labels}
    \PYG{n}{plt}\PYG{o}{.}\PYG{n}{legend}\PYG{p}{(}\PYG{n}{label}\PYG{p}{,} \PYG{n}{loc}\PYG{o}{=}\PYG{l+m+mi}{0}\PYG{p}{)}\PYG{p}{;}\PYG{n}{plt}\PYG{o}{.}\PYG{n}{grid}\PYG{p}{(}\PYG{p}{)}\PYG{p}{;} \PYG{n}{plt}\PYG{o}{.}\PYG{n}{xlim}\PYG{p}{(}\PYG{p}{[}\PYG{l+m+mi}{0}\PYG{p}{,}\PYG{l+m+mi}{20}\PYG{p}{]}\PYG{p}{)}\PYG{p}{;} \PYG{n}{plt}\PYG{o}{.}\PYG{n}{ylim}\PYG{p}{(}\PYG{n}{bottom}\PYG{o}{=}\PYG{l+m+mi}{0}\PYG{p}{)} \PYG{c+c1}{\PYGZsh{} legends, grids, x,y limits}
    \PYG{n}{plt}\PYG{o}{.}\PYG{n}{show}\PYG{p}{(}\PYG{p}{)} \PYG{c+c1}{\PYGZsh{} display plot}
    
    \PYG{n}{df\PYGZus{}pane} \PYG{o}{=} \PYG{n}{pn}\PYG{o}{.}\PYG{n}{pane}\PYG{o}{.}\PYG{n}{DataFrame}\PYG{p}{(}\PYG{n}{df}\PYG{p}{)}
    \PYG{k}{return} \PYG{n+nb}{print}\PYG{p}{(}\PYG{n}{df}\PYG{o}{.}\PYG{n}{round}\PYG{p}{(}\PYG{l+m+mi}{2}\PYG{p}{)}\PYG{p}{)} 

\PYG{n}{N} \PYG{o}{=} \PYG{n}{widgets}\PYG{o}{.}\PYG{n}{BoundedIntText}\PYG{p}{(}\PYG{n}{value}\PYG{o}{=}\PYG{l+m+mi}{20}\PYG{p}{,}\PYG{n+nb}{min}\PYG{o}{=}\PYG{l+m+mi}{0}\PYG{p}{,}\PYG{n+nb}{max}\PYG{o}{=}\PYG{l+m+mi}{100}\PYG{p}{,}\PYG{n}{step}\PYG{o}{=}\PYG{l+m+mi}{1}\PYG{p}{,}\PYG{n}{description}\PYG{o}{=} \PYG{l+s+s1}{\PYGZsq{}}\PYG{l+s+s1}{\PYGZam{}Delta; t (day)}\PYG{l+s+s1}{\PYGZsq{}}\PYG{p}{,}\PYG{n}{disabled}\PYG{o}{=}\PYG{k+kc}{False}\PYG{p}{)}

\PYG{n}{A} \PYG{o}{=} \PYG{n}{widgets}\PYG{o}{.}\PYG{n}{BoundedFloatText}\PYG{p}{(}\PYG{n}{value}\PYG{o}{=}\PYG{l+m+mi}{100}\PYG{p}{,}\PYG{n+nb}{min}\PYG{o}{=}\PYG{l+m+mi}{0}\PYG{p}{,}\PYG{n+nb}{max}\PYG{o}{=}\PYG{l+m+mf}{1000.0}\PYG{p}{,}\PYG{n}{step}\PYG{o}{=}\PYG{l+m+mi}{1}\PYG{p}{,}\PYG{n}{description}\PYG{o}{=}\PYG{l+s+s1}{\PYGZsq{}}\PYG{l+s+s1}{M\PYGZlt{}sub\PYGZgt{}A\PYGZlt{}/sub\PYGZgt{} (kg)}\PYG{l+s+s1}{\PYGZsq{}}\PYG{p}{,}\PYG{n}{disabled}\PYG{o}{=}\PYG{k+kc}{False}\PYG{p}{)}

\PYG{n}{B} \PYG{o}{=} \PYG{n}{widgets}\PYG{o}{.}\PYG{n}{BoundedFloatText}\PYG{p}{(}\PYG{n}{value}\PYG{o}{=}\PYG{l+m+mi}{5}\PYG{p}{,}\PYG{n+nb}{min}\PYG{o}{=}\PYG{l+m+mi}{0}\PYG{p}{,}\PYG{n+nb}{max}\PYG{o}{=}\PYG{l+m+mf}{1000.0}\PYG{p}{,}\PYG{n}{step}\PYG{o}{=}\PYG{l+m+mi}{1}\PYG{p}{,}\PYG{n}{description}\PYG{o}{=}\PYG{l+s+s1}{\PYGZsq{}}\PYG{l+s+s1}{M\PYGZlt{}sub\PYGZgt{}B\PYGZlt{}/sub\PYGZgt{} (kg)}\PYG{l+s+s1}{\PYGZsq{}}\PYG{p}{,}\PYG{n}{disabled}\PYG{o}{=}\PYG{k+kc}{False}\PYG{p}{)}

\PYG{n}{C} \PYG{o}{=} \PYG{n}{widgets}\PYG{o}{.}\PYG{n}{BoundedFloatText}\PYG{p}{(}\PYG{n}{value}\PYG{o}{=}\PYG{l+m+mi}{10}\PYG{p}{,}\PYG{n+nb}{min}\PYG{o}{=}\PYG{l+m+mi}{0}\PYG{p}{,}\PYG{n+nb}{max}\PYG{o}{=}\PYG{l+m+mi}{1000}\PYG{p}{,}\PYG{n}{step}\PYG{o}{=}\PYG{l+m+mf}{0.1}\PYG{p}{,}\PYG{n}{description}\PYG{o}{=}\PYG{l+s+s1}{\PYGZsq{}}\PYG{l+s+s1}{M\PYGZlt{}sub\PYGZgt{}C\PYGZlt{}/sub\PYGZgt{} (kg)}\PYG{l+s+s1}{\PYGZsq{}}\PYG{p}{,}\PYG{n}{disabled}\PYG{o}{=}\PYG{k+kc}{False}\PYG{p}{)}

\PYG{n}{RA} \PYG{o}{=} \PYG{n}{widgets}\PYG{o}{.}\PYG{n}{BoundedFloatText}\PYG{p}{(}\PYG{n}{value}\PYG{o}{=}\PYG{l+m+mf}{0.2}\PYG{p}{,}\PYG{n+nb}{min}\PYG{o}{=}\PYG{l+m+mi}{0}\PYG{p}{,}\PYG{n+nb}{max}\PYG{o}{=}\PYG{l+m+mi}{100}\PYG{p}{,}\PYG{n}{step}\PYG{o}{=}\PYG{l+m+mf}{0.1}\PYG{p}{,}\PYG{n}{description}\PYG{o}{=}\PYG{l+s+s1}{\PYGZsq{}}\PYG{l+s+s1}{R\PYGZlt{}sub\PYGZgt{}A\PYGZlt{}/sub\PYGZgt{} (day\PYGZlt{}sup\PYGZgt{}\PYGZhy{}1 \PYGZlt{}/sup\PYGZgt{})}\PYG{l+s+s1}{\PYGZsq{}}\PYG{p}{,}\PYG{n}{disabled}\PYG{o}{=}\PYG{k+kc}{False}\PYG{p}{)}

\PYG{n}{RB} \PYG{o}{=} \PYG{n}{widgets}\PYG{o}{.}\PYG{n}{BoundedFloatText}\PYG{p}{(}\PYG{n}{value}\PYG{o}{=}\PYG{l+m+mf}{0.2}\PYG{p}{,}\PYG{n+nb}{min}\PYG{o}{=}\PYG{l+m+mi}{0}\PYG{p}{,}\PYG{n+nb}{max}\PYG{o}{=}\PYG{l+m+mi}{100}\PYG{p}{,}\PYG{n}{step}\PYG{o}{=}\PYG{l+m+mf}{0.1}\PYG{p}{,}\PYG{n}{description}\PYG{o}{=}\PYG{l+s+s1}{\PYGZsq{}}\PYG{l+s+s1}{R\PYGZlt{}sub\PYGZgt{}B\PYGZlt{}/sub\PYGZgt{} (day\PYGZlt{}sup\PYGZgt{}\PYGZhy{}1 \PYGZlt{}/sup\PYGZgt{})}\PYG{l+s+s1}{\PYGZsq{}}\PYG{p}{,}\PYG{n}{disabled}\PYG{o}{=}\PYG{k+kc}{False}\PYG{p}{)}


\PYG{n}{interactive\PYGZus{}plot} \PYG{o}{=} \PYG{n}{widgets}\PYG{o}{.}\PYG{n}{interactive}\PYG{p}{(}\PYG{n}{mass\PYGZus{}bal}\PYG{p}{,} \PYG{n}{n\PYGZus{}simulation} \PYG{o}{=} \PYG{n}{N}\PYG{p}{,} \PYG{n}{MA}\PYG{o}{=}\PYG{n}{A}\PYG{p}{,} \PYG{n}{MB}\PYG{o}{=}\PYG{n}{B}\PYG{p}{,} \PYG{n}{MC}\PYG{o}{=}\PYG{n}{C}\PYG{p}{,} \PYG{n}{R\PYGZus{}A}\PYG{o}{=}\PYG{n}{RA}\PYG{p}{,} \PYG{n}{R\PYGZus{}B}\PYG{o}{=}\PYG{n}{RB}\PYG{p}{,}\PYG{p}{)}
\PYG{n}{output} \PYG{o}{=} \PYG{n}{interactive\PYGZus{}plot}\PYG{o}{.}\PYG{n}{children}\PYG{p}{[}\PYG{o}{\PYGZhy{}}\PYG{l+m+mi}{1}\PYG{p}{]}  
\PYG{c+c1}{\PYGZsh{}output.layout.height = \PYGZsq{}350px\PYGZsq{}}
\PYG{n}{interactive\PYGZus{}plot}
\end{sphinxVerbatim}

\begin{sphinxVerbatim}[commandchars=\\\{\}]
interactive(children=(BoundedIntText(value=20, description=\PYGZsq{}\PYGZam{}Delta; t (day)\PYGZsq{}), BoundedFloatText(value=100.0, d…
\end{sphinxVerbatim}


\subsection{Comparison of Mass and Volume Budgets}
\label{\detokenize{contents/background/03_basic_hydrogeology:comparison-of-mass-and-volume-budgets}}
\sphinxstylestrong{mass budget}:	\(\Delta M = J_{in} \cdot \Delta t - J_{out} \cdot \Delta t\)

\sphinxstylestrong{volume budget}:	\(\Delta V = Q_{in} \cdot \Delta t - Q_{out} \cdot \Delta t \)
\begin{itemize}
\item {} 
Mass and volume budgets are equivalent if there is no change of density \(\rho\) {[}M/L\(^3\){]} with time. In this case the well known relationship \(\Delta M\) = \(\rho \cdot \Delta V\) holds and each equation given above can be directly transformed into the other one.

\item {} 
If density changes have to be considered (e.g. for gas flow), the mass budget equation remains valid but the volume budget equation must be modified because \(\Delta M = \rho \cdot \Delta V + \Delta \rho \cdot V\) with \(\Delta \rho\)= change in density.

\item {} 
Cases with changing density have proven to be more easily tractable if the mass budget equation is used.

\end{itemize}

\begin{sphinxVerbatim}[commandchars=\\\{\}]
\PYG{k+kn}{import} \PYG{n+nn}{matplotlib}\PYG{n+nn}{.}\PYG{n+nn}{pyplot} \PYG{k}{as} \PYG{n+nn}{plt}
\PYG{k+kn}{import} \PYG{n+nn}{numpy} \PYG{k}{as} \PYG{n+nn}{np}
\PYG{k+kn}{import} \PYG{n+nn}{pandas} \PYG{k}{as} \PYG{n+nn}{pd}
\PYG{k+kn}{import} \PYG{n+nn}{ipywidgets} \PYG{k}{as} \PYG{n+nn}{widgets}
\PYG{o}{\PYGZpc{}}\PYG{k}{matplotlib} widget
\PYG{k+kn}{import} \PYG{n+nn}{warnings}\PYG{p}{;} \PYG{n}{warnings}\PYG{o}{.}\PYG{n}{simplefilter}\PYG{p}{(}\PYG{l+s+s1}{\PYGZsq{}}\PYG{l+s+s1}{ignore}\PYG{l+s+s1}{\PYGZsq{}}\PYG{p}{)}
\end{sphinxVerbatim}


\section{Subsurface Structure}
\label{\detokenize{contents/flow/12_subsurface_structure:subsurface-structure}}\label{\detokenize{contents/flow/12_subsurface_structure::doc}}

\subsection{Porous Media}
\label{\detokenize{contents/flow/12_subsurface_structure:porous-media}}
The general definition of the porous media is a \sphinxstylestrong{solid which contains voids}. This definition applies to the subsurface contains solid material plus voids which represent storage and transmission of the water. The voids may have various shapes and contain fluids (mostly air and/or water). Moreover, voids may be connected to or disconnected from each other.
Generally voids and their properties are important to determine water storage (how much water is or could be available?) and water transmission (How fast the water can move?).


\subsection{Types of porous media in the subsurface}
\label{\detokenize{contents/flow/12_subsurface_structure:types-of-porous-media-in-the-subsurface}}
\noindent{\hspace*{\fill}\sphinxincludegraphics[height=400\sphinxpxdimen]{{L02_fig1}.png}\hspace*{\fill}}
\begin{enumerate}
\sphinxsetlistlabels{\arabic}{enumi}{enumii}{}{.}%
\item {} 
\sphinxstylestrong{unconsolidated porous medium (Sediments)}: it is non\sphinxhyphen{}cemented porous media and the grains can be taken away. The formation of such porous media is due to deposition of solid material mostly by water.

\end{enumerate}

\noindent{\hspace*{\fill}\sphinxincludegraphics[height=300\sphinxpxdimen]{{L02_fig3}.png}\hspace*{\fill}}
\begin{enumerate}
\sphinxsetlistlabels{\arabic}{enumi}{enumii}{}{.}%
\item {} 
\sphinxstylestrong{Consolidated porous medium (Rocks)}: the formation is due to increased pressure acting together with thermal and chemical processes. It has two types:

\end{enumerate}
\begin{quote}

Fractured porous media
\end{quote}
\begin{quote}

Karstified porous media
\end{quote}

\noindent{\hspace*{\fill}\sphinxincludegraphics[height=600\sphinxpxdimen]{{L02_fig4}.png}\hspace*{\fill}}


\subsection{Porosity (Total porosity):}
\label{\detokenize{contents/flow/12_subsurface_structure:porosity-total-porosity}}
Is defined as the volumetric share of voids in a porous media. It is a number between 0 and 1 and can be expressed as percentage (0\%: no voids, 100\%:no solid)
\begin{equation*}
\begin{split}{n}=\frac{V_{v}}{V_{T}}\end{split}
\end{equation*}\begin{itemize}
\item {} 
\(n\)= total porosity

\item {} 
\({V_{v}}\)= voids volume

\item {} 
\({V_{T}}\)= total volume

\end{itemize}

\sphinxstylestrong{Sample question}

If the total volume of a media is 254 cubic meters, and the volume of the void is 27 cubic meters, what is the porosity (give as a percent)?

\begin{sphinxVerbatim}[commandchars=\\\{\}]
\PYG{c+c1}{\PYGZsh{} input data}
\PYG{n}{V\PYGZus{}T}\PYG{o}{=} \PYG{l+m+mi}{254} \PYG{c+c1}{\PYGZsh{}m\PYGZca{}3 total volume}
\PYG{n}{V\PYGZus{}v}\PYG{o}{=}\PYG{l+m+mi}{27} \PYG{c+c1}{\PYGZsh{}m\PYGZca{}3 voids volume}

\PYG{c+c1}{\PYGZsh{}calculations}
\PYG{n}{n}\PYG{o}{=}\PYG{p}{(}\PYG{n}{V\PYGZus{}v}\PYG{o}{/}\PYG{n}{V\PYGZus{}T}\PYG{p}{)}\PYG{o}{*}\PYG{l+m+mi}{100}

\PYG{c+c1}{\PYGZsh{} Output}
\PYG{n+nb}{print}\PYG{p}{(}\PYG{l+s+s2}{\PYGZdq{}}\PYG{l+s+s2}{ Total porosity is:}\PYG{l+s+s2}{\PYGZdq{}}\PYG{p}{,} \PYG{n}{n}\PYG{p}{,} \PYG{l+s+s2}{\PYGZdq{}}\PYG{l+s+s2}{\PYGZpc{}}\PYG{l+s+s2}{\PYGZdq{}} \PYG{p}{)}
\end{sphinxVerbatim}

\begin{sphinxVerbatim}[commandchars=\\\{\}]
 Total porosity is: 10.62992125984252 \PYGZpc{}
\end{sphinxVerbatim}


\subsubsection{Total porosity of artificial porous media:}
\label{\detokenize{contents/flow/12_subsurface_structure:total-porosity-of-artificial-porous-media}}
If “grains” have identical shape and are regularly arranged, it is possible to exactly compute total porosity, the pores should have the same size.

\noindent{\hspace*{\fill}\sphinxincludegraphics[height=300\sphinxpxdimen]{{L02_fig5}.png}\hspace*{\fill}}
\begin{itemize}
\item {} 
Loose packing(first picture): each hole placed on top of the hole underneath

\item {} 
Dense packing (second picture): each hole is placed at the deepest position possible

\end{itemize}

These schematics provides a practical range of porosity in the subsurface. The general range is between 25\% to about 50\%. In more extreme cases porosity higher than 60\% is possible, e.g., cobbles, gravel. The other extreme, subsurface with no porosity (0\%) is also encountered in the subsurface, e.g., in consolidated rocks.


\subsubsection{Total porosity of natural unconsolidated porous media:}
\label{\detokenize{contents/flow/12_subsurface_structure:total-porosity-of-natural-unconsolidated-porous-media}}
In the subsurface (natural unconsolidated porous media) there are grains of different size and total porosity depends on the grain size distribution.


\begin{savenotes}\sphinxattablestart
\centering
\begin{tabulary}{\linewidth}[t]{|T|T|T|}
\hline


&\sphinxstyletheadfamily 
grain  diameter  (mm)
&\sphinxstyletheadfamily 
Total  Porosity (\%)
\\
\hline
Coarse gravel
&
20 \sphinxhyphen{} 60
&
24 \sphinxhyphen{} 36
\\
\hline
Fine gravel
&
2 \sphinxhyphen{} 6
&
25 \sphinxhyphen{} 38
\\
\hline
Coarse sand
&
0.6 \sphinxhyphen{} 2
&
31 \sphinxhyphen{} 46
\\
\hline
Fine sand
&
0.006 \sphinxhyphen{} 0.2
&
26 \sphinxhyphen{} 53
\\
\hline
Silt
&
0.002 \sphinxhyphen{} 0.06
&
34 \sphinxhyphen{} 61
\\
\hline
Clay
&
\textless{} 0.002
&
34 \sphinxhyphen{} 60
\\
\hline
\end{tabulary}
\par
\sphinxattableend\end{savenotes}


\begin{savenotes}\sphinxattablestart
\centering
\begin{tabulary}{\linewidth}[t]{|T|T|}
\hline


&\sphinxstyletheadfamily 
Total  Porosity (\%)
\\
\hline
Siltstone
&
21 \sphinxhyphen{} 41
\\
\hline
Sandstone
&
5 \sphinxhyphen{} 30
\\
\hline
Basalt
&
3 \sphinxhyphen{} 35
\\
\hline
Claystone
&
1 \sphinxhyphen{} 10
\\
\hline
Limestone
&
1 \sphinxhyphen{} 10
\\
\hline
Shale
&
\textless{} 10
\\
\hline
Granite
&
\textless{} 1
\\
\hline
\end{tabulary}
\par
\sphinxattableend\end{savenotes}

Total porosity of consolidated porous media (rocks) is usually smaller than total porosity of unconsolidated porous media. However, weathering effect may lead to increase the value of porosity.
only for unconsolidated porous media, total porosity tends to increase with decreasing grain size.


\section{Grain size distribution of unconsolidated porous media}
\label{\detokenize{contents/flow/12_subsurface_structure:grain-size-distribution-of-unconsolidated-porous-media}}
Unconsolidated porous media are able to store and transmit water that can be influenced by grain size distribution. Therefore, the grain size distribution is frequently determined in laboratory experiments in order to deduce important flow properties.
There are five major grain size classes (observed by increasing diameter): clay, silt, sand, and gravel (or debris). The classes for silt, clay and gavel are usually subdivided by “fine”, “medium”, and “coarse” (or “very fine”, “fine”, “medium”, “coarse”, and “very coarse”). Different ranges for individual grain size classes have been defined by different authorities or regulations. However, the standard method to determine the grain size distribution of a sample is sieve analysis.


\subsection{Classification schemes:}
\label{\detokenize{contents/flow/12_subsurface_structure:classification-schemes}}
The diagrams below include a couple of classification schemes to define ranges of grain diameter for clay, silt, sand, and gravel:

\noindent{\hspace*{\fill}\sphinxincludegraphics[height=400\sphinxpxdimen]{{L02_fig7}.png}\hspace*{\fill}}

\sphinxstyleemphasis{USDA: United States Department of Agriculture}, \sphinxstyleemphasis{ISSS: International Soil Science Society (ISSS)}, \sphinxstyleemphasis{MIT: Massachusetts Institute of Technology}, \sphinxstyleemphasis{ASTM: American Society for Testing and Materials}, \sphinxstyleemphasis{AASHTO: American Association of State Highway and Transportation Officials}, \sphinxstyleemphasis{FAA: Federal Aviation Administration}

As can be observed that there exist several standards. These are often based on local requirements e.g., based on countries. In Germany the DIN standards are used.


\subsection{Sieve analysis:}
\label{\detokenize{contents/flow/12_subsurface_structure:sieve-analysis}}
The results from a sample consist of different grain size fractions should be transferred on granulometric curve. This curve provides cumulative information; vertical axis shows the mass fraction, and horizontal axis shows the grain diameter. For example, if 1mm grain diameter has 80\% of cumulative mass fraction it means that 80\% of this sample contains 1mm grain diameter or less than 1 mm (see the picture below).

\sphinxstylestrong{How to get granulometric curve?}
In order to perform sieve analysis we can use sieve machine. Sieve machine consist of sets of sieves from coarse sieve on top to fine sieve and a cup at the bottom. The mechanism is to shake the set. Finally, each sieve consists of grain sizes which are bigger than the sieve.

\noindent{\hspace*{\fill}\sphinxincludegraphics[height=400\sphinxpxdimen]{{L02_fig8}.png}\hspace*{\fill}}

\sphinxstylestrong{dx and U}: From the granulometric curve, several parameters can be determined in order to characterize the sample. \({d_{x}}\)   denotes the grain diameter for which x\% (in mass or weight, not volume) of the sieve material is smaller than this diameter.

\noindent{\hspace*{\fill}\sphinxincludegraphics[height=400\sphinxpxdimen]{{L02_fig9}.png}\hspace*{\fill}}

Grain diameters \({d_{10}}, {d_{60}}, {d_{75}}\) are of practical importance with regard to groundwater flow properties. The     ratio of d60 and d10 is called \sphinxstylestrong{coefficient of uniformity, U}:
\begin{equation*}
\begin{split}{U}=\frac{d_{60}}{d_{10}}\end{split}
\end{equation*}
\({d_{75}}\) is specifically used for well construction purpose (not covered by this lecture)


\section{Subterranean water}
\label{\detokenize{contents/flow/12_subsurface_structure:subterranean-water}}
The subsurface can be regarded as a three\sphinxhyphen{}phase system consisting of a solid phase (soil particles), a water phase, and a gas phase. a schematic illustration for voids or pores in an unconsolidated porous medium is given in the figure below. Each phase has similar density and other properties. Sometimes it is possible for the fourth phase which is contamination.Voids are filled with water and gas. The volumetric ratio of water in voids can be calculated by water content.

\noindent{\hspace*{\fill}\sphinxincludegraphics[height=200\sphinxpxdimen]{{L02_fig10}.png}\hspace*{\fill}}


\subsection{Water content:}
\label{\detokenize{contents/flow/12_subsurface_structure:water-content}}
Water content is defined as the share of water in the porous medium:
\begin{equation*}
\begin{split}{\theta}=\frac{V_{w}}{V_{T}}\end{split}
\end{equation*}\begin{itemize}
\item {} 
\({\theta}\)= water content

\item {} 
\({V_{w}}\)= water volume

\item {} 
\({V_{T}}\)= total volume

\end{itemize}

Water content cannot exceed the total porosity. i.e. θ≤n ( total porosity is independent of the fluid content of porous medium).


\subsection{Degree of saturation:}
\label{\detokenize{contents/flow/12_subsurface_structure:degree-of-saturation}}
Another way to express the ratio of water in the porous medium is the degree of saturation, i.e. the ratio of water volume to void volume:
\begin{equation*}
\begin{split}{S}=\frac{V_{w}}{V_{v}}\end{split}
\end{equation*}\begin{itemize}
\item {} 
\({S}\)= degree of saturation

\item {} 
\({V_{w}}\)= water volume

\item {} 
\({V_{v}}\)= voids volume

\end{itemize}

The degree of saturation is equal to \(\frac{θ}{n}\) . S can vary between 0 to 1 (or between 0\% to 100\%), S=0 means no water in the voids, whereas S=100 means voids are completely filled with water.

\sphinxstylestrong{sample question}


\subsection{Forcing act on subterranean water:}
\label{\detokenize{contents/flow/12_subsurface_structure:forcing-act-on-subterranean-water}}
Subterranean water is subject to several forces. The most important ones are:
\begin{itemize}
\item {} 
gravity

\item {} 
attractive forces between the water molecules (cohesion)

\item {} 
attractive forces between water and solids (adhesion)

\end{itemize}

\noindent{\hspace*{\fill}\sphinxincludegraphics[height=250\sphinxpxdimen]{{L02_fig11}.png}\hspace*{\fill}}

In the figure above, dotted area represent the solid phase. In the pore channel the dominant force is gravity, shown as G. getting closer to the solid surface, adhesive force become more important. The numbers indicate the required pressure to remove the corresponding layer of water from the solid surface. As an example, in order to remove the last layer of water from the solid surface, 31 bar pressure needs to be applied. Another easy way to remove the water is boiling the sample in the oven.


\subsection{Surface tension:}
\label{\detokenize{contents/flow/12_subsurface_structure:surface-tension}}
Cohesive forces acting on water molecules compensate each other if the molecule is not located near water\sphinxhyphen{}air or water\sphinxhyphen{}solid interface. This is no longer true at an interface: cohesive interaction is reduced on one side. The resulting force tends to minimize the interface area. Macroscopically, this effect is parametrized by the “surface tension”, which is defined as the energy needed to increase the area of the interface by one unit.

\noindent{\hspace*{\fill}\sphinxincludegraphics[height=300\sphinxpxdimen]{{L02_fig12}.png}\hspace*{\fill}}

Common units of the surface tension σ are \(\frac{J}{m^2}\) or \(\frac{N}{m}\) (Its dimension is \(\frac{M}{T^2}\)). The surface tension of water is about 7.5 . 10 \sphinxhyphen{}2 \(\frac{N}{m}\) at 10 ֯C.


\subsection{Capillary action:}
\label{\detokenize{contents/flow/12_subsurface_structure:capillary-action}}
\noindent{\hspace*{\fill}\sphinxincludegraphics[height=300\sphinxpxdimen]{{L02_fig13}.png}\hspace*{\fill}}

Water is subject to capillary action when adhesion is strongr than cohesion. The capillary rise of water in a tube depends on the surface tension and the tube redius. The maximum capillary rise is given by:
\begin{equation*}
\begin{split}{h_{c}}=\frac{2\sigma{w}}{\rho_{w}{g}{r}}\end{split}
\end{equation*}\begin{itemize}
\item {} 
\({h_{c}}\)= maximum capillary rise

\item {} 
\(sigma{w}\)= surface tension

\item {} 
\(\rho_{w}\)= water density

\item {} 
\({g}\)= acceleration of gravity

\item {} 
\({r}\)= radius of the tube

\end{itemize}


\subsection{Capillary action in the subsurface:}
\label{\detokenize{contents/flow/12_subsurface_structure:capillary-action-in-the-subsurface}}
Capillary actions play a dominant role in the subsurface. The capillaries are given by individual pore channels. Poor channels in poorly sorted material may strongly differ in diameter, such that a certain variability in capillary rise is observed.

\noindent{\hspace*{\fill}\sphinxincludegraphics[height=400\sphinxpxdimen]{{L02_fig14}.png}\hspace*{\fill}}

Left sketch shows the capillary rise in a perfectly sorted material which all the pores have the same size. So capillary rise is similar in every single pores. The right sketch, shows a real situation of subsurface. There are different grain size and then different pore channels, which results in various capillary rise.

\sphinxstylestrong{Sample question}
For water at at a tube with reduce of R, the surface tension is 73\(\frac{g}{s^2}\), the density is 0.999 \(\frac{g}{cm^3}\). Compute the rise of water in the capillary tube

\begin{sphinxVerbatim}[commandchars=\\\{\}]
\PYG{c+c1}{\PYGZsh{} input data}
\PYG{n}{sigma}\PYG{o}{=} \PYG{l+m+mi}{73} \PYG{c+c1}{\PYGZsh{}g/s\PYGZca{}2 surface tension}
\PYG{n}{rho}\PYG{o}{=} \PYG{l+m+mf}{0.999} \PYG{c+c1}{\PYGZsh{} g/cm\PYGZca{}3 water density}
\PYG{n}{g}\PYG{o}{=}\PYG{l+m+mi}{980} \PYG{c+c1}{\PYGZsh{}cm/s\PYGZca{}2 acceleration of gravity}

\PYG{c+c1}{\PYGZsh{}calculation}
\PYG{n}{h\PYGZus{}c}\PYG{o}{=}\PYG{p}{(}\PYG{l+m+mi}{2}\PYG{o}{*}\PYG{n}{sigma}\PYG{p}{)}\PYG{o}{/}\PYG{p}{(}\PYG{n}{rho}\PYG{o}{*}\PYG{n}{g}\PYG{p}{)}

\PYG{c+c1}{\PYGZsh{}output}
\PYG{n+nb}{print}\PYG{p}{(}\PYG{l+s+s2}{\PYGZdq{}}\PYG{l+s+s2}{the maximum water rise in this tube is:}\PYG{l+s+s2}{\PYGZdq{}}\PYG{p}{,} \PYG{n}{h\PYGZus{}c}\PYG{p}{,}\PYG{l+s+s2}{\PYGZdq{}}\PYG{l+s+s2}{1/R}\PYG{l+s+s2}{\PYGZdq{}}\PYG{p}{,}  \PYG{l+s+s2}{\PYGZdq{}}\PYG{l+s+s2}{cm}\PYG{l+s+s2}{\PYGZdq{}}\PYG{p}{)}
\end{sphinxVerbatim}

\begin{sphinxVerbatim}[commandchars=\\\{\}]
the maximum water rise in this tube is: 0.14912872055729198 1/R cm
\end{sphinxVerbatim}

\begin{sphinxVerbatim}[commandchars=\\\{\}]
\PYG{k+kn}{import} \PYG{n+nn}{numpy} \PYG{k}{as} \PYG{n+nn}{np}
\PYG{k+kn}{from} \PYG{n+nn}{scipy}\PYG{n+nn}{.}\PYG{n+nn}{integrate} \PYG{k+kn}{import} \PYG{n}{odeint}
\PYG{k+kn}{import} \PYG{n+nn}{matplotlib}\PYG{n+nn}{.}\PYG{n+nn}{pyplot} \PYG{k}{as} \PYG{n+nn}{plt} 
\PYG{k+kn}{import} \PYG{n+nn}{pandas} \PYG{k}{as} \PYG{n+nn}{pd} 
\PYG{k+kn}{import} \PYG{n+nn}{panel} \PYG{k}{as} \PYG{n+nn}{pn}
\PYG{k+kn}{import} \PYG{n+nn}{ipywidgets} \PYG{k}{as} \PYG{n+nn}{widgets}
\end{sphinxVerbatim}


\section{Simulating Mass budget}
\label{\detokenize{contents/tools/decay:simulating-mass-budget}}\label{\detokenize{contents/tools/decay::doc}}
\begin{sphinxVerbatim}[commandchars=\\\{\}]
\PYG{k}{def} \PYG{n+nf}{mass\PYGZus{}bal}\PYG{p}{(}\PYG{n}{n\PYGZus{}simulation}\PYG{p}{,} \PYG{n}{MA}\PYG{p}{,} \PYG{n}{MB}\PYG{p}{,} \PYG{n}{MC}\PYG{p}{,} \PYG{n}{R\PYGZus{}A}\PYG{p}{,} \PYG{n}{R\PYGZus{}B}\PYG{p}{)}\PYG{p}{:}
    
    \PYG{n}{A} \PYG{o}{=} \PYG{n}{np}\PYG{o}{.}\PYG{n}{zeros}\PYG{p}{(}\PYG{n}{n\PYGZus{}simulation}\PYG{p}{)} \PYG{c+c1}{\PYGZsh{} creat an array with zros}
    \PYG{n}{B} \PYG{o}{=} \PYG{n}{np}\PYG{o}{.}\PYG{n}{zeros}\PYG{p}{(}\PYG{n}{n\PYGZus{}simulation}\PYG{p}{)}
    \PYG{n}{C} \PYG{o}{=} \PYG{n}{np}\PYG{o}{.}\PYG{n}{zeros}\PYG{p}{(}\PYG{n}{n\PYGZus{}simulation}\PYG{p}{)} 
    \PYG{n}{time}  \PYG{o}{=} \PYG{n}{np}\PYG{o}{.}\PYG{n}{arange}\PYG{p}{(}\PYG{n}{n\PYGZus{}simulation}\PYG{p}{)}
    
    \PYG{k}{for} \PYG{n}{i} \PYG{o+ow}{in} \PYG{n+nb}{range}\PYG{p}{(}\PYG{l+m+mi}{0}\PYG{p}{,}\PYG{n}{n\PYGZus{}simulation}\PYG{o}{\PYGZhy{}}\PYG{l+m+mi}{1}\PYG{p}{)}\PYG{p}{:}
        \PYG{n}{A}\PYG{p}{[}\PYG{l+m+mi}{0}\PYG{p}{]} \PYG{o}{=} \PYG{n}{MA}  \PYG{c+c1}{\PYGZsh{} starting input value}
        
        \PYG{n}{B}\PYG{p}{[}\PYG{l+m+mi}{0}\PYG{p}{]} \PYG{o}{=} \PYG{n}{MB}
        \PYG{n}{C}\PYG{p}{[}\PYG{l+m+mi}{0}\PYG{p}{]} \PYG{o}{=} \PYG{n}{MC}
        \PYG{n}{A}\PYG{p}{[}\PYG{n}{i}\PYG{o}{+}\PYG{l+m+mi}{1}\PYG{p}{]} \PYG{o}{=} \PYG{n}{A}\PYG{p}{[}\PYG{n}{i}\PYG{p}{]}\PYG{o}{\PYGZhy{}}\PYG{n}{R\PYGZus{}A}\PYG{o}{*}\PYG{n}{A}\PYG{p}{[}\PYG{n}{i}\PYG{p}{]}
        \PYG{n}{B}\PYG{p}{[}\PYG{n}{i}\PYG{o}{+}\PYG{l+m+mi}{1}\PYG{p}{]} \PYG{o}{=} \PYG{n}{B}\PYG{p}{[}\PYG{n}{i}\PYG{p}{]}\PYG{o}{+}\PYG{n}{R\PYGZus{}A}\PYG{o}{*}\PYG{n}{A}\PYG{p}{[}\PYG{n}{i}\PYG{p}{]}\PYG{o}{\PYGZhy{}}\PYG{n}{R\PYGZus{}B}\PYG{o}{*}\PYG{n}{B}\PYG{p}{[}\PYG{n}{i}\PYG{p}{]} 
        \PYG{n}{C}\PYG{p}{[}\PYG{n}{i}\PYG{o}{+}\PYG{l+m+mi}{1}\PYG{p}{]} \PYG{o}{=} \PYG{n}{C}\PYG{p}{[}\PYG{n}{i}\PYG{p}{]}\PYG{o}{+}\PYG{n}{R\PYGZus{}B}\PYG{o}{*}\PYG{n}{B}\PYG{p}{[}\PYG{n}{i}\PYG{p}{]}
        \PYG{n}{summ} \PYG{o}{=} \PYG{n}{A}\PYG{p}{[}\PYG{n}{i}\PYG{p}{]}\PYG{o}{+}\PYG{n}{B}\PYG{p}{[}\PYG{n}{i}\PYG{p}{]}\PYG{o}{+}\PYG{n}{C}\PYG{p}{[}\PYG{n}{i}\PYG{p}{]}
        
    \PYG{n}{d} \PYG{o}{=} \PYG{p}{\PYGZob{}}\PYG{l+s+s2}{\PYGZdq{}}\PYG{l+s+s2}{Mass\PYGZus{}A}\PYG{l+s+s2}{\PYGZdq{}}\PYG{p}{:} \PYG{n}{A}\PYG{p}{,} \PYG{l+s+s2}{\PYGZdq{}}\PYG{l+s+s2}{Mass\PYGZus{}B}\PYG{l+s+s2}{\PYGZdq{}}\PYG{p}{:} \PYG{n}{B}\PYG{p}{,} \PYG{l+s+s2}{\PYGZdq{}}\PYG{l+s+s2}{Mass\PYGZus{}C}\PYG{l+s+s2}{\PYGZdq{}}\PYG{p}{:} \PYG{n}{C}\PYG{p}{,} \PYG{l+s+s2}{\PYGZdq{}}\PYG{l+s+s2}{Total Mass}\PYG{l+s+s2}{\PYGZdq{}}\PYG{p}{:} \PYG{n}{summ}\PYG{p}{\PYGZcb{}}
    \PYG{n}{df} \PYG{o}{=} \PYG{n}{pd}\PYG{o}{.}\PYG{n}{DataFrame}\PYG{p}{(}\PYG{n}{d}\PYG{p}{)} \PYG{c+c1}{\PYGZsh{} Generating result table}
    \PYG{n}{label} \PYG{o}{=} \PYG{p}{[}\PYG{l+s+s2}{\PYGZdq{}}\PYG{l+s+s2}{Mass A (g)}\PYG{l+s+s2}{\PYGZdq{}}\PYG{p}{,} \PYG{l+s+s2}{\PYGZdq{}}\PYG{l+s+s2}{Mass B (g)}\PYG{l+s+s2}{\PYGZdq{}}\PYG{p}{,} \PYG{l+s+s2}{\PYGZdq{}}\PYG{l+s+s2}{Mass C (g)}\PYG{l+s+s2}{\PYGZdq{}}\PYG{p}{]}
    \PYG{n}{fig} \PYG{o}{=} \PYG{n}{plt}\PYG{o}{.}\PYG{n}{figure}\PYG{p}{(}\PYG{n}{figsize}\PYG{o}{=}\PYG{p}{(}\PYG{l+m+mi}{6}\PYG{p}{,}\PYG{l+m+mi}{4}\PYG{p}{)}\PYG{p}{)}
    \PYG{n}{plt}\PYG{o}{.}\PYG{n}{plot}\PYG{p}{(}\PYG{n}{time}\PYG{p}{,} \PYG{n}{A}\PYG{p}{,} \PYG{n}{time}\PYG{p}{,} \PYG{n}{B}\PYG{p}{,} \PYG{n}{time}\PYG{p}{,} \PYG{n}{C}\PYG{p}{,} \PYG{n}{linewidth}\PYG{o}{=}\PYG{l+m+mi}{3}\PYG{p}{)}\PYG{p}{;}  \PYG{c+c1}{\PYGZsh{} plotting the results}
    \PYG{n}{plt}\PYG{o}{.}\PYG{n}{xlabel}\PYG{p}{(}\PYG{l+s+s2}{\PYGZdq{}}\PYG{l+s+s2}{Time [Time Unit]}\PYG{l+s+s2}{\PYGZdq{}}\PYG{p}{)}\PYG{p}{;} \PYG{n}{plt}\PYG{o}{.}\PYG{n}{ylabel}\PYG{p}{(}\PYG{l+s+s2}{\PYGZdq{}}\PYG{l+s+s2}{Mass [g]}\PYG{l+s+s2}{\PYGZdq{}}\PYG{p}{)} \PYG{c+c1}{\PYGZsh{} placing axis labels}
    \PYG{n}{plt}\PYG{o}{.}\PYG{n}{legend}\PYG{p}{(}\PYG{n}{label}\PYG{p}{,} \PYG{n}{loc}\PYG{o}{=}\PYG{l+m+mi}{0}\PYG{p}{)}\PYG{p}{;}\PYG{n}{plt}\PYG{o}{.}\PYG{n}{grid}\PYG{p}{(}\PYG{p}{)}\PYG{p}{;} \PYG{n}{plt}\PYG{o}{.}\PYG{n}{xlim}\PYG{p}{(}\PYG{p}{[}\PYG{l+m+mi}{0}\PYG{p}{,}\PYG{n}{n\PYGZus{}simulation}\PYG{p}{]}\PYG{p}{)}\PYG{p}{;} \PYG{n}{plt}\PYG{o}{.}\PYG{n}{ylim}\PYG{p}{(}\PYG{n}{bottom}\PYG{o}{=}\PYG{l+m+mi}{0}\PYG{p}{)} \PYG{c+c1}{\PYGZsh{} legends, grids, x,y limits}
    \PYG{n}{plt}\PYG{o}{.}\PYG{n}{show}\PYG{p}{(}\PYG{p}{)} \PYG{c+c1}{\PYGZsh{} display plot}
    
    \PYG{k}{return} \PYG{n+nb}{print}\PYG{p}{(}\PYG{n}{df}\PYG{o}{.}\PYG{n}{round}\PYG{p}{(}\PYG{l+m+mi}{2}\PYG{p}{)}\PYG{p}{)} 

\PYG{n}{N} \PYG{o}{=} \PYG{n}{widgets}\PYG{o}{.}\PYG{n}{BoundedIntText}\PYG{p}{(}\PYG{n}{value}\PYG{o}{=}\PYG{l+m+mi}{20}\PYG{p}{,}\PYG{n+nb}{min}\PYG{o}{=}\PYG{l+m+mi}{0}\PYG{p}{,}\PYG{n+nb}{max}\PYG{o}{=}\PYG{l+m+mi}{100}\PYG{p}{,}\PYG{n}{step}\PYG{o}{=}\PYG{l+m+mi}{1}\PYG{p}{,}\PYG{n}{description}\PYG{o}{=} \PYG{l+s+s1}{\PYGZsq{}}\PYG{l+s+s1}{\PYGZam{}Delta; t (day)}\PYG{l+s+s1}{\PYGZsq{}}\PYG{p}{,}\PYG{n}{disabled}\PYG{o}{=}\PYG{k+kc}{False}\PYG{p}{)}

\PYG{n}{A} \PYG{o}{=} \PYG{n}{widgets}\PYG{o}{.}\PYG{n}{BoundedFloatText}\PYG{p}{(}\PYG{n}{value}\PYG{o}{=}\PYG{l+m+mi}{100}\PYG{p}{,}\PYG{n+nb}{min}\PYG{o}{=}\PYG{l+m+mi}{0}\PYG{p}{,}\PYG{n+nb}{max}\PYG{o}{=}\PYG{l+m+mf}{1000.0}\PYG{p}{,}\PYG{n}{step}\PYG{o}{=}\PYG{l+m+mi}{1}\PYG{p}{,}\PYG{n}{description}\PYG{o}{=}\PYG{l+s+s1}{\PYGZsq{}}\PYG{l+s+s1}{M\PYGZlt{}sub\PYGZgt{}A\PYGZlt{}/sub\PYGZgt{} (kg)}\PYG{l+s+s1}{\PYGZsq{}}\PYG{p}{,}\PYG{n}{disabled}\PYG{o}{=}\PYG{k+kc}{False}\PYG{p}{)}

\PYG{n}{B} \PYG{o}{=} \PYG{n}{widgets}\PYG{o}{.}\PYG{n}{BoundedFloatText}\PYG{p}{(}\PYG{n}{value}\PYG{o}{=}\PYG{l+m+mi}{5}\PYG{p}{,}\PYG{n+nb}{min}\PYG{o}{=}\PYG{l+m+mi}{0}\PYG{p}{,}\PYG{n+nb}{max}\PYG{o}{=}\PYG{l+m+mf}{1000.0}\PYG{p}{,}\PYG{n}{step}\PYG{o}{=}\PYG{l+m+mi}{1}\PYG{p}{,}\PYG{n}{description}\PYG{o}{=}\PYG{l+s+s1}{\PYGZsq{}}\PYG{l+s+s1}{M\PYGZlt{}sub\PYGZgt{}B\PYGZlt{}/sub\PYGZgt{} (kg)}\PYG{l+s+s1}{\PYGZsq{}}\PYG{p}{,}\PYG{n}{disabled}\PYG{o}{=}\PYG{k+kc}{False}\PYG{p}{)}

\PYG{n}{C} \PYG{o}{=} \PYG{n}{widgets}\PYG{o}{.}\PYG{n}{BoundedFloatText}\PYG{p}{(}\PYG{n}{value}\PYG{o}{=}\PYG{l+m+mi}{10}\PYG{p}{,}\PYG{n+nb}{min}\PYG{o}{=}\PYG{l+m+mi}{0}\PYG{p}{,}\PYG{n+nb}{max}\PYG{o}{=}\PYG{l+m+mi}{1000}\PYG{p}{,}\PYG{n}{step}\PYG{o}{=}\PYG{l+m+mf}{0.1}\PYG{p}{,}\PYG{n}{description}\PYG{o}{=}\PYG{l+s+s1}{\PYGZsq{}}\PYG{l+s+s1}{M\PYGZlt{}sub\PYGZgt{}C\PYGZlt{}/sub\PYGZgt{} (kg)}\PYG{l+s+s1}{\PYGZsq{}}\PYG{p}{,}\PYG{n}{disabled}\PYG{o}{=}\PYG{k+kc}{False}\PYG{p}{)}

\PYG{n}{RA} \PYG{o}{=} \PYG{n}{widgets}\PYG{o}{.}\PYG{n}{BoundedFloatText}\PYG{p}{(}\PYG{n}{value}\PYG{o}{=}\PYG{l+m+mf}{0.2}\PYG{p}{,}\PYG{n+nb}{min}\PYG{o}{=}\PYG{l+m+mi}{0}\PYG{p}{,}\PYG{n+nb}{max}\PYG{o}{=}\PYG{l+m+mi}{100}\PYG{p}{,}\PYG{n}{step}\PYG{o}{=}\PYG{l+m+mf}{0.1}\PYG{p}{,}\PYG{n}{description}\PYG{o}{=}\PYG{l+s+s1}{\PYGZsq{}}\PYG{l+s+s1}{R\PYGZlt{}sub\PYGZgt{}A\PYGZlt{}/sub\PYGZgt{} (day\PYGZlt{}sup\PYGZgt{}\PYGZhy{}1 \PYGZlt{}/sup\PYGZgt{})}\PYG{l+s+s1}{\PYGZsq{}}\PYG{p}{,}\PYG{n}{disabled}\PYG{o}{=}\PYG{k+kc}{False}\PYG{p}{)}

\PYG{n}{RB} \PYG{o}{=} \PYG{n}{widgets}\PYG{o}{.}\PYG{n}{BoundedFloatText}\PYG{p}{(}\PYG{n}{value}\PYG{o}{=}\PYG{l+m+mf}{0.2}\PYG{p}{,}\PYG{n+nb}{min}\PYG{o}{=}\PYG{l+m+mi}{0}\PYG{p}{,}\PYG{n+nb}{max}\PYG{o}{=}\PYG{l+m+mi}{100}\PYG{p}{,}\PYG{n}{step}\PYG{o}{=}\PYG{l+m+mf}{0.1}\PYG{p}{,}\PYG{n}{description}\PYG{o}{=}\PYG{l+s+s1}{\PYGZsq{}}\PYG{l+s+s1}{R\PYGZlt{}sub\PYGZgt{}B\PYGZlt{}/sub\PYGZgt{} (day\PYGZlt{}sup\PYGZgt{}\PYGZhy{}1 \PYGZlt{}/sup\PYGZgt{})}\PYG{l+s+s1}{\PYGZsq{}}\PYG{p}{,}\PYG{n}{disabled}\PYG{o}{=}\PYG{k+kc}{False}\PYG{p}{)}

\PYG{n}{interactive\PYGZus{}plot} \PYG{o}{=} \PYG{n}{widgets}\PYG{o}{.}\PYG{n}{interactive}\PYG{p}{(}\PYG{n}{mass\PYGZus{}bal}\PYG{p}{,} \PYG{n}{n\PYGZus{}simulation} \PYG{o}{=} \PYG{n}{N}\PYG{p}{,} \PYG{n}{MA}\PYG{o}{=}\PYG{n}{A}\PYG{p}{,} \PYG{n}{MB}\PYG{o}{=}\PYG{n}{B}\PYG{p}{,} \PYG{n}{MC}\PYG{o}{=}\PYG{n}{C}\PYG{p}{,} \PYG{n}{R\PYGZus{}A}\PYG{o}{=}\PYG{n}{RA}\PYG{p}{,} \PYG{n}{R\PYGZus{}B}\PYG{o}{=}\PYG{n}{RB}\PYG{p}{,}\PYG{p}{)}
\PYG{n}{output} \PYG{o}{=} \PYG{n}{interactive\PYGZus{}plot}\PYG{o}{.}\PYG{n}{children}\PYG{p}{[}\PYG{o}{\PYGZhy{}}\PYG{l+m+mi}{1}\PYG{p}{]}  
\PYG{c+c1}{\PYGZsh{}output.layout.height = \PYGZsq{}350px\PYGZsq{}}
\PYG{n}{interactive\PYGZus{}plot}
\end{sphinxVerbatim}

\begin{sphinxVerbatim}[commandchars=\\\{\}]
interactive(children=(BoundedIntText(value=20, description=\PYGZsq{}\PYGZam{}Delta; t (day)\PYGZsq{}), BoundedFloatText(value=100.0, d…
\end{sphinxVerbatim}

\begin{sphinxVerbatim}[commandchars=\\\{\}]
\PYG{k+kn}{import} \PYG{n+nn}{matplotlib}\PYG{n+nn}{.}\PYG{n+nn}{pyplot} \PYG{k}{as} \PYG{n+nn}{plt}
\PYG{k+kn}{import} \PYG{n+nn}{numpy} \PYG{k}{as} \PYG{n+nn}{np}
\PYG{k+kn}{import} \PYG{n+nn}{pandas} \PYG{k}{as} \PYG{n+nn}{pd}
\PYG{k+kn}{import} \PYG{n+nn}{ipywidgets} \PYG{k}{as} \PYG{n+nn}{widgets}
\PYG{o}{\PYGZpc{}}\PYG{k}{matplotlib} widget
\PYG{k+kn}{import} \PYG{n+nn}{warnings}\PYG{p}{;} \PYG{n}{warnings}\PYG{o}{.}\PYG{n}{simplefilter}\PYG{p}{(}\PYG{l+s+s1}{\PYGZsq{}}\PYG{l+s+s1}{ignore}\PYG{l+s+s1}{\PYGZsq{}}\PYG{p}{)}
\end{sphinxVerbatim}


\section{Simulating Seive Analysis}
\label{\detokenize{contents/tools/sieve_analysis:simulating-seive-analysis}}\label{\detokenize{contents/tools/sieve_analysis::doc}}
\begin{sphinxVerbatim}[commandchars=\\\{\}]
\PYG{n+nb}{print}\PYG{p}{(}\PYG{l+s+s2}{\PYGZdq{}}\PYG{l+s+s2}{Please provide the seive data in the boxes:  }\PYG{l+s+s2}{\PYGZdq{}}\PYG{p}{)}

\PYG{k}{def} \PYG{n+nf}{SA}\PYG{p}{(}\PYG{n}{mu}\PYG{p}{,} \PYG{n}{m1}\PYG{p}{,} \PYG{n}{m2}\PYG{p}{,} \PYG{n}{m3}\PYG{p}{,} \PYG{n}{m4}\PYG{p}{,} \PYG{n}{ml}\PYG{p}{)}\PYG{p}{:}
    \PYG{n}{dia} \PYG{o}{=} \PYG{p}{[}\PYG{l+m+mi}{6}\PYG{p}{,}\PYG{l+m+mi}{2}\PYG{p}{,}\PYG{l+m+mf}{0.6}\PYG{p}{,}\PYG{l+m+mf}{0.2}\PYG{p}{,} \PYG{l+m+mf}{0.06}\PYG{p}{,} \PYG{l+m+mf}{0.01}\PYG{p}{]} \PYG{c+c1}{\PYGZsh{} mm, diameter \PYGZlt{}0.06 (cup)= 0.01, \PYGZgt{}2 = 6}
    \PYG{n}{mass} \PYG{o}{=} \PYG{p}{[}\PYG{n}{mu}\PYG{p}{,} \PYG{n}{m1}\PYG{p}{,} \PYG{n}{m2}\PYG{p}{,} \PYG{n}{m3}\PYG{p}{,} \PYG{n}{m4}\PYG{p}{,} \PYG{n}{ml}\PYG{p}{]} \PYG{c+c1}{\PYGZsh{} g, the residue in seive }
    \PYG{n}{Total\PYGZus{}mass} \PYG{o}{=} \PYG{n}{np}\PYG{o}{.}\PYG{n}{sum}\PYG{p}{(}\PYG{n}{mass}\PYG{p}{)}  \PYG{c+c1}{\PYGZsh{} add the mass column to get total mass}
    \PYG{n}{retain\PYGZus{}per} \PYG{o}{=} \PYG{n}{np}\PYG{o}{.}\PYG{n}{round}\PYG{p}{(}\PYG{n}{mass}\PYG{o}{/}\PYG{n}{Total\PYGZus{}mass}\PYG{o}{*}\PYG{l+m+mi}{100}\PYG{p}{,}\PYG{l+m+mi}{3}\PYG{p}{)}   \PYG{c+c1}{\PYGZsh{} retain percentage}
    \PYG{n}{retain\PYGZus{}per\PYGZus{}cumsum} \PYG{o}{=} \PYG{n}{np}\PYG{o}{.}\PYG{n}{round}\PYG{p}{(}\PYG{n}{np}\PYG{o}{.}\PYG{n}{cumsum}\PYG{p}{(}\PYG{n}{retain\PYGZus{}per}\PYG{p}{)}\PYG{p}{,}\PYG{l+m+mi}{3}\PYG{p}{)} \PYG{c+c1}{\PYGZsh{} get the cummulative sum of the reatined}
    \PYG{n}{passing\PYGZus{}per} \PYG{o}{=} \PYG{n}{np}\PYG{o}{.}\PYG{n}{round}\PYG{p}{(}\PYG{l+m+mi}{100} \PYG{o}{\PYGZhy{}} \PYG{n}{retain\PYGZus{}per\PYGZus{}cumsum}\PYG{p}{,} \PYG{l+m+mi}{3}\PYG{p}{)} \PYG{c+c1}{\PYGZsh{} substract 100\PYGZhy{}cummsum to get passing \PYGZpc{}}
    \PYG{n}{data} \PYG{o}{=} \PYG{p}{\PYGZob{}}\PYG{l+s+s2}{\PYGZdq{}}\PYG{l+s+s2}{mesh diameter [mm]}\PYG{l+s+s2}{\PYGZdq{}}\PYG{p}{:} \PYG{n}{dia}\PYG{p}{,} \PYG{l+s+s2}{\PYGZdq{}}\PYG{l+s+s2}{residue in the sieve [g]}\PYG{l+s+s2}{\PYGZdq{}}\PYG{p}{:} \PYG{n}{mass}\PYG{p}{,} \PYG{l+s+s2}{\PYGZdq{}}\PYG{l+s+s2}{Σtotal}\PYG{l+s+s2}{\PYGZdq{}}\PYG{p}{:} \PYG{n}{retain\PYGZus{}per}\PYG{p}{,} \PYG{l+s+s2}{\PYGZdq{}}\PYG{l+s+s2}{Σ/Σtotal}\PYG{l+s+s2}{\PYGZdq{}}\PYG{p}{:} \PYG{n}{passing\PYGZus{}per} \PYG{p}{\PYGZcb{}}

    \PYG{n}{df1}\PYG{o}{=} \PYG{n}{pd}\PYG{o}{.}\PYG{n}{DataFrame}\PYG{p}{(}\PYG{n}{data}\PYG{p}{)}
    \PYG{n}{df1} \PYG{o}{=} \PYG{n}{df1}\PYG{o}{.}\PYG{n}{set\PYGZus{}index}\PYG{p}{(}\PYG{l+s+s2}{\PYGZdq{}}\PYG{l+s+s2}{mesh diameter [mm]}\PYG{l+s+s2}{\PYGZdq{}}\PYG{p}{)}
    \PYG{n+nb}{print}\PYG{p}{(}\PYG{n}{df1}\PYG{p}{)}

    \PYG{n}{plt}\PYG{o}{.}\PYG{n}{rcParams}\PYG{p}{[}\PYG{l+s+s1}{\PYGZsq{}}\PYG{l+s+s1}{axes.linewidth}\PYG{l+s+s1}{\PYGZsq{}}\PYG{p}{]}\PYG{o}{=}\PYG{l+m+mi}{2}
    \PYG{c+c1}{\PYGZsh{}plt.rcParams[\PYGZdq{}axes.edgecolor\PYGZdq{}]=\PYGZsq{}white\PYGZsq{}}
    \PYG{n}{plt}\PYG{o}{.}\PYG{n}{rcParams}\PYG{p}{[}\PYG{l+s+s1}{\PYGZsq{}}\PYG{l+s+s1}{grid.linestyle}\PYG{l+s+s1}{\PYGZsq{}}\PYG{p}{]}\PYG{o}{=}\PYG{l+s+s1}{\PYGZsq{}}\PYG{l+s+s1}{\PYGZhy{}\PYGZhy{}}\PYG{l+s+s1}{\PYGZsq{}}
    \PYG{n}{plt}\PYG{o}{.}\PYG{n}{rcParams}\PYG{p}{[}\PYG{l+s+s1}{\PYGZsq{}}\PYG{l+s+s1}{grid.linewidth}\PYG{l+s+s1}{\PYGZsq{}}\PYG{p}{]}\PYG{o}{=}\PYG{l+m+mi}{1}
    \PYG{n}{x} \PYG{o}{=} \PYG{n}{np}\PYG{o}{.}\PYG{n}{append}\PYG{p}{(}\PYG{p}{[}\PYG{l+m+mi}{20}\PYG{p}{]}\PYG{p}{,}\PYG{n}{dia}\PYG{p}{)} \PYG{c+c1}{\PYGZsh{} adding data to extend over 6 mm dia}
    \PYG{n}{y} \PYG{o}{=} \PYG{n}{np}\PYG{o}{.}\PYG{n}{append}\PYG{p}{(}\PYG{p}{[}\PYG{l+m+mi}{100}\PYG{p}{]}\PYG{p}{,}\PYG{n}{passing\PYGZus{}per}\PYG{p}{)} \PYG{c+c1}{\PYGZsh{} adding 100\PYGZpc{} to plot}

    \PYG{n}{fig}\PYG{p}{,} \PYG{n}{ax} \PYG{o}{=} \PYG{n}{plt}\PYG{o}{.}\PYG{n}{subplots}\PYG{p}{(}\PYG{n}{figsize}\PYG{o}{=}\PYG{p}{(}\PYG{l+m+mi}{6}\PYG{p}{,}\PYG{l+m+mi}{4}\PYG{p}{)}\PYG{p}{)}
    \PYG{n}{fig}\PYG{o}{.}\PYG{n}{canvas}\PYG{o}{.}\PYG{n}{header\PYGZus{}visible} \PYG{o}{=} \PYG{k+kc}{False}
    \PYG{n}{plt}\PYG{o}{.}\PYG{n}{semilogx}\PYG{p}{(}\PYG{n}{x}\PYG{p}{,} \PYG{n}{y}\PYG{p}{,} \PYG{l+s+s1}{\PYGZsq{}}\PYG{l+s+s1}{x\PYGZhy{}}\PYG{l+s+s1}{\PYGZsq{}}\PYG{p}{,} \PYG{n}{color}\PYG{o}{=}\PYG{l+s+s1}{\PYGZsq{}}\PYG{l+s+s1}{red}\PYG{l+s+s1}{\PYGZsq{}}\PYG{p}{)}  
    \PYG{n}{tics}\PYG{o}{=}\PYG{n}{x}\PYG{o}{.}\PYG{n}{tolist}\PYG{p}{(}\PYG{p}{)}

    \PYG{n}{ax}\PYG{o}{.}\PYG{n}{grid}\PYG{p}{(}\PYG{n}{which}\PYG{o}{=}\PYG{l+s+s1}{\PYGZsq{}}\PYG{l+s+s1}{major}\PYG{l+s+s1}{\PYGZsq{}}\PYG{p}{,} \PYG{n}{color}\PYG{o}{=}\PYG{l+s+s1}{\PYGZsq{}}\PYG{l+s+s1}{k}\PYG{l+s+s1}{\PYGZsq{}}\PYG{p}{,} \PYG{n}{alpha}\PYG{o}{=}\PYG{l+m+mf}{0.7}\PYG{p}{)} 
    \PYG{n}{ax}\PYG{o}{.}\PYG{n}{grid}\PYG{p}{(}\PYG{n}{which}\PYG{o}{=}\PYG{l+s+s1}{\PYGZsq{}}\PYG{l+s+s1}{minor}\PYG{l+s+s1}{\PYGZsq{}}\PYG{p}{,} \PYG{n}{color}\PYG{o}{=}\PYG{l+s+s1}{\PYGZsq{}}\PYG{l+s+s1}{k}\PYG{l+s+s1}{\PYGZsq{}}\PYG{p}{,} \PYG{n}{alpha}\PYG{o}{=}\PYG{l+m+mf}{0.3}\PYG{p}{)}
    \PYG{n}{ax}\PYG{o}{.}\PYG{n}{set\PYGZus{}xticks}\PYG{p}{(}\PYG{n}{x}\PYG{p}{)}\PYG{p}{;}  
    \PYG{n}{ax}\PYG{o}{.}\PYG{n}{set\PYGZus{}yticks}\PYG{p}{(}\PYG{n}{np}\PYG{o}{.}\PYG{n}{arange}\PYG{p}{(}\PYG{l+m+mi}{0}\PYG{p}{,}\PYG{l+m+mi}{110}\PYG{p}{,}\PYG{l+m+mi}{10}\PYG{p}{)}\PYG{p}{)}\PYG{p}{;}
    \PYG{n}{plt}\PYG{o}{.}\PYG{n}{title}\PYG{p}{(}\PYG{l+s+s1}{\PYGZsq{}}\PYG{l+s+s1}{grain size distribution}\PYG{l+s+s1}{\PYGZsq{}}\PYG{p}{)}\PYG{p}{;}
    \PYG{n}{plt}\PYG{o}{.}\PYG{n}{xlabel}\PYG{p}{(}\PYG{l+s+s1}{\PYGZsq{}}\PYG{l+s+s1}{grain size d [mm]}\PYG{l+s+s1}{\PYGZsq{}}\PYG{p}{)}\PYG{p}{;}
    \PYG{n}{plt}\PYG{o}{.}\PYG{n}{ylabel}\PYG{p}{(}\PYG{l+s+s1}{\PYGZsq{}}\PYG{l+s+s1}{grain fraction \PYGZlt{} d ins }\PYG{l+s+si}{\PYGZpc{} o}\PYG{l+s+s1}{f total mass}\PYG{l+s+s1}{\PYGZsq{}}\PYG{p}{)}\PYG{p}{;}
    \PYG{n}{ax}\PYG{o}{.}\PYG{n}{set\PYGZus{}xlim}\PYG{p}{(}\PYG{l+m+mi}{0}\PYG{p}{,} \PYG{l+m+mi}{30}\PYG{p}{)}
    \PYG{k+kn}{from} \PYG{n+nn}{matplotlib}\PYG{n+nn}{.}\PYG{n+nn}{ticker} \PYG{k+kn}{import} \PYG{n}{StrMethodFormatter}
    \PYG{n}{ax}\PYG{o}{.}\PYG{n}{xaxis}\PYG{o}{.}\PYG{n}{set\PYGZus{}major\PYGZus{}formatter}\PYG{p}{(}\PYG{n}{StrMethodFormatter}\PYG{p}{(}\PYG{l+s+s1}{\PYGZsq{}}\PYG{l+s+si}{\PYGZob{}x:0.2f\PYGZcb{}}\PYG{l+s+s1}{\PYGZsq{}}\PYG{p}{)}\PYG{p}{)}


\PYG{n}{style} \PYG{o}{=} \PYG{p}{\PYGZob{}}\PYG{l+s+s1}{\PYGZsq{}}\PYG{l+s+s1}{description\PYGZus{}width}\PYG{l+s+s1}{\PYGZsq{}}\PYG{p}{:} \PYG{l+s+s1}{\PYGZsq{}}\PYG{l+s+s1}{200px}\PYG{l+s+s1}{\PYGZsq{}}\PYG{p}{\PYGZcb{}}    

\PYG{n}{Inter}\PYG{o}{=}\PYG{n}{widgets}\PYG{o}{.}\PYG{n}{interact\PYGZus{}manual}\PYG{p}{(}\PYG{n}{SA}\PYG{p}{,} 
                       \PYG{n}{mu}\PYG{o}{=} \PYG{n}{widgets}\PYG{o}{.}\PYG{n}{FloatText}\PYG{p}{(}\PYG{n}{description}\PYG{o}{=}\PYG{l+s+s2}{\PYGZdq{}}\PYG{l+s+s2}{6 mm}\PYG{l+s+s2}{\PYGZdq{}}\PYG{p}{,} \PYG{n}{style}\PYG{o}{=}\PYG{n}{style}\PYG{p}{)}\PYG{p}{,}
                       \PYG{n}{m1}\PYG{o}{=} \PYG{n}{widgets}\PYG{o}{.}\PYG{n}{FloatText}\PYG{p}{(}\PYG{n}{description}\PYG{o}{=}\PYG{l+s+s2}{\PYGZdq{}}\PYG{l+s+s2}{2 mm}\PYG{l+s+s2}{\PYGZdq{}}\PYG{p}{,}\PYG{n}{style}\PYG{o}{=}\PYG{n}{style}\PYG{p}{)}\PYG{p}{,}
                       \PYG{n}{m2}\PYG{o}{=} \PYG{n}{widgets}\PYG{o}{.}\PYG{n}{FloatText}\PYG{p}{(}\PYG{n}{description}\PYG{o}{=}\PYG{l+s+s2}{\PYGZdq{}}\PYG{l+s+s2}{0.6 mm}\PYG{l+s+s2}{\PYGZdq{}}\PYG{p}{,} \PYG{n}{style}\PYG{o}{=}\PYG{n}{style}\PYG{p}{)}\PYG{p}{,}
                       \PYG{n}{m3}\PYG{o}{=} \PYG{n}{widgets}\PYG{o}{.}\PYG{n}{FloatText}\PYG{p}{(}\PYG{n}{description}\PYG{o}{=}\PYG{l+s+s2}{\PYGZdq{}}\PYG{l+s+s2}{0.2 mm}\PYG{l+s+s2}{\PYGZdq{}}\PYG{p}{,} \PYG{n}{style}\PYG{o}{=}\PYG{n}{style}\PYG{p}{)}\PYG{p}{,}
                       \PYG{n}{m4}\PYG{o}{=} \PYG{n}{widgets}\PYG{o}{.}\PYG{n}{FloatText}\PYG{p}{(}\PYG{n}{description}\PYG{o}{=}\PYG{l+s+s2}{\PYGZdq{}}\PYG{l+s+s2}{0.06 mm}\PYG{l+s+s2}{\PYGZdq{}}\PYG{p}{,} \PYG{n}{style}\PYG{o}{=}\PYG{n}{style}\PYG{p}{)}\PYG{p}{,}
                       \PYG{n}{ml}\PYG{o}{=} \PYG{n}{widgets}\PYG{o}{.}\PYG{n}{FloatText}\PYG{p}{(}\PYG{n}{description}\PYG{o}{=}\PYG{l+s+s2}{\PYGZdq{}}\PYG{l+s+s2}{0.01 mm}\PYG{l+s+s2}{\PYGZdq{}}\PYG{p}{,} \PYG{n}{style}\PYG{o}{=}\PYG{n}{style}\PYG{p}{)}\PYG{p}{)}
\end{sphinxVerbatim}

\begin{sphinxVerbatim}[commandchars=\\\{\}]
Please provide the seive data in the boxes:  
\end{sphinxVerbatim}

\begin{sphinxVerbatim}[commandchars=\\\{\}]
interactive(children=(FloatText(value=0.0, description=\PYGZsq{}6 mm\PYGZsq{}, style=DescriptionStyle(description\PYGZus{}width=\PYGZsq{}200px…
\end{sphinxVerbatim}

\begin{sphinxVerbatim}[commandchars=\\\{\}]
\PYG{k}{def} \PYG{n+nf}{SA2}\PYG{p}{(}\PYG{n}{d10}\PYG{p}{,} \PYG{n}{d60}\PYG{p}{,} \PYG{n}{t}\PYG{p}{)}\PYG{p}{:}
    \PYG{n}{U} \PYG{o}{=} \PYG{n}{d60}\PYG{o}{/}\PYG{n}{d10}
    \PYG{n}{K\PYGZus{}h} \PYG{o}{=}  \PYG{l+m+mf}{0.0116}\PYG{o}{*}\PYG{p}{(}\PYG{l+m+mf}{0.7}\PYG{o}{+}\PYG{l+m+mf}{0.03}\PYG{o}{*}\PYG{n}{t}\PYG{p}{)}\PYG{o}{*}\PYG{n}{d10}\PYG{o}{*}\PYG{o}{*}\PYG{l+m+mi}{2}
    \PYG{n+nb}{print}\PYG{p}{(}\PYG{l+s+s2}{\PYGZdq{}}\PYG{l+s+se}{\PYGZbs{}n}\PYG{l+s+s2}{ The coefficient of non\PYGZhy{}uniformity: }\PYG{l+s+si}{\PYGZob{}0:0.2f\PYGZcb{}}\PYG{l+s+s2}{\PYGZdq{}}\PYG{o}{.}\PYG{n}{format}\PYG{p}{(}\PYG{n}{U}\PYG{p}{)}\PYG{p}{,} \PYG{l+s+s2}{\PYGZdq{}}\PYG{l+s+se}{\PYGZbs{}n}\PYG{l+s+s2}{\PYGZdq{}}\PYG{p}{)}
    \PYG{n+nb}{print}\PYG{p}{(}\PYG{l+s+s2}{\PYGZdq{}}\PYG{l+s+s2}{The Hydraulic Conductivity based on Hazen Formula: }\PYG{l+s+si}{\PYGZob{}0:0.2e\PYGZcb{}}\PYG{l+s+s2}{ m/s}\PYG{l+s+s2}{\PYGZdq{}}\PYG{o}{.}\PYG{n}{format}\PYG{p}{(}\PYG{n}{K\PYGZus{}h}\PYG{p}{)}\PYG{p}{)}

\PYG{n}{style} \PYG{o}{=} \PYG{p}{\PYGZob{}}\PYG{l+s+s1}{\PYGZsq{}}\PYG{l+s+s1}{description\PYGZus{}width}\PYG{l+s+s1}{\PYGZsq{}}\PYG{p}{:} \PYG{l+s+s1}{\PYGZsq{}}\PYG{l+s+s1}{200px}\PYG{l+s+s1}{\PYGZsq{}}\PYG{p}{\PYGZcb{}}    

\PYG{n}{Inter}\PYG{o}{=}\PYG{n}{widgets}\PYG{o}{.}\PYG{n}{interact\PYGZus{}manual}\PYG{p}{(}\PYG{n}{SA2}\PYG{p}{,} 
                       \PYG{n}{d10}\PYG{o}{=} \PYG{n}{widgets}\PYG{o}{.}\PYG{n}{FloatText}\PYG{p}{(}\PYG{n}{description}\PYG{o}{=}\PYG{l+s+s2}{\PYGZdq{}}\PYG{l+s+s2}{d10 (mm)}\PYG{l+s+s2}{\PYGZdq{}}\PYG{p}{,} \PYG{n}{style}\PYG{o}{=}\PYG{n}{style}\PYG{p}{)}\PYG{p}{,}
                       \PYG{n}{d60}\PYG{o}{=} \PYG{n}{widgets}\PYG{o}{.}\PYG{n}{FloatText}\PYG{p}{(}\PYG{n}{description}\PYG{o}{=}\PYG{l+s+s2}{\PYGZdq{}}\PYG{l+s+s2}{d60 (mm)}\PYG{l+s+s2}{\PYGZdq{}}\PYG{p}{,}\PYG{n}{style}\PYG{o}{=}\PYG{n}{style}\PYG{p}{)}\PYG{p}{,}
                       \PYG{n}{t}\PYG{o}{=} \PYG{n}{widgets}\PYG{o}{.}\PYG{n}{FloatText}\PYG{p}{(}\PYG{n}{description}\PYG{o}{=}\PYG{l+s+s2}{\PYGZdq{}}\PYG{l+s+s2}{Temperature (°C)}\PYG{l+s+s2}{\PYGZdq{}}\PYG{p}{,} \PYG{n}{style}\PYG{o}{=}\PYG{n}{style}\PYG{p}{)}\PYG{p}{)}
\end{sphinxVerbatim}

\begin{sphinxVerbatim}[commandchars=\\\{\}]
interactive(children=(FloatText(value=0.0, description=\PYGZsq{}d10 (mm)\PYGZsq{}, style=DescriptionStyle(description\PYGZus{}width=\PYGZsq{}2…
\end{sphinxVerbatim}

\begin{sphinxVerbatim}[commandchars=\\\{\}]
\PYG{k+kn}{import} \PYG{n+nn}{numpy} \PYG{k}{as} \PYG{n+nn}{np}
\PYG{k+kn}{import} \PYG{n+nn}{pandas} \PYG{k}{as} \PYG{n+nn}{pd} 
\PYG{k+kn}{import} \PYG{n+nn}{matplotlib}\PYG{n+nn}{.}\PYG{n+nn}{pyplot} \PYG{k}{as} \PYG{n+nn}{plt}
\PYG{k+kn}{from} \PYG{n+nn}{xlrd} \PYG{k+kn}{import} \PYG{o}{*}
\PYG{k+kn}{import} \PYG{n+nn}{ipysheet} \PYG{k}{as} \PYG{n+nn}{ips}
\PYG{k+kn}{import} \PYG{n+nn}{panel} \PYG{k}{as} \PYG{n+nn}{pn}
\PYG{o}{\PYGZpc{}}\PYG{k}{matplotlib} inline 
\PYG{k+kn}{from} \PYG{n+nn}{scipy} \PYG{k+kn}{import} \PYG{n}{stats} 
\PYG{n}{pn}\PYG{o}{.}\PYG{n}{extension}\PYG{p}{(}\PYG{l+s+s1}{\PYGZsq{}}\PYG{l+s+s1}{katex}\PYG{l+s+s1}{\PYGZsq{}}\PYG{p}{)} 
\end{sphinxVerbatim}


\section{Groundwater Exam Solution \sphinxhyphen{}  2019\sphinxhyphen{}2020}
\label{\detokenize{contents/questions/GW_exam_2019_20:groundwater-exam-solution-2019-2020}}\label{\detokenize{contents/questions/GW_exam_2019_20::doc}}
\sphinxstylestrong{Q1. Aquifer Types}   (ca. 5 pts.)

a. Differentiate between Aquifer, Aquitard and Aquiclude (3 points)

b. Schematically present a confined aquifer (vertical cross\sphinxhyphen{}section) providing essential features with their legends (2 points)

\sphinxstylestrong{Solution 1. a.}

See slide: L03/08

An \sphinxstylestrong{aquifer} or a groundwater reservoir can store and transmit significant (= exploitable) amounts of groundwater.

An \sphinxstylestrong{aquitard} can store and transmit groundwater but to a much lesser extent than an (adjacent) aquifer.

An \sphinxstylestrong{aquiclude} can store groundwater but cannot transmit groundwater.

\sphinxstylestrong{Solution 1b} \sphinxhyphen{} (L03/11)

 

\noindent\sphinxincludegraphics{{contents/questions/figs/Q1b_2019-20}.png}

\sphinxstylestrong{Confined Aquifer}
\begin{enumerate}
\sphinxsetlistlabels{\arabic}{enumi}{enumii}{}{.}%
\item {} 
The essential feature of confined aquifer is provided in the figure above.

\end{enumerate}

\sphinxstylestrong{Q2. Groundwater storage (3 pts.)}

Dry season in Dresden (2018\sphinxhyphen{}19) led to intense extraction of groundawater in rural areas. At one location in a confined aquifer (storage coeff. 4·10\sphinxhyphen{}4, total porosity = 30\%),  the pressure head was lowered by 150 m. The thickness of the aquifer was measured to be 90 m before the beginning of extraction and the compressibility of the porous medium in that region is estimated 6·10\sphinxhyphen{}8 m2/N. Density of water can be assumed to be 1000 kg/m3.

(Hint: \(\Delta V_T = \alpha_{pm}\cdot\rho_w \cdot g \cdot  V_T \cdot  \Delta \psi \)).

a. Approximately how much water was extracted? (1 point)

b. How much land subsidence due to water extraction is expected? (2 point)

\sphinxstylestrong{Solution 2}

Given relation:

For part a. (see Tut 02/P4)

\(
S_s  = \frac{\Delta V_w}{V_T\cdot \Delta \psi}
\)

In confined aquifer \(S\) is used, which is obtained from:

\(S = S_s \cdot m\)

\(
 \frac{S}{m} = \frac{\Delta V_w}{A_T\cdot m \cdot \Delta \psi}
\)

So,

\(  \frac{\Delta V_w}{A}  = S \cdot \Delta \psi\)

For part b.

\(\Delta V_T = \alpha_{pm}\cdot\rho_w \cdot g \cdot  V_T \cdot  \Delta \psi \)

\(V_T = A\times h\), with \(A\) surface area and \(h\) aquifer thickness.

\(\Delta V_T = A \times \Delta h\), with \(\Delta h\) change in thickness.

\(A \times \Delta h= \alpha_{pm}\cdot\rho_w \cdot g \cdot  A \cdot h \cdot  \Delta \psi \)

\(\Delta h = \alpha_{pm}\cdot\rho_w \cdot g \cdot h \cdot  \Delta \psi\), with \(\Delta h\) being the land subsidence

\begin{sphinxVerbatim}[commandchars=\\\{\}]
\PYG{c+c1}{\PYGZsh{}  solution 2 a}

\PYG{c+c1}{\PYGZsh{} Given}

\PYG{n}{A} \PYG{o}{=} \PYG{l+m+mi}{1} \PYG{c+c1}{\PYGZsh{} m\(\sp{\text{2}}\), assuming 1 m\(\sp{\text{2}}\) aquifer area}
\PYG{n}{h} \PYG{o}{=} \PYG{l+m+mi}{90} \PYG{c+c1}{\PYGZsh{} m, aquifer height before extraction}
\PYG{n}{d\PYGZus{}psi} \PYG{o}{=} \PYG{l+m+mi}{150} \PYG{c+c1}{\PYGZsh{} m, change in pressure head}
\PYG{n}{S\PYGZus{}2} \PYG{o}{=} \PYG{l+m+mi}{4}\PYG{o}{*}\PYG{l+m+mi}{10}\PYG{o}{*}\PYG{o}{*}\PYG{o}{\PYGZhy{}}\PYG{l+m+mi}{4} \PYG{c+c1}{\PYGZsh{}  specific storage}
\PYG{n}{rho\PYGZus{}w} \PYG{o}{=} \PYG{l+m+mi}{1000} \PYG{c+c1}{\PYGZsh{} Kg/m\(\sp{\text{3}}\), density of water}
\PYG{n}{a\PYGZus{}pm} \PYG{o}{=} \PYG{l+m+mi}{6}\PYG{o}{*} \PYG{l+m+mi}{10}\PYG{o}{*}\PYG{o}{*}\PYG{o}{\PYGZhy{}}\PYG{l+m+mi}{8} \PYG{c+c1}{\PYGZsh{} m\(\sp{\text{2}}\)/N = m\PYGZhy{}s\(\sp{\text{2}}\)/kg, compressibility of porous medium}
\PYG{n}{g} \PYG{o}{=} \PYG{l+m+mf}{9.81} \PYG{c+c1}{\PYGZsh{} m/s\(\sp{\text{2}}\), gravity factor}

\PYG{c+c1}{\PYGZsh{}Solution}

\PYG{n}{d\PYGZus{}V\PYGZus{}w}  \PYG{o}{=} \PYG{n}{S\PYGZus{}2}\PYG{o}{*}\PYG{n}{A}\PYG{o}{*}\PYG{n}{d\PYGZus{}psi}

\PYG{n+nb}{print}\PYG{p}{(}\PYG{l+s+s2}{\PYGZdq{}}\PYG{l+s+s2}{The water abstraction volume per m}\PYG{l+s+se}{\PYGZbs{}u00b2}\PYG{l+s+s2}{ aquifer is }\PYG{l+s+si}{\PYGZob{}0:0.3f\PYGZcb{}}\PYG{l+s+s2}{\PYGZdq{}}\PYG{o}{.}\PYG{n}{format}\PYG{p}{(}\PYG{n}{d\PYGZus{}V\PYGZus{}w}\PYG{p}{)}\PYG{p}{,} \PYG{l+s+s2}{\PYGZdq{}}\PYG{l+s+s2}{m}\PYG{l+s+se}{\PYGZbs{}u00b3}\PYG{l+s+s2}{\PYGZdq{}}\PYG{p}{)}
\end{sphinxVerbatim}

\begin{sphinxVerbatim}[commandchars=\\\{\}]
The water abstraction volume per m\(\sp{\text{2}}\) aquifer is 0.060 m\(\sp{\text{3}}\)
\end{sphinxVerbatim}

\begin{sphinxVerbatim}[commandchars=\\\{\}]
\PYG{c+c1}{\PYGZsh{}  solution 2 b}


\PYG{c+c1}{\PYGZsh{} Given}

\PYG{n}{A} \PYG{o}{=} \PYG{l+m+mi}{1} \PYG{c+c1}{\PYGZsh{} m\(\sp{\text{2}}\), assuming 1 m\(\sp{\text{2}}\) aquifer area}
\PYG{n}{h} \PYG{o}{=} \PYG{l+m+mi}{90} \PYG{c+c1}{\PYGZsh{} m, aquifer height before extraction}
\PYG{n}{d\PYGZus{}psi} \PYG{o}{=} \PYG{l+m+mi}{150} \PYG{c+c1}{\PYGZsh{} m, change in pressure head}
\PYG{n}{Ss} \PYG{o}{=} \PYG{l+m+mi}{4}\PYG{o}{*}\PYG{l+m+mi}{10}\PYG{o}{*}\PYG{o}{*}\PYG{o}{\PYGZhy{}}\PYG{l+m+mi}{4} \PYG{c+c1}{\PYGZsh{} specific storage}
\PYG{n}{rho\PYGZus{}w} \PYG{o}{=} \PYG{l+m+mi}{1000} \PYG{c+c1}{\PYGZsh{} Kg/m\(\sp{\text{3}}\), density of water}
\PYG{n}{a\PYGZus{}pm} \PYG{o}{=} \PYG{l+m+mi}{6}\PYG{o}{*} \PYG{l+m+mi}{10}\PYG{o}{*}\PYG{o}{*}\PYG{o}{\PYGZhy{}}\PYG{l+m+mi}{8} \PYG{c+c1}{\PYGZsh{} m\(\sp{\text{2}}\)/N = m\PYGZhy{}s\(\sp{\text{2}}\)/kg, compressibility of porous medium}
\PYG{n}{g} \PYG{o}{=} \PYG{l+m+mf}{9.81} \PYG{c+c1}{\PYGZsh{} m/s\(\sp{\text{2}}\), gravity factor}

\PYG{c+c1}{\PYGZsh{}interim calculation}
\PYG{n}{V\PYGZus{}T} \PYG{o}{=} \PYG{n}{A}\PYG{o}{*}\PYG{n}{h} \PYG{c+c1}{\PYGZsh{} m\(\sp{\text{3}}\), Aquifer volume before extraction }

\PYG{c+c1}{\PYGZsh{}Solution}

\PYG{n}{d\PYGZus{}h}  \PYG{o}{=} \PYG{n}{a\PYGZus{}pm}\PYG{o}{*}\PYG{n}{rho\PYGZus{}w}\PYG{o}{*}\PYG{n}{g}\PYG{o}{*}\PYG{n}{h}\PYG{o}{*}\PYG{n}{d\PYGZus{}psi}

\PYG{n+nb}{print}\PYG{p}{(}\PYG{l+s+s2}{\PYGZdq{}}\PYG{l+s+s2}{The water abstraction volume is }\PYG{l+s+si}{\PYGZob{}0:0.2f\PYGZcb{}}\PYG{l+s+s2}{\PYGZdq{}}\PYG{o}{.}\PYG{n}{format}\PYG{p}{(}\PYG{n}{d\PYGZus{}h}\PYG{p}{)}\PYG{p}{,} \PYG{l+s+s2}{\PYGZdq{}}\PYG{l+s+s2}{m}\PYG{l+s+s2}{\PYGZdq{}}\PYG{p}{)}
\end{sphinxVerbatim}

\begin{sphinxVerbatim}[commandchars=\\\{\}]
The water abstraction volume is 7.95 m
\end{sphinxVerbatim}

\sphinxstylestrong{Q3. Aquifer Properties} (ca. 10 pts.)

The hydraulic conductivity of a sample (length 15 cm, diameter 5 cm) is to be determined using a constant\sphinxhyphen{}head permeameter. For that 250 ml water is passed through the sample in 30 s while maintaining the head difference of 2.5 cm. Properties of water provided are:
density of water at 20°C: 1000 kg/m3;                                                  dynamic viscosity of water at 20°C: 1.0087·10\sphinxhyphen{}3 Pa·s

a.  Sketch the problem as accurately as possible providing essential features with legends (3 points)

b. What will be the conductivity of the sample? (4 points)

What is the intrinsic permeability of the sample? (2 points)

c. What soil type is likely the sample? (1 point)

(Hint: For the calculation of permeability, dynamic viscosity/density ratio is required.)

\sphinxstylestrong{Solution 3} \sphinxhyphen{}

\sphinxstylestrong{Solution 3a} (L05/15)

 

\noindent\sphinxincludegraphics{{contents/questions/figs/Q3a_2019-20}.png}

\begin{sphinxVerbatim}[commandchars=\\\{\}]
\PYG{c+c1}{\PYGZsh{}Solution 3b** (L05/15)}

\PYG{c+c1}{\PYGZsh{} Given}

\PYG{n}{L\PYGZus{}c} \PYG{o}{=} \PYG{l+m+mi}{15} \PYG{c+c1}{\PYGZsh{} cm, column length}
\PYG{n}{Dia\PYGZus{}c} \PYG{o}{=} \PYG{l+m+mi}{5} \PYG{c+c1}{\PYGZsh{} cm, diameter column}
\PYG{n}{V\PYGZus{}in}\PYG{o}{=} \PYG{l+m+mi}{250} \PYG{c+c1}{\PYGZsh{} mL, water entering the column}
\PYG{n}{t\PYGZus{}c} \PYG{o}{=} \PYG{l+m+mi}{30} \PYG{c+c1}{\PYGZsh{} s, time required to pass}
\PYG{n}{d\PYGZus{}3h} \PYG{o}{=} \PYG{l+m+mf}{2.5} \PYG{c+c1}{\PYGZsh{} cm, head difference}

\PYG{c+c1}{\PYGZsh{} interim calculation}
\PYG{n}{A\PYGZus{}c} \PYG{o}{=} \PYG{n}{np}\PYG{o}{.}\PYG{n}{pi}\PYG{o}{*}\PYG{n}{Dia\PYGZus{}c}\PYG{o}{*}\PYG{o}{*}\PYG{l+m+mi}{2}\PYG{o}{/}\PYG{l+m+mi}{4} \PYG{c+c1}{\PYGZsh{} cm\(\sp{\text{2}}\), Area of column}
\PYG{n}{Q\PYGZus{}c} \PYG{o}{=} \PYG{n}{V\PYGZus{}in}\PYG{o}{/}\PYG{n}{t\PYGZus{}c} \PYG{c+c1}{\PYGZsh{} cm\(\sp{\text{3}}\)/s, assume 1mL = 1 cm\(\sp{\text{3}}\), Discharge out of column}

\PYG{c+c1}{\PYGZsh{}solution }
\PYG{n}{K\PYGZus{}c} \PYG{o}{=} \PYG{p}{(}\PYG{n}{Q\PYGZus{}c}\PYG{o}{*}\PYG{n}{L\PYGZus{}c}\PYG{p}{)}\PYG{o}{/}\PYG{p}{(}\PYG{n}{A\PYGZus{}c}\PYG{o}{*}\PYG{n}{d\PYGZus{}3h}\PYG{p}{)} \PYG{c+c1}{\PYGZsh{} cm/s, conductivity }


\PYG{n+nb}{print}\PYG{p}{(}\PYG{l+s+s2}{\PYGZdq{}}\PYG{l+s+s2}{The area of the column is }\PYG{l+s+si}{\PYGZob{}0:0.2f\PYGZcb{}}\PYG{l+s+s2}{\PYGZdq{}}\PYG{o}{.}\PYG{n}{format}\PYG{p}{(}\PYG{n}{A\PYGZus{}c}\PYG{p}{)}\PYG{p}{,} \PYG{l+s+s2}{\PYGZdq{}}\PYG{l+s+s2}{cm}\PYG{l+s+se}{\PYGZbs{}u00b2}\PYG{l+s+s2}{\PYGZdq{}}\PYG{p}{)}
\PYG{n+nb}{print}\PYG{p}{(}\PYG{l+s+s2}{\PYGZdq{}}\PYG{l+s+s2}{The discharge from the aquifer is }\PYG{l+s+si}{\PYGZob{}0:0.2f\PYGZcb{}}\PYG{l+s+s2}{\PYGZdq{}}\PYG{o}{.}\PYG{n}{format}\PYG{p}{(}\PYG{n}{Q\PYGZus{}c}\PYG{p}{)}\PYG{p}{,} \PYG{l+s+s2}{\PYGZdq{}}\PYG{l+s+s2}{cm}\PYG{l+s+se}{\PYGZbs{}u00b3}\PYG{l+s+s2}{/s}\PYG{l+s+s2}{\PYGZdq{}}\PYG{p}{)}
\PYG{n+nb}{print}\PYG{p}{(}\PYG{l+s+s2}{\PYGZdq{}}\PYG{l+s+s2}{The conductivity of the sample is }\PYG{l+s+si}{\PYGZob{}0:0.2f\PYGZcb{}}\PYG{l+s+s2}{\PYGZdq{}}\PYG{o}{.}\PYG{n}{format}\PYG{p}{(}\PYG{n}{K\PYGZus{}c}\PYG{p}{)}\PYG{p}{,} \PYG{l+s+s2}{\PYGZdq{}}\PYG{l+s+s2}{cm/s}\PYG{l+s+s2}{\PYGZdq{}}\PYG{p}{)}
\PYG{n+nb}{print}\PYG{p}{(}\PYG{l+s+s2}{\PYGZdq{}}\PYG{l+s+s2}{The conductivity of the sample is }\PYG{l+s+si}{\PYGZob{}0:0.4f\PYGZcb{}}\PYG{l+s+s2}{\PYGZdq{}}\PYG{o}{.}\PYG{n}{format}\PYG{p}{(}\PYG{n}{K\PYGZus{}c}\PYG{o}{/}\PYG{l+m+mi}{100}\PYG{p}{)}\PYG{p}{,} \PYG{l+s+s2}{\PYGZdq{}}\PYG{l+s+s2}{m/s}\PYG{l+s+s2}{\PYGZdq{}}\PYG{p}{)}
\end{sphinxVerbatim}

\begin{sphinxVerbatim}[commandchars=\\\{\}]
The area of the column is 19.63 cm\(\sp{\text{2}}\)
The discharge from the aquifer is 8.33 cm\(\sp{\text{3}}\)/s
The conductivity of the sample is 2.55 cm/s
The conductivity of the sample is 0.0255 m/s
\end{sphinxVerbatim}

\begin{sphinxVerbatim}[commandchars=\\\{\}]
\PYG{c+c1}{\PYGZsh{}Solution 3c** (L05/18)}

\PYG{c+c1}{\PYGZsh{} Given}

\PYG{n}{K\PYGZus{}m} \PYG{o}{=} \PYG{n}{K\PYGZus{}c}\PYG{o}{/}\PYG{l+m+mi}{100} \PYG{c+c1}{\PYGZsh{} m/s, conductivity}
\PYG{n}{rho\PYGZus{}3w} \PYG{o}{=} \PYG{l+m+mi}{1000} \PYG{c+c1}{\PYGZsh{} Kg/m\(\sp{\text{3}}\), density of water}
\PYG{n}{nu\PYGZus{}w} \PYG{o}{=} \PYG{l+m+mf}{1.0087}\PYG{o}{*}\PYG{l+m+mi}{10}\PYG{o}{*}\PYG{o}{*}\PYG{o}{\PYGZhy{}}\PYG{l+m+mi}{3} \PYG{c+c1}{\PYGZsh{} Pa\PYGZhy{}s = Kg/m\PYGZhy{}s, dynamic viscosity }
\PYG{n}{g} \PYG{o}{=} \PYG{l+m+mf}{9.81} \PYG{c+c1}{\PYGZsh{} m/s\(\sp{\text{2}}\), gravity factor}

\PYG{c+c1}{\PYGZsh{}solution}
\PYG{n}{k\PYGZus{}c} \PYG{o}{=} \PYG{n}{K\PYGZus{}m}\PYG{o}{*}\PYG{n}{nu\PYGZus{}w}\PYG{o}{/}\PYG{p}{(}\PYG{n}{rho\PYGZus{}3w}\PYG{o}{*}\PYG{n}{g}\PYG{p}{)}

\PYG{n+nb}{print}\PYG{p}{(}\PYG{l+s+s2}{\PYGZdq{}}\PYG{l+s+s2}{The permeability of the sample is }\PYG{l+s+si}{\PYGZob{}0:0.10f\PYGZcb{}}\PYG{l+s+s2}{\PYGZdq{}}\PYG{o}{.}\PYG{n}{format}\PYG{p}{(}\PYG{n}{k\PYGZus{}c}\PYG{p}{)}\PYG{p}{,} \PYG{l+s+s2}{\PYGZdq{}}\PYG{l+s+s2}{m}\PYG{l+s+se}{\PYGZbs{}u00b2}\PYG{l+s+s2}{\PYGZdq{}}\PYG{p}{)}
\PYG{n+nb}{print}\PYG{p}{(}\PYG{l+s+s2}{\PYGZdq{}}\PYG{l+s+s2}{The permeability of the sample is }\PYG{l+s+si}{\PYGZob{}0:0.2E\PYGZcb{}}\PYG{l+s+s2}{\PYGZdq{}}\PYG{o}{.}\PYG{n}{format}\PYG{p}{(}\PYG{n}{k\PYGZus{}c}\PYG{p}{)}\PYG{p}{,} \PYG{l+s+s2}{\PYGZdq{}}\PYG{l+s+s2}{m}\PYG{l+s+se}{\PYGZbs{}u00b2}\PYG{l+s+s2}{\PYGZdq{}}\PYG{p}{)}
\end{sphinxVerbatim}

\begin{sphinxVerbatim}[commandchars=\\\{\}]
The permeability of the sample is 0.0000000026 m\(\sp{\text{2}}\)
The permeability of the sample is 2.62E\PYGZhy{}09 m\(\sp{\text{2}}\)
\end{sphinxVerbatim}

\sphinxstylestrong{Solution 3d}
(L05/11)

The sample in the column is likely gravel or coarse sand.

\sphinxstylestrong{Q4. Sieve Analysis (ca. 6 pts.)}

Sieve experiments were performed with the bore samples and the following observations were obtained:


\begin{savenotes}\sphinxattablestart
\centering
\begin{tabulary}{\linewidth}[t]{|T|T|T|T|}
\hline
\sphinxstyletheadfamily 
mesh diameter {[}mm{]}
&\sphinxstyletheadfamily 
residue in the sieve {[}g{]}
&\sphinxstyletheadfamily 
Σ total
&\sphinxstyletheadfamily 
Σ/Σtotal
\\
\hline
6
&
0
&

&

\\
\hline
2
&
40
&

&

\\
\hline
0.6
&
250
&

&

\\
\hline
0.2
&
150
&

&

\\
\hline
0.06
&
60
&

&

\\
\hline
\textless{}0.06 (cup)
&
10
&

&

\\
\hline
\end{tabulary}
\par
\sphinxattableend\end{savenotes}

a. Draw the granulometric curve in the diagram below.			(ca. 5 pts.)

b. Briefly characterise the sediment.						(ca. 1 pt.)

\begin{sphinxVerbatim}[commandchars=\\\{\}]
\PYG{c+c1}{\PYGZsh{}Solution 4 \PYGZhy{}(L03/18)}

\PYG{n}{dia} \PYG{o}{=} \PYG{p}{[}\PYG{l+m+mi}{6}\PYG{p}{,}\PYG{l+m+mi}{2}\PYG{p}{,}\PYG{l+m+mf}{0.6}\PYG{p}{,}\PYG{l+m+mf}{0.2}\PYG{p}{,} \PYG{l+m+mf}{0.06}\PYG{p}{,} \PYG{l+m+mf}{0.001}\PYG{p}{]} \PYG{c+c1}{\PYGZsh{} mm, diameter \PYGZlt{}0.06 (cup)= 0.001}
\PYG{n}{mass} \PYG{o}{=} \PYG{p}{[}\PYG{l+m+mi}{0}\PYG{p}{,} \PYG{l+m+mi}{40}\PYG{p}{,} \PYG{l+m+mi}{250}\PYG{p}{,} \PYG{l+m+mi}{150}\PYG{p}{,} \PYG{l+m+mi}{60}\PYG{p}{,} \PYG{l+m+mi}{10}\PYG{p}{]} \PYG{c+c1}{\PYGZsh{} g, the residue in seive }

\PYG{c+c1}{\PYGZsh{} Calculation steps \PYGZhy{} filling table}
\PYG{n}{Total\PYGZus{}mass} \PYG{o}{=} \PYG{n}{np}\PYG{o}{.}\PYG{n}{sum}\PYG{p}{(}\PYG{n}{mass}\PYG{p}{)}  \PYG{c+c1}{\PYGZsh{} add the mass column to get total mass}
\PYG{n}{retain\PYGZus{}per} \PYG{o}{=} \PYG{n}{mass}\PYG{o}{/}\PYG{n}{Total\PYGZus{}mass}\PYG{o}{*}\PYG{l+m+mi}{100}   \PYG{c+c1}{\PYGZsh{} retain percentage}
\PYG{n}{retain\PYGZus{}per\PYGZus{}cumsum} \PYG{o}{=} \PYG{n}{np}\PYG{o}{.}\PYG{n}{cumsum}\PYG{p}{(}\PYG{n}{retain\PYGZus{}per}\PYG{p}{)} \PYG{c+c1}{\PYGZsh{} get the cummulative sum of the reatined}
\PYG{n}{passing\PYGZus{}per} \PYG{o}{=} \PYG{l+m+mi}{100} \PYG{o}{\PYGZhy{}} \PYG{n}{retain\PYGZus{}per\PYGZus{}cumsum} \PYG{c+c1}{\PYGZsh{} substract 100\PYGZhy{}cummsum to get passing \PYGZpc{}}

\PYG{n}{data} \PYG{o}{=} \PYG{p}{\PYGZob{}}\PYG{l+s+s2}{\PYGZdq{}}\PYG{l+s+s2}{mesh diameter [mm]}\PYG{l+s+s2}{\PYGZdq{}}\PYG{p}{:} \PYG{n}{dia}\PYG{p}{,} \PYG{l+s+s2}{\PYGZdq{}}\PYG{l+s+s2}{residue in the sieve [g]}\PYG{l+s+s2}{\PYGZdq{}}\PYG{p}{:} \PYG{n}{mass}\PYG{p}{,} \PYG{l+s+s2}{\PYGZdq{}}\PYG{l+s+s2}{Σtotal}\PYG{l+s+s2}{\PYGZdq{}}\PYG{p}{:} \PYG{n}{retain\PYGZus{}per}\PYG{p}{,} \PYG{l+s+s2}{\PYGZdq{}}\PYG{l+s+s2}{Σ/Σtotal}\PYG{l+s+s2}{\PYGZdq{}}\PYG{p}{:} \PYG{n}{passing\PYGZus{}per} \PYG{p}{\PYGZcb{}}

\PYG{n}{df1}\PYG{o}{=} \PYG{n}{pd}\PYG{o}{.}\PYG{n}{DataFrame}\PYG{p}{(}\PYG{n}{data}\PYG{p}{)}
\PYG{n}{df1} 
\end{sphinxVerbatim}

\begin{sphinxVerbatim}[commandchars=\\\{\}]
   mesh diameter [mm]  residue in the sieve [g]     Σtotal    Σ/Σtotal
0               6.000                         0   0.000000  100.000000
1               2.000                        40   7.843137   92.156863
2               0.600                       250  49.019608   43.137255
3               0.200                       150  29.411765   13.725490
4               0.060                        60  11.764706    1.960784
5               0.001                        10   1.960784    0.000000
\end{sphinxVerbatim}

\begin{sphinxVerbatim}[commandchars=\\\{\}]
\PYG{c+c1}{\PYGZsh{} plotting}
\PYG{n}{plt}\PYG{o}{.}\PYG{n}{rcParams}\PYG{p}{[}\PYG{l+s+s1}{\PYGZsq{}}\PYG{l+s+s1}{axes.linewidth}\PYG{l+s+s1}{\PYGZsq{}}\PYG{p}{]}\PYG{o}{=}\PYG{l+m+mi}{2}
\PYG{c+c1}{\PYGZsh{}plt.rcParams[\PYGZdq{}axes.edgecolor\PYGZdq{}]=\PYGZsq{}white\PYGZsq{}}
\PYG{n}{plt}\PYG{o}{.}\PYG{n}{rcParams}\PYG{p}{[}\PYG{l+s+s1}{\PYGZsq{}}\PYG{l+s+s1}{grid.linestyle}\PYG{l+s+s1}{\PYGZsq{}}\PYG{p}{]}\PYG{o}{=}\PYG{l+s+s1}{\PYGZsq{}}\PYG{l+s+s1}{\PYGZhy{}\PYGZhy{}}\PYG{l+s+s1}{\PYGZsq{}}
\PYG{n}{plt}\PYG{o}{.}\PYG{n}{rcParams}\PYG{p}{[}\PYG{l+s+s1}{\PYGZsq{}}\PYG{l+s+s1}{grid.linewidth}\PYG{l+s+s1}{\PYGZsq{}}\PYG{p}{]}\PYG{o}{=}\PYG{l+m+mi}{1}
\PYG{n}{x} \PYG{o}{=} \PYG{n}{np}\PYG{o}{.}\PYG{n}{append}\PYG{p}{(}\PYG{p}{[}\PYG{l+m+mi}{10}\PYG{p}{]}\PYG{p}{,}\PYG{n}{dia}\PYG{p}{)} \PYG{c+c1}{\PYGZsh{} adding data to extend over 6 mm dia}
\PYG{n}{y} \PYG{o}{=} \PYG{n}{np}\PYG{o}{.}\PYG{n}{append}\PYG{p}{(}\PYG{p}{[}\PYG{l+m+mi}{100}\PYG{p}{]}\PYG{p}{,}\PYG{n}{passing\PYGZus{}per}\PYG{p}{)} \PYG{c+c1}{\PYGZsh{} adding 100\PYGZpc{} to plot}

\PYG{n}{fig} \PYG{o}{=} \PYG{n}{plt}\PYG{o}{.}\PYG{n}{figure}\PYG{p}{(}\PYG{n}{figsize}\PYG{o}{=}\PYG{p}{(}\PYG{l+m+mi}{9}\PYG{p}{,}\PYG{l+m+mi}{6}\PYG{p}{)}\PYG{p}{)}
\PYG{n}{plt}\PYG{o}{.}\PYG{n}{semilogx}\PYG{p}{(}\PYG{n}{x}\PYG{p}{,} \PYG{n}{y}\PYG{p}{,} \PYG{l+s+s1}{\PYGZsq{}}\PYG{l+s+s1}{x\PYGZhy{}}\PYG{l+s+s1}{\PYGZsq{}}\PYG{p}{,} \PYG{n}{color}\PYG{o}{=}\PYG{l+s+s1}{\PYGZsq{}}\PYG{l+s+s1}{red}\PYG{l+s+s1}{\PYGZsq{}}\PYG{p}{)}  
\PYG{n}{tics}\PYG{o}{=}\PYG{n}{x}\PYG{o}{.}\PYG{n}{tolist}\PYG{p}{(}\PYG{p}{)}

\PYG{n}{plt}\PYG{o}{.}\PYG{n}{grid}\PYG{p}{(}\PYG{n}{which}\PYG{o}{=}\PYG{l+s+s1}{\PYGZsq{}}\PYG{l+s+s1}{major}\PYG{l+s+s1}{\PYGZsq{}}\PYG{p}{,} \PYG{n}{color}\PYG{o}{=}\PYG{l+s+s1}{\PYGZsq{}}\PYG{l+s+s1}{k}\PYG{l+s+s1}{\PYGZsq{}}\PYG{p}{,} \PYG{n}{alpha}\PYG{o}{=}\PYG{l+m+mf}{0.7}\PYG{p}{)} 
\PYG{n}{plt}\PYG{o}{.}\PYG{n}{grid}\PYG{p}{(}\PYG{n}{which}\PYG{o}{=}\PYG{l+s+s1}{\PYGZsq{}}\PYG{l+s+s1}{minor}\PYG{l+s+s1}{\PYGZsq{}}\PYG{p}{,} \PYG{n}{color}\PYG{o}{=}\PYG{l+s+s1}{\PYGZsq{}}\PYG{l+s+s1}{k}\PYG{l+s+s1}{\PYGZsq{}}\PYG{p}{,} \PYG{n}{alpha}\PYG{o}{=}\PYG{l+m+mf}{0.3}\PYG{p}{)}
\PYG{n}{plt}\PYG{o}{.}\PYG{n}{xticks}\PYG{p}{(}\PYG{n}{x}\PYG{p}{,} \PYG{n}{tics}\PYG{p}{)}\PYG{p}{;}  
\PYG{n}{plt}\PYG{o}{.}\PYG{n}{yticks}\PYG{p}{(}\PYG{n}{np}\PYG{o}{.}\PYG{n}{arange}\PYG{p}{(}\PYG{l+m+mi}{0}\PYG{p}{,}\PYG{l+m+mi}{110}\PYG{p}{,}\PYG{l+m+mi}{10}\PYG{p}{)}\PYG{p}{)}\PYG{p}{;}
\PYG{n}{plt}\PYG{o}{.}\PYG{n}{title}\PYG{p}{(}\PYG{l+s+s1}{\PYGZsq{}}\PYG{l+s+s1}{grain size distribution}\PYG{l+s+s1}{\PYGZsq{}}\PYG{p}{)}\PYG{p}{;}
\PYG{n}{plt}\PYG{o}{.}\PYG{n}{xlabel}\PYG{p}{(}\PYG{l+s+s1}{\PYGZsq{}}\PYG{l+s+s1}{grain size d [mm]}\PYG{l+s+s1}{\PYGZsq{}}\PYG{p}{)}\PYG{p}{;}
\PYG{n}{plt}\PYG{o}{.}\PYG{n}{ylabel}\PYG{p}{(}\PYG{l+s+s1}{\PYGZsq{}}\PYG{l+s+s1}{grain fraction \PYGZlt{} d ins }\PYG{l+s+si}{\PYGZpc{} o}\PYG{l+s+s1}{f total mass}\PYG{l+s+s1}{\PYGZsq{}}\PYG{p}{)}\PYG{p}{;}
\end{sphinxVerbatim}

\noindent\sphinxincludegraphics{{GW_exam_2019_20_16_0}.png}

\sphinxstylestrong{solution 4b}

The sample can be considered uniformly distributed as over 70\% of sample falls in the sand size (0.2 mm\sphinxhyphen{}2 mm). Therefore, the sample can be considered sandy.

\sphinxstylestrong{Q5. Aquifer characterization} (ca. 8 pts.)

Water levels in m a.s.l. were measured at three observation wells (see figure).

\noindent\sphinxincludegraphics{{contents/questions/figs/Q5_2019-2020}.png}

a. Sketch hydraulic head isolines for increments of 0.5 m. (ca. 3 points.)

b. Gravel layer (thickness (t1) = 1.5 m, and conductivity (K1) = 3.7 10\sphinxhyphen{}3 is embedded between two sandy layers (t2 = 2 m, K2 = 3·10\sphinxhyphen{}4 m/s; and t3 = 3 m, K3 = 4·10\sphinxhyphen{}4 m/s). If the hydraulic gradient is 1\% and overall discharge is 1 m3/d per unit width of the aquifer, find the effective hydraulic conductivity considering a parallelly layered aquifer.

(Hint: \(K_{eff} = \frac{m}{\sum_{i=1}^n \frac{m_i}{K_i}}\) or \(K_{eff} =  \sum_{i=1}^n\frac{m_i\cdot K_i}{m}\)  ) (ca. 2 points)

c. Distinguish between homogeneity and heterogeneity, and isotropy and anisotropy (ca. 3 points)

\sphinxstylestrong{Solution 5a}
(L07/08\sphinxhyphen{}09)

The isolines and flow direction is provided in the figure below.

\noindent\sphinxincludegraphics{{contents/questions/figs/Q5a_2019-2020}.png}

\begin{sphinxVerbatim}[commandchars=\\\{\}]
\PYG{c+c1}{\PYGZsh{} Solution 5b (L06/08\PYGZhy{}13)}

\PYG{c+c1}{\PYGZsh{} Given:}

\PYG{n}{G\PYGZus{}t1} \PYG{o}{=} \PYG{l+m+mi}{2} \PYG{c+c1}{\PYGZsh{} m, sandy layer top}
\PYG{n}{G\PYGZus{}t2} \PYG{o}{=} \PYG{l+m+mf}{1.5} \PYG{c+c1}{\PYGZsh{} m, gravel layer middle}
\PYG{n}{G\PYGZus{}t3} \PYG{o}{=} \PYG{l+m+mi}{3} \PYG{c+c1}{\PYGZsh{} m, sandy layer bottom }
\PYG{n}{K\PYGZus{}1} \PYG{o}{=}  \PYG{l+m+mf}{3.0}\PYG{o}{*}\PYG{l+m+mi}{10}\PYG{o}{*}\PYG{o}{*}\PYG{o}{\PYGZhy{}}\PYG{l+m+mi}{4} \PYG{c+c1}{\PYGZsh{} m/s cond. in G\PYGZus{}t1}
\PYG{n}{K\PYGZus{}2} \PYG{o}{=} \PYG{l+m+mf}{3.7}\PYG{o}{*}\PYG{l+m+mi}{10}\PYG{o}{*}\PYG{o}{*}\PYG{o}{\PYGZhy{}}\PYG{l+m+mi}{3} \PYG{c+c1}{\PYGZsh{} m/s cond. in G\PYGZus{}t2}
\PYG{n}{K\PYGZus{}3} \PYG{o}{=}  \PYG{l+m+mf}{4.0}\PYG{o}{*}\PYG{l+m+mi}{10}\PYG{o}{*}\PYG{o}{*}\PYG{o}{\PYGZhy{}}\PYG{l+m+mi}{4} \PYG{c+c1}{\PYGZsh{} m/s cond. in G\PYGZus{}t3}
\PYG{n}{i} \PYG{o}{=} \PYG{l+m+mi}{1}\PYG{o}{/}\PYG{l+m+mi}{100} \PYG{c+c1}{\PYGZsh{} (), hydraulic gradient 1\PYGZpc{}}
\PYG{n}{Q\PYGZus{}5} \PYG{o}{=} \PYG{l+m+mi}{1} \PYG{c+c1}{\PYGZsh{} m\(\sp{\text{3}}\)/d per\PYGZhy{}W, discharge per unit width}

\PYG{c+c1}{\PYGZsh{}intermediate calculation}
\PYG{n}{m} \PYG{o}{=} \PYG{n}{G\PYGZus{}t1}\PYG{o}{+}\PYG{n}{G\PYGZus{}t2}\PYG{o}{+}\PYG{n}{G\PYGZus{}t3} \PYG{c+c1}{\PYGZsh{} m, total aq. thickness}

\PYG{n}{K\PYGZus{}ef\PYGZus{}h} \PYG{o}{=} \PYG{p}{(}\PYG{l+m+mi}{1}\PYG{o}{/}\PYG{n}{m}\PYG{p}{)} \PYG{o}{*} \PYG{p}{(}\PYG{n}{G\PYGZus{}t1}\PYG{o}{*}\PYG{n}{K\PYGZus{}1} \PYG{o}{+} \PYG{n}{G\PYGZus{}t2}\PYG{o}{*}\PYG{n}{K\PYGZus{}2} \PYG{o}{+} \PYG{n}{G\PYGZus{}t3}\PYG{o}{*}\PYG{n}{K\PYGZus{}3}\PYG{p}{)} \PYG{c+c1}{\PYGZsh{} m/s, eff. horizontal cond.}
\PYG{n}{K\PYGZus{}ef\PYGZus{}v} \PYG{o}{=} \PYG{n}{m}\PYG{o}{/}\PYG{p}{(}\PYG{n}{G\PYGZus{}t1}\PYG{o}{/}\PYG{n}{K\PYGZus{}1} \PYG{o}{+} \PYG{n}{G\PYGZus{}t2}\PYG{o}{/}\PYG{n}{K\PYGZus{}2} \PYG{o}{+} \PYG{n}{G\PYGZus{}t3}\PYG{o}{/}\PYG{n}{K\PYGZus{}3}\PYG{p}{)} \PYG{c+c1}{\PYGZsh{} \PYGZsh{} m/s, eff. vertical cond.}

\PYG{n+nb}{print}\PYG{p}{(}\PYG{l+s+s2}{\PYGZdq{}}\PYG{l+s+s2}{The thickness of the aquifer is }\PYG{l+s+si}{\PYGZob{}0:0.3f\PYGZcb{}}\PYG{l+s+s2}{\PYGZdq{}}\PYG{o}{.}\PYG{n}{format}\PYG{p}{(}\PYG{n}{m}\PYG{p}{)}\PYG{p}{,} \PYG{l+s+s2}{\PYGZdq{}}\PYG{l+s+s2}{m}\PYG{l+s+s2}{\PYGZdq{}}\PYG{p}{)}
\PYG{n+nb}{print}\PYG{p}{(}\PYG{l+s+s2}{\PYGZdq{}}\PYG{l+s+s2}{The effective horizontal conductivity of the aquifer is }\PYG{l+s+si}{\PYGZob{}0:0.2E\PYGZcb{}}\PYG{l+s+s2}{\PYGZdq{}}\PYG{o}{.}\PYG{n}{format}\PYG{p}{(}\PYG{n}{K\PYGZus{}ef\PYGZus{}h}\PYG{p}{)}\PYG{p}{,} \PYG{l+s+s2}{\PYGZdq{}}\PYG{l+s+s2}{m/s}\PYG{l+s+s2}{\PYGZdq{}}\PYG{p}{)}
\PYG{n+nb}{print}\PYG{p}{(}\PYG{l+s+s2}{\PYGZdq{}}\PYG{l+s+s2}{The effective vertical conductivity of the aquifer is }\PYG{l+s+si}{\PYGZob{}0:0.2E\PYGZcb{}}\PYG{l+s+s2}{\PYGZdq{}}\PYG{o}{.}\PYG{n}{format}\PYG{p}{(}\PYG{n}{K\PYGZus{}ef\PYGZus{}v}\PYG{p}{)}\PYG{p}{,} \PYG{l+s+s2}{\PYGZdq{}}\PYG{l+s+s2}{m/s}\PYG{l+s+s2}{\PYGZdq{}}\PYG{p}{)}
\end{sphinxVerbatim}

\begin{sphinxVerbatim}[commandchars=\\\{\}]
The thickness of the aquifer is 6.500 m
The effective horizontal conductivity of the aquifer is 1.13E\PYGZhy{}03 m/s
The effective vertical conductivity of the aquifer is 4.46E\PYGZhy{}04 m/s
\end{sphinxVerbatim}

\sphinxstylestrong{Solution 5c} (L06/23)

\sphinxstylestrong{Homogeneity}: An aquifer is homogeneous when its parameters are constant throughout the porous medium, i.e. the properties of the medium are independent of space

\sphinxstylestrong{Heterogeneity}: Heterogeneous aquifer have its properties varies in space or the properties are space dependent.

\sphinxstylestrong{Isotropy}: This relates to properties of aquifer being independent of direction, i.e., \(K_v = K_h\)

\sphinxstylestrong{Anisotropy}: In this case the aquifer properties are direction dependent, i.e., \(K_v \neq K_h\).

\sphinxstylestrong{Q6. Well} (ca. 5 pts.)

a. Sketch the pumping scenario of an unconfined aquifer (vertical cross section) and label all possible quantities (ca 3 pts.)

b. The conductivity of a confined aquifer (8 m thick) is estimated to be \(4\cdot 10^{-4}\) m/s. If the steady\sphinxhyphen{}state discharge 50 m3/s, using the Theis equation (\(s = Q/4\pi T·W(u)\), with \(W(u) = 15\)), find the drawdown in the aquifer. (2 points)

\sphinxstylestrong{Solution 6a}
(L08/16)

Figure below presents the scenario of a well in an \sphinxstyleemphasis{unconfined} aquifer.

\noindent\sphinxincludegraphics{{contents/questions/figs/Q6_2019-2020}.png}

\begin{sphinxVerbatim}[commandchars=\\\{\}]
\PYG{c+c1}{\PYGZsh{}Solution 6b}

\PYG{c+c1}{\PYGZsh{}Given}

\PYG{n}{Q\PYGZus{}6} \PYG{o}{=} \PYG{l+m+mi}{50} \PYG{c+c1}{\PYGZsh{} m\(\sp{\text{3}}\)/s, discharge}
\PYG{n}{K\PYGZus{}6} \PYG{o}{=} \PYG{l+m+mi}{4}\PYG{o}{*}\PYG{l+m+mi}{10}\PYG{o}{*}\PYG{o}{*}\PYG{o}{\PYGZhy{}}\PYG{l+m+mi}{4} \PYG{c+c1}{\PYGZsh{} m/s, conductivity}
\PYG{n}{m\PYGZus{}6} \PYG{o}{=} \PYG{l+m+mi}{8} \PYG{c+c1}{\PYGZsh{} m, thickness}
\PYG{n}{W\PYGZus{}u} \PYG{o}{=} \PYG{l+m+mi}{15} \PYG{c+c1}{\PYGZsh{} (), well function}

\PYG{c+c1}{\PYGZsh{} interim cal.}
\PYG{n}{T\PYGZus{}6} \PYG{o}{=} \PYG{n}{K\PYGZus{}6} \PYG{o}{*} \PYG{n}{m\PYGZus{}6} \PYG{c+c1}{\PYGZsh{} m\(\sp{\text{2}}\)/s, Transmissivity T = K*m}

\PYG{c+c1}{\PYGZsh{} solution}
\PYG{n}{s\PYGZus{}6} \PYG{o}{=} \PYG{p}{(}\PYG{n}{Q\PYGZus{}6}\PYG{o}{/}\PYG{p}{(}\PYG{l+m+mi}{4}\PYG{o}{*}\PYG{n}{np}\PYG{o}{.}\PYG{n}{pi}\PYG{o}{*}\PYG{n}{T\PYGZus{}6}\PYG{p}{)}\PYG{p}{)} \PYG{o}{*} \PYG{n}{W\PYGZus{}u} \PYG{c+c1}{\PYGZsh{} m, drawdown}

\PYG{n+nb}{print}\PYG{p}{(}\PYG{l+s+s2}{\PYGZdq{}}\PYG{l+s+s2}{The Transmissivity of the aquifer is }\PYG{l+s+si}{\PYGZob{}0:0.5f\PYGZcb{}}\PYG{l+s+s2}{\PYGZdq{}}\PYG{o}{.}\PYG{n}{format}\PYG{p}{(}\PYG{n}{T\PYGZus{}6}\PYG{p}{)}\PYG{p}{,} \PYG{l+s+s2}{\PYGZdq{}}\PYG{l+s+s2}{m}\PYG{l+s+se}{\PYGZbs{}u00b2}\PYG{l+s+s2}{/s}\PYG{l+s+s2}{\PYGZdq{}}\PYG{p}{)}
\PYG{n+nb}{print}\PYG{p}{(}\PYG{l+s+s2}{\PYGZdq{}}\PYG{l+s+s2}{The drawdown in the well is }\PYG{l+s+si}{\PYGZob{}0:0.2f\PYGZcb{}}\PYG{l+s+s2}{\PYGZdq{}}\PYG{o}{.}\PYG{n}{format}\PYG{p}{(}\PYG{n}{s\PYGZus{}6}\PYG{p}{)}\PYG{p}{,} \PYG{l+s+s2}{\PYGZdq{}}\PYG{l+s+s2}{m}\PYG{l+s+s2}{\PYGZdq{}}\PYG{p}{)}
\end{sphinxVerbatim}

\begin{sphinxVerbatim}[commandchars=\\\{\}]
The Transmissivity of the aquifer is 0.00320 m\(\sp{\text{2}}\)/s
The drawdown in the well is 18650.97 m
\end{sphinxVerbatim}

\sphinxstylestrong{Q7. Conservative Transport}  (ca. 7 pts.)

a. How is reactive transport different to conservative transport in the aquifers. (2 points)

b. With suitable sketch distinguish between advective flux and dispersive flux. (2 points)

c. A column (L = 1.2 m and Ø = 5 cm) was packed with sandy soil (ne= 35\%  K= 0,0002 m/s). The hydraulic head at the inlet and the outlet was set to 230 m and 235 m, resp. The NaCl solution with conc. 10 mg/L was steadily introduced to the column after saturating it with distilled water. The experiment condition was such that diffusive flow could be neglected.  You may make justified assumption for any missing information.

c.i. What will be the advective mass flux at the outlet of the column? (1.5 points)

c.ii. Considering initial concentration difference between inlet and outlet to be 10 mg/L, what    will be the dispersive mass flux at the outlet? (1.5 points)

(Hint: Dispersive and Advective fluxes are either of \( n_e \cdot v\cdot C\) and \(n_e\cdot \alpha \cdot v \cdot \Delta C/L\))

\sphinxstylestrong{Solution 7a} (L09/05)

A chemical in groundwater is subject to conservative transport processes if there is:
\begin{itemize}
\item {} 
no interaction with the solid material,

\item {} 
no interaction with other chemicals,

\item {} 
no interaction with microbes.

\end{itemize}

When either of the above are part of the groundwater, the transport process is reactive.

\sphinxstylestrong{Solution 7b} (L09/09)

The sketch below distinguish between advective and dispersive fluxes. The figure in the left is of advective process and that in the right results to dispersive flux.

\noindent\sphinxincludegraphics{{contents/questions/figs/Q7b_2019-2020}.png}

\begin{sphinxVerbatim}[commandchars=\\\{\}]
\PYG{c+c1}{\PYGZsh{}Solution 7c}

\PYG{n}{L\PYGZus{}7} \PYG{o}{=} \PYG{l+m+mf}{1.2} \PYG{c+c1}{\PYGZsh{} m, col. length}
\PYG{n}{Dia\PYGZus{}7} \PYG{o}{=} \PYG{l+m+mi}{5} \PYG{c+c1}{\PYGZsh{} cm, col. diameter }
\PYG{n}{ne\PYGZus{}7} \PYG{o}{=} \PYG{l+m+mf}{0.35} \PYG{c+c1}{\PYGZsh{} (), effective porosity}
\PYG{n}{K\PYGZus{}7} \PYG{o}{=} \PYG{l+m+mf}{0.0002} \PYG{c+c1}{\PYGZsh{} m/s, conductivity}
\PYG{n}{H\PYGZus{}7in} \PYG{o}{=} \PYG{l+m+mi}{235} \PYG{c+c1}{\PYGZsh{} m, head inlet}
\PYG{n}{H\PYGZus{}7out} \PYG{o}{=} \PYG{l+m+mi}{230} \PYG{c+c1}{\PYGZsh{} m, head outlet}
\PYG{n}{C\PYGZus{}7} \PYG{o}{=} \PYG{l+m+mi}{10} \PYG{c+c1}{\PYGZsh{} mg/L, NaCl concentration}
\PYG{n}{al\PYGZus{}7} \PYG{o}{=} \PYG{l+m+mi}{1} \PYG{c+c1}{\PYGZsh{} m, assumed}
\PYG{n}{C\PYGZus{}7d} \PYG{o}{=} \PYG{l+m+mi}{10} \PYG{c+c1}{\PYGZsh{} mg/L}

\PYG{c+c1}{\PYGZsh{}intermediate calc.}
\PYG{n}{i\PYGZus{}7} \PYG{o}{=} \PYG{p}{(}\PYG{n}{H\PYGZus{}7in}\PYG{o}{\PYGZhy{}}\PYG{n}{H\PYGZus{}7out}\PYG{p}{)}\PYG{o}{/}\PYG{n}{L\PYGZus{}7} \PYG{c+c1}{\PYGZsh{} (), head gradient}
\PYG{n}{v\PYGZus{}7dar} \PYG{o}{=} \PYG{n}{K\PYGZus{}7}\PYG{o}{*}\PYG{n}{i\PYGZus{}7} \PYG{c+c1}{\PYGZsh{} m/s, darcy velocity}
\PYG{n}{v\PYGZus{}7av} \PYG{o}{=} \PYG{n}{v\PYGZus{}7dar}\PYG{o}{/}\PYG{n}{ne\PYGZus{}7} \PYG{c+c1}{\PYGZsh{} m/s, average linear velocity}


\PYG{c+c1}{\PYGZsh{} Solution}
\PYG{n}{F\PYGZus{}7ad} \PYG{o}{=} \PYG{n}{ne\PYGZus{}7}\PYG{o}{*}\PYG{n}{v\PYGZus{}7av}\PYG{o}{*}\PYG{n}{C\PYGZus{}7} \PYG{c+c1}{\PYGZsh{} mg\PYGZhy{}m/L\PYGZhy{}s, advective flux }
\PYG{n}{F\PYGZus{}7dis} \PYG{o}{=} \PYG{n}{ne\PYGZus{}7}\PYG{o}{*}\PYG{n}{al\PYGZus{}7}\PYG{o}{*}\PYG{n}{v\PYGZus{}7av}\PYG{o}{*}\PYG{n}{C\PYGZus{}7d}\PYG{o}{/}\PYG{n}{L\PYGZus{}7} \PYG{c+c1}{\PYGZsh{} mg\PYGZhy{}m/L\PYGZhy{}s, dispersive flux }

\PYG{n+nb}{print}\PYG{p}{(}\PYG{l+s+s2}{\PYGZdq{}}\PYG{l+s+s2}{The hydraulic gradient is }\PYG{l+s+si}{\PYGZob{}0:0.4f\PYGZcb{}}\PYG{l+s+s2}{\PYGZdq{}}\PYG{o}{.}\PYG{n}{format}\PYG{p}{(}\PYG{n}{i\PYGZus{}7}\PYG{p}{)}\PYG{p}{,} \PYG{l+s+s2}{\PYGZdq{}}\PYG{l+s+s2}{\PYGZdq{}}\PYG{p}{)}
\PYG{n+nb}{print}\PYG{p}{(}\PYG{l+s+s2}{\PYGZdq{}}\PYG{l+s+s2}{The Darcy velocity is }\PYG{l+s+si}{\PYGZob{}0:0.4f\PYGZcb{}}\PYG{l+s+s2}{\PYGZdq{}}\PYG{o}{.}\PYG{n}{format}\PYG{p}{(}\PYG{n}{v\PYGZus{}7dar}\PYG{p}{)}\PYG{p}{,} \PYG{l+s+s2}{\PYGZdq{}}\PYG{l+s+s2}{m/s}\PYG{l+s+s2}{\PYGZdq{}}\PYG{p}{)}
\PYG{n+nb}{print}\PYG{p}{(}\PYG{l+s+s2}{\PYGZdq{}}\PYG{l+s+s2}{The average linear velocity is }\PYG{l+s+si}{\PYGZob{}0:0.4f\PYGZcb{}}\PYG{l+s+s2}{\PYGZdq{}}\PYG{o}{.}\PYG{n}{format}\PYG{p}{(}\PYG{n}{v\PYGZus{}7av}\PYG{p}{)}\PYG{p}{,} \PYG{l+s+s2}{\PYGZdq{}}\PYG{l+s+s2}{m/s}\PYG{l+s+s2}{\PYGZdq{}}\PYG{p}{)}
\PYG{n+nb}{print}\PYG{p}{(}\PYG{l+s+s2}{\PYGZdq{}}\PYG{l+s+s2}{The advective flux is }\PYG{l+s+si}{\PYGZob{}0:0.10f\PYGZcb{}}\PYG{l+s+s2}{\PYGZdq{}}\PYG{o}{.}\PYG{n}{format}\PYG{p}{(}\PYG{n}{F\PYGZus{}7ad}\PYG{p}{)}\PYG{p}{,} \PYG{l+s+s2}{\PYGZdq{}}\PYG{l+s+s2}{mg\PYGZhy{}m/L\PYGZhy{}s}\PYG{l+s+s2}{\PYGZdq{}}\PYG{p}{)}
\PYG{n+nb}{print}\PYG{p}{(}\PYG{l+s+s2}{\PYGZdq{}}\PYG{l+s+s2}{The dispersive flux is }\PYG{l+s+si}{\PYGZob{}0:0.10f\PYGZcb{}}\PYG{l+s+s2}{\PYGZdq{}}\PYG{o}{.}\PYG{n}{format}\PYG{p}{(}\PYG{n}{F\PYGZus{}7dis}\PYG{p}{)}\PYG{p}{,} \PYG{l+s+s2}{\PYGZdq{}}\PYG{l+s+s2}{mg\PYGZhy{}m/L\PYGZhy{}s}\PYG{l+s+s2}{\PYGZdq{}}\PYG{p}{)}
\end{sphinxVerbatim}

\begin{sphinxVerbatim}[commandchars=\\\{\}]
The hydraulic gradient is 4.1667 
The Darcy velocity is 0.0008 m/s
The average linear velocity is 0.0024 m/s
The advective flux is 0.0083333333 mg\PYGZhy{}m/L\PYGZhy{}s
The dispersive flux is 0.0069444444 mg\PYGZhy{}m/L\PYGZhy{}s
\end{sphinxVerbatim}

\sphinxstylestrong{Q8. Sorption Isotherms} (ca. 10 pts)

Five batch tests (different initial concentrations \textendash{} see table below) were performed to determine the sorption properties of a sediment. For each batch 20 g of sediment in 30 mL of water were used. The measured equilibrium solute concentrations are also provided in the table.

a)	Complete the above table								(ca. 3 pts.)

b)	Plot the results in the diagram below and draw a Henry isotherm			(ca. 3 pts.)

c)	How is retardation related to isotherm (ca. 2 points)
(value and unit!)		(ca. 2 pts.).

\begin{sphinxVerbatim}[commandchars=\\\{\}]
\PYG{n}{head} \PYG{o}{=} \PYG{p}{[}\PYG{l+s+s2}{\PYGZdq{}}\PYG{l+s+s2}{Batch nr. }\PYG{l+s+s2}{\PYGZdq{}}\PYG{p}{,} \PYG{l+s+s2}{\PYGZdq{}}\PYG{l+s+s2}{Initial Conc. (mg/L) }\PYG{l+s+s2}{\PYGZdq{}}\PYG{p}{,} \PYG{l+s+s2}{\PYGZdq{}}\PYG{l+s+s2}{Equi. Conc. (mg/L)}\PYG{l+s+s2}{\PYGZdq{}}\PYG{p}{,} \PYG{l+s+s2}{\PYGZdq{}}\PYG{l+s+s2}{Sorbed mass (g)}\PYG{l+s+s2}{\PYGZdq{}}\PYG{p}{,} \PYG{l+s+s2}{\PYGZdq{}}\PYG{l+s+s2}{Sorbed mass/solid (mg/g)}\PYG{l+s+s2}{\PYGZdq{}} \PYG{p}{]}
\PYG{n}{bn} \PYG{o}{=} \PYG{n}{np}\PYG{o}{.}\PYG{n}{array}\PYG{p}{(}\PYG{p}{[}\PYG{l+m+mi}{1}\PYG{p}{,}\PYG{l+m+mi}{2}\PYG{p}{,}\PYG{l+m+mi}{3}\PYG{p}{,}\PYG{l+m+mi}{4}\PYG{p}{,}\PYG{l+m+mi}{5}\PYG{p}{]}\PYG{p}{)}
\PYG{n}{C\PYGZus{}0} \PYG{o}{=} \PYG{n}{np}\PYG{o}{.}\PYG{n}{array}\PYG{p}{(}\PYG{p}{[}\PYG{l+m+mi}{5}\PYG{p}{,} \PYG{l+m+mi}{10}\PYG{p}{,} \PYG{l+m+mi}{15}\PYG{p}{,} \PYG{l+m+mi}{20}\PYG{p}{,} \PYG{l+m+mi}{25}\PYG{p}{]}\PYG{p}{)}\PYG{c+c1}{\PYGZsh{} mg/L, initial conc.}
\PYG{n}{C\PYGZus{}eq} \PYG{o}{=} \PYG{n}{np}\PYG{o}{.}\PYG{n}{array}\PYG{p}{(}\PYG{p}{[}\PYG{l+m+mf}{2.5}\PYG{p}{,} \PYG{l+m+mf}{4.9}\PYG{p}{,} \PYG{l+m+mi}{8}\PYG{p}{,} \PYG{l+m+mf}{9.8}\PYG{p}{,} \PYG{l+m+mf}{13.2}\PYG{p}{]}\PYG{p}{)}\PYG{c+c1}{\PYGZsh{} mg/L, equilibrium conc.}
\PYG{n}{s2} \PYG{o}{=} \PYG{n}{ips}\PYG{o}{.}\PYG{n}{sheet}\PYG{p}{(}\PYG{n}{rows}\PYG{o}{=}\PYG{l+m+mi}{6}\PYG{p}{,} \PYG{n}{columns}\PYG{o}{=}\PYG{l+m+mi}{5}\PYG{p}{,} \PYG{n}{row\PYGZus{}headers}\PYG{o}{=}\PYG{k+kc}{False}\PYG{p}{,} \PYG{n}{column\PYGZus{}headers}\PYG{o}{=}\PYG{n}{head}\PYG{p}{)}
\PYG{n}{ips}\PYG{o}{.}\PYG{n}{column}\PYG{p}{(}\PYG{l+m+mi}{0}\PYG{p}{,} \PYG{n}{bn}\PYG{p}{,} \PYG{n}{row\PYGZus{}start}\PYG{o}{=}\PYG{l+m+mi}{0}\PYG{p}{)} 
\PYG{n}{ips}\PYG{o}{.}\PYG{n}{column}\PYG{p}{(}\PYG{l+m+mi}{1}\PYG{p}{,} \PYG{n}{C\PYGZus{}0}\PYG{p}{,} \PYG{n}{row\PYGZus{}start}\PYG{o}{=}\PYG{l+m+mi}{0}\PYG{p}{)}
\PYG{n}{ips}\PYG{o}{.}\PYG{n}{column}\PYG{p}{(}\PYG{l+m+mi}{2}\PYG{p}{,} \PYG{n}{C\PYGZus{}eq}\PYG{p}{,} \PYG{n}{row\PYGZus{}start}\PYG{o}{=}\PYG{l+m+mi}{0}\PYG{p}{)}\PYG{p}{;} 
\PYG{n}{s2}
\end{sphinxVerbatim}

\begin{sphinxVerbatim}[commandchars=\\\{\}]
Sheet(cells=(Cell(column\PYGZus{}end=0, column\PYGZus{}start=0, row\PYGZus{}end=4, row\PYGZus{}start=0, squeeze\PYGZus{}row=False, type=\PYGZsq{}numeric\PYGZsq{}, val…
\end{sphinxVerbatim}

\begin{sphinxVerbatim}[commandchars=\\\{\}]
\PYG{c+c1}{\PYGZsh{} SOlution of Problem 10 a (T07/HP9)}

\PYG{c+c1}{\PYGZsh{} Given}
\PYG{n}{v\PYGZus{}ml} \PYG{o}{=} \PYG{l+m+mi}{30} \PYG{c+c1}{\PYGZsh{} ml of water used in expt.}
\PYG{n}{v\PYGZus{}l} \PYG{o}{=} \PYG{n}{v\PYGZus{}ml}\PYG{o}{/}\PYG{l+m+mi}{1000} \PYG{c+c1}{\PYGZsh{} L, unit conversion}
\PYG{n}{m\PYGZus{}s} \PYG{o}{=} \PYG{l+m+mi}{20} \PYG{c+c1}{\PYGZsh{} g, solid mass used in expt.}

\PYG{n}{bn} \PYG{o}{=} \PYG{n}{np}\PYG{o}{.}\PYG{n}{array}\PYG{p}{(}\PYG{p}{[}\PYG{l+m+mi}{1}\PYG{p}{,}\PYG{l+m+mi}{2}\PYG{p}{,}\PYG{l+m+mi}{3}\PYG{p}{,}\PYG{l+m+mi}{4}\PYG{p}{,}\PYG{l+m+mi}{5}\PYG{p}{]}\PYG{p}{)}
\PYG{n}{C\PYGZus{}0} \PYG{o}{=} \PYG{n}{np}\PYG{o}{.}\PYG{n}{array}\PYG{p}{(}\PYG{p}{[}\PYG{l+m+mi}{5}\PYG{p}{,} \PYG{l+m+mi}{10}\PYG{p}{,} \PYG{l+m+mi}{15}\PYG{p}{,} \PYG{l+m+mi}{20}\PYG{p}{,} \PYG{l+m+mi}{25}\PYG{p}{]}\PYG{p}{)}\PYG{c+c1}{\PYGZsh{} mg/L, initial conc.}
\PYG{n}{C\PYGZus{}eq} \PYG{o}{=} \PYG{n}{np}\PYG{o}{.}\PYG{n}{array}\PYG{p}{(}\PYG{p}{[}\PYG{l+m+mf}{2.5}\PYG{p}{,} \PYG{l+m+mf}{4.9}\PYG{p}{,} \PYG{l+m+mi}{8}\PYG{p}{,} \PYG{l+m+mf}{9.8}\PYG{p}{,} \PYG{l+m+mf}{13.2}\PYG{p}{]}\PYG{p}{)}\PYG{c+c1}{\PYGZsh{} mg/L, equilibrium conc.}
\PYG{n}{s\PYGZus{}m} \PYG{o}{=} \PYG{p}{(}\PYG{n}{C\PYGZus{}0}\PYG{o}{\PYGZhy{}}\PYG{n}{C\PYGZus{}eq}\PYG{p}{)}\PYG{o}{*}\PYG{n}{v\PYGZus{}l}
\PYG{n}{m\PYGZus{}m} \PYG{o}{=} \PYG{n}{s\PYGZus{}m}\PYG{o}{/}\PYG{n}{m\PYGZus{}s}\PYG{c+c1}{\PYGZsh{} mg/g, mass ratio}

\PYG{c+c1}{\PYGZsh{}output}
\PYG{n}{d8} \PYG{o}{=} \PYG{p}{\PYGZob{}}\PYG{l+s+s2}{\PYGZdq{}}\PYG{l+s+s2}{Batch Nr}\PYG{l+s+s2}{\PYGZdq{}}\PYG{p}{:} \PYG{n}{bn}\PYG{p}{,} \PYG{l+s+s2}{\PYGZdq{}}\PYG{l+s+s2}{Initial Conc. (mg/L)}\PYG{l+s+s2}{\PYGZdq{}}\PYG{p}{:} \PYG{n}{C\PYGZus{}0}\PYG{p}{,} \PYG{l+s+s2}{\PYGZdq{}}\PYG{l+s+s2}{Equi. Conc. (mg/L)}\PYG{l+s+s2}{\PYGZdq{}}\PYG{p}{:} \PYG{n}{C\PYGZus{}eq}\PYG{p}{,} \PYG{l+s+s2}{\PYGZdq{}}\PYG{l+s+s2}{Sorbed mass (g)}\PYG{l+s+s2}{\PYGZdq{}}\PYG{p}{:} \PYG{n}{s\PYGZus{}m}\PYG{p}{,} \PYG{l+s+s2}{\PYGZdq{}}\PYG{l+s+s2}{Sorbed mass/solid (mg/g)}\PYG{l+s+s2}{\PYGZdq{}} \PYG{p}{:}\PYG{n}{m\PYGZus{}m}\PYG{p}{\PYGZcb{}}
\PYG{n}{df9} \PYG{o}{=} \PYG{n}{pd}\PYG{o}{.}\PYG{n}{DataFrame}\PYG{p}{(}\PYG{n}{d8}\PYG{p}{)}\PYG{p}{;} \PYG{n}{df9}
\end{sphinxVerbatim}

\begin{sphinxVerbatim}[commandchars=\\\{\}]
   Batch Nr  Initial Conc. (mg/L)  Equi. Conc. (mg/L)  Sorbed mass (g)  \PYGZbs{}
0         1                     5                 2.5            0.075   
1         2                    10                 4.9            0.153   
2         3                    15                 8.0            0.210   
3         4                    20                 9.8            0.306   
4         5                    25                13.2            0.354   

   Sorbed mass/solid (mg/g)  
0                   0.00375  
1                   0.00765  
2                   0.01050  
3                   0.01530  
4                   0.01770  
\end{sphinxVerbatim}

\begin{sphinxVerbatim}[commandchars=\\\{\}]
\PYG{c+c1}{\PYGZsh{} Solution of proble 10 (b) (T07/HP9)}
\PYG{c+c1}{\PYGZsh{} fit}
\PYG{n}{slope}\PYG{p}{,} \PYG{n}{intercept}\PYG{p}{,} \PYG{n}{r\PYGZus{}value}\PYG{p}{,} \PYG{n}{p\PYGZus{}value}\PYG{p}{,} \PYG{n}{std\PYGZus{}err} \PYG{o}{=} \PYG{n}{stats}\PYG{o}{.}\PYG{n}{linregress}\PYG{p}{(}\PYG{n}{C\PYGZus{}eq}\PYG{p}{,} \PYG{n}{m\PYGZus{}m}\PYG{p}{)} \PYG{c+c1}{\PYGZsh{} linear regression}

\PYG{c+c1}{\PYGZsh{}plot and fit}
\PYG{n}{fig} \PYG{o}{=} \PYG{n}{plt}\PYG{o}{.}\PYG{n}{figure}\PYG{p}{(}\PYG{p}{)}\PYG{p}{;} \PYG{n}{plt}\PYG{o}{.}\PYG{n}{plot}\PYG{p}{(}\PYG{n}{C\PYGZus{}eq}\PYG{p}{,} \PYG{n}{m\PYGZus{}m}\PYG{p}{,} \PYG{l+s+s1}{\PYGZsq{}}\PYG{l+s+s1}{bo}\PYG{l+s+s1}{\PYGZsq{}}\PYG{p}{,} \PYG{n}{label}\PYG{o}{=}\PYG{l+s+s1}{\PYGZsq{}}\PYG{l+s+s1}{ provided data}\PYG{l+s+s1}{\PYGZsq{}}\PYG{p}{)}\PYG{p}{;}
\PYG{n}{pred} \PYG{o}{=} \PYG{n}{intercept} \PYG{o}{+} \PYG{n}{slope}\PYG{o}{*}\PYG{n}{C\PYGZus{}eq} \PYG{c+c1}{\PYGZsh{} fit line}
\PYG{n}{plt}\PYG{o}{.}\PYG{n}{plot}\PYG{p}{(}\PYG{n}{C\PYGZus{}eq}\PYG{p}{,} \PYG{n}{pred}\PYG{p}{,} \PYG{l+s+s1}{\PYGZsq{}}\PYG{l+s+s1}{r}\PYG{l+s+s1}{\PYGZsq{}}\PYG{p}{,} \PYG{n}{label}\PYG{o}{=}\PYG{l+s+s1}{\PYGZsq{}}\PYG{l+s+s1}{y=}\PYG{l+s+si}{\PYGZob{}:.2E\PYGZcb{}}\PYG{l+s+s1}{x+}\PYG{l+s+si}{\PYGZob{}:.2f\PYGZcb{}}\PYG{l+s+s1}{\PYGZsq{}}\PYG{o}{.}\PYG{n}{format}\PYG{p}{(}\PYG{n}{slope}\PYG{p}{,}\PYG{n}{intercept}\PYG{p}{)}\PYG{p}{)} \PYG{p}{;}
\PYG{n}{plt}\PYG{o}{.}\PYG{n}{xlabel}\PYG{p}{(}\PYG{l+s+sa}{r}\PYG{l+s+s2}{\PYGZdq{}}\PYG{l+s+s2}{\PYGZdl{}C\PYGZus{}}\PYG{l+s+si}{\PYGZob{}eq\PYGZcb{}}\PYG{l+s+s2}{ \PYGZdl{} mg/L}\PYG{l+s+s2}{\PYGZdq{}}\PYG{p}{)}\PYG{p}{;} \PYG{n}{plt}\PYG{o}{.}\PYG{n}{ylabel}\PYG{p}{(}\PYG{l+s+sa}{r}\PYG{l+s+s2}{\PYGZdq{}}\PYG{l+s+s2}{\PYGZdl{}C\PYGZus{}}\PYG{l+s+si}{\PYGZob{}a\PYGZcb{}}\PYG{l+s+s2}{ \PYGZdl{} mg/g}\PYG{l+s+s2}{\PYGZdq{}}\PYG{p}{)}\PYG{p}{;}
\PYG{n}{plt}\PYG{o}{.}\PYG{n}{grid}\PYG{p}{(}\PYG{p}{)}\PYG{p}{;} \PYG{n}{plt}\PYG{o}{.}\PYG{n}{legend}\PYG{p}{(}\PYG{n}{fontsize}\PYG{o}{=}\PYG{l+m+mi}{11}\PYG{p}{)}\PYG{p}{;}  \PYG{n}{plt}\PYG{o}{.}\PYG{n}{text}\PYG{p}{(}\PYG{l+m+mf}{2.2}\PYG{p}{,} \PYG{l+m+mf}{0.014}\PYG{p}{,}\PYG{l+s+s1}{\PYGZsq{}}\PYG{l+s+s1}{\PYGZdl{}R\PYGZca{}2 = }\PYG{l+s+si}{\PYGZpc{}0.2f}\PYG{l+s+s1}{\PYGZdl{}}\PYG{l+s+s1}{\PYGZsq{}} \PYG{o}{\PYGZpc{}} \PYG{n}{r\PYGZus{}value}\PYG{p}{)}
\PYG{n}{plt}\PYG{o}{.}\PYG{n}{text}\PYG{p}{(}\PYG{l+m+mf}{2.2}\PYG{p}{,} \PYG{l+m+mf}{0.012}\PYG{p}{,}\PYG{l+s+s1}{\PYGZsq{}}\PYG{l+s+s1}{\PYGZdl{}C\PYGZus{}a = K\PYGZus{}}\PYG{l+s+si}{\PYGZob{}d\PYGZcb{}}\PYG{l+s+s1}{\PYGZbs{}}\PYG{l+s+s1}{cdot C\PYGZus{}}\PYG{l+s+si}{\PYGZob{}eq\PYGZcb{}}\PYG{l+s+s1}{\PYGZdl{}}\PYG{l+s+s1}{\PYGZsq{}}\PYG{p}{)}\PYG{p}{;} 
\end{sphinxVerbatim}

\noindent\sphinxincludegraphics{{GW_exam_2019_20_32_0}.png}

\sphinxstylestrong{solution 8c}
(L10/13)

The following relation relates Retardation (\(R\)) with linear isotherm (\(K_d\))
\begin{equation*}
\begin{split}
R = 1+ \frac{1-n_e}{n_e}\rho_s K_d
\end{split}
\end{equation*}
with effective porosity \(n_e\), solid density \(\rho_s\).

\sphinxstylestrong{Q9. Groundwater Modelling} (ca. 8 points.)

a. Distinguish between conceptual model and mathematical model; and between analytical solution and empirical solution (ca. 4 points).

b. Draw a conceptual model for a rectangular aquifer 100 m long and 20 m wide. Discretize the domain with 1/10 of the length length\sphinxhyphen{}wise and 1/5 of the width width\sphinxhyphen{}wise. Assure that flow in the model is from left to right direction (ca. 3 points).

c. How is a no\sphinxhyphen{}flow boundary condition mathematically defined? (ca. 1 point)

\sphinxstylestrong{Solution 9a}
(L11/04\sphinxhyphen{}06)

A model or also a \sphinxstyleemphasis{conceptual model} is a representation, an image or a description of a real system.

example for a real system: porous medium with water flowing through the pores (Darcy experiment)

A \sphinxstyleemphasis{mathematical model} provides a quantitative representation of the relevant system components, processes and impacts in the area of investigation. The quantitative representation is based on mathematical equations.

\sphinxstyleemphasis{Analytical solution} : These are exact mathematical expressions solving the model equations.

\sphinxstyleemphasis{Emperical solution} : These are solution based on experimental results.

\sphinxstylestrong{Solution 9b} \sphinxhyphen{} (L14/12)

\noindent\sphinxincludegraphics{{contents/questions/figs/Q9b_2019-2020}.png}

\sphinxstylestrong{Solution 9c}
(L13/16)

A no\sphinxhyphen{}flow boundary condition is special case of second type or Neumann boundary condition. For no flow condition head gradient is equated to zero, i.e., there is no gradient and thus no flow (water flows from high to low head). Mathematically, this is:

\(\frac{dh}{dx} = 0 \) for no\sphinxhyphen{}flow along \(x-\)axis, with \(h\) representing head.

Good Luck.







\renewcommand{\indexname}{Index}
\printindex
\end{document}